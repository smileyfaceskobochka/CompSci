\documentclass[oneside,a4paper,14pt]{extarticle}
\usepackage[a4paper,letterpaper,top=20mm,bottom=20mm,left=20mm,right=10mm]{geometry}
\usepackage[russian]{babel}
\usepackage{indentfirst}
\usepackage{graphicx}
\usepackage{caption}
\usepackage{titlesec}
\usepackage{minted, fancyvrb}
\usepackage{hyperref}
\usepackage{enumitem}
\usepackage{csvsimple-l3}
\usepackage{array}

% Форматирование листингов с кодом
\setminted{style = rainbow_dash, fontsize = \small} % https://pygments.org/styles/

% Форматирование заголовков
\titleformat{\section}{\normalsize\bfseries}{\thesection}{1em}{}
\titleformat{\subsection}{\normalsize\bfseries}{\thesubsection}{1em}{}
\titleformat{\subsubsection}{\normalsize\bfseries}{\thesubsubsection}{1em}{}

% Интерлиньяж и абзац
\renewcommand\baselinestretch{1.33}

\setlength{\parindent}{1.25cm}  % длина красной строки

% Для всех списков
\setlist[enumerate]{
  left=\parindent,       % отступ слева
  label=\arabic*.,       % цифры
  itemsep=0pt,           % расстояние между пунктами
  topsep=5pt,            % отступ сверху
  partopsep=0pt,         % дополнительный отступ сверху, если абзац до списка
  parsep=0pt             % отступ между абзацами внутри пункта
}

\setlist[itemize]{
  left=\parindent,       % отступ слева
  itemsep=0pt,           % расстояние между пунктами
  topsep=5pt,            % отступ сверху
  partopsep=0pt,
  parsep=0pt
}

% Гиперссылки
\hypersetup{
  colorlinks=true,
  linkcolor=black,
  urlcolor=blue,
  pdfborder={0 0 0},
  pdftitle={Основы DML-запросов в PostgreSQL. Заполнение данных и представления},
  pdfauthor={Черкасов А.А.}
}

\begin{document}

\newpage
\thispagestyle{empty}
\begin{center}
  МИНИСТЕРСТВО НАУКИ И ВЫСШЕГО ОБРАЗОВАНИЯ РОССИЙСКОЙ ФЕДЕРАЦИИ ФЕДЕРАЛЬНОЕ ГОСУДАРСТВЕННОЕ БЮДЖЕТНОЕ ОБРАЗОВАТЕЛЬНОЕ УЧРЕЖДЕНИЕ ВЫСШЕГО ОБРАЗОВАНИЯ\\
  «ВЯТСКИЙ ГОСУДАРСТВЕННЫЙ УНИВЕРСИТЕТ»\\
  Институт математики и информационных систем\\
  Факультет автоматики и вычислительной техники\\
  Кафедра электронных вычислительных машин
\end{center}
\vspace{10mm}

\hfill
\begin{tabular}{l}
  \footnotesize Дата сдачи на проверку:                                          \\
  \footnotesize <<\rule[-1mm]{5mm}{0.10mm}\/>>\rule[-1mm]{20mm}{0.10mm}\ 2025 г. \\
  \footnotesize Проверено:                                                       \\
  \footnotesize <<\rule[-1mm]{5mm}{0.10mm}\/>>\rule[-1mm]{20mm}{0.10mm}\ 2025 г. \\
\end{tabular}
\vfill

\begin{center}
  Основы DML-запросов в PostgreSQL.\\
  Отчёт по лабораторной работе №2.1\\
  по дисциплине\\
  <<Управление данными>>\\
\end{center}
\vspace{25mm}
\noindent
\begin{tabular}{ll}
  Разработал студент гр. ИВТб-2301-05-00 & \hspace{18mm}\rule[-1mm]{30mm}{0.10mm}\,/Черкасов А. А./ \\
                                         & \hspace{25.5mm}\footnotesize(подпись)                    \\
  Старший Преподователь                  & \hspace{18mm}\rule[-1mm]{30mm}{0.10mm}\,/Клюкин В. Л./   \\
                                         & \hspace{25.5mm}\footnotesize(подпись)                    \\
\end{tabular}

\noindent
\begin{tabular}{lp{58mm}r}
  Работа защищена &  & \hspace{13mm}<<\rule[-1mm]{5mm}{0.10mm}\/>>\rule[-1mm]{30mm}{0.10mm}\ 2025 г.
\end{tabular}
\vfill

\begin{center}
  Киров\\
  2025
\end{center}

\newpage\thispagestyle{plain}

\section*{Цели лабораторной работы}
\begin{itemize}
  \item[$-$] освоить основные варианты DML-запросов в PostgreSQL;
  \item[$-$] научиться создавать SQL-скрипты для заполнения таблиц данными;
  \item[$-$] научиться использованию команд \texttt{UPDATE} и \texttt{DELETE};
  \item[$-$] научиться работать с представлениями.
\end{itemize}

\section*{Задание}
\begin{enumerate}
  \item Создать и выполнить SQL-скрипт, который будет заполнять таблицы данными (не менее 3–5 строк в каждую таблицу).
  \item Создать представления для нескольких таблиц, в которых собираются данные из самой таблицы и других, на которые она ссылается. Хотя бы одно из представлений должно быть сделано с использованием соединений (\texttt{JOIN}).
  \item Для таблицы, содержащей столбец с числовыми данными, создать представление, отражающее статистику по этому столбцу (минимум, максимум, среднее, сумма).
\end{enumerate}

\clearpage
\section*{Тема БД: <<Система управления умным домом>>}
База данных предназначена для хранения информации о пользователях, хабах и устройствах умного дома, а также событий, которые генерируют эти устройства. Она обеспечивает:
\begin{itemize}
  \item[$-$] регистрацию и управление пользователями;
  \item[$-$] хранение информации о хабах и их местоположении;
  \item[$-$] классификацию устройств по типам и отслеживание их состояния;
  \item[$-$] фиксацию событий устройств для мониторинга и анализа.
\end{itemize}


\section*{Заполнение таблиц данными}

Для заполнения таблиц были использованы команды \texttt{INSERT}. Для обеспечения ссылочной целостности использовались возвращаемые значения первичных ключей через \texttt{RETURNING}. Ниже приведён фрагмент скрипта \texttt{init\_data.sql}.

\section*{Представления}

Были созданы три представления, описанные в файле \texttt{views.sql}.

\subsection*{1. Полная информация об устройствах}

Представление \texttt{device\_full\_info} объединяет данные из таблиц \texttt{devices}, \\
\texttt{device\_types}, \texttt{hubs} и \texttt{users}, предоставляя полную информацию об устройстве, его типе, хабе и владельце.

\begin{table}[H]
  \centering
  \csvreader[
    respect all,
    tabular=|c|c|c|c|c|,
    table head=\hline ID & Устройство & Тип & Хаб & Владелец \\ \hline,
    late after line=\\\hline
  ]{code/device_full_info.csv}{}{%
    \csvcoli & \csvcolii & \csvcoliii & \csvcoliv & \csvcolv
  }
  \caption{Полная информация об устройствах}
  \label{tab:device_full_info}
\end{table}

\subsection*{2. Информация о событиях}

Представление \texttt{event\_info} показывает события вместе с именем устройства и хаба, к которому оно относится.

\begin{table}[H]
  \centering
  \footnotesize
  \csvreader[
  respect all,
  tabular=|c|c|>{\ttfamily}c|c|c|,
  table head=\hline ID события & Тип события & \multicolumn{1}{c|}{Данные} & Устройство & Хаб \\ \hline,
  late after line=\\\hline
  ]{code/event_info.csv}{}{%
  \csvcoli & \csvcolii & \csvcoliii & \csvcoliv & \csvcolv
  }
  \caption{Информация о событиях}
  \label{tab:event_info}
\end{table}

\subsection*{3. Статистика по числовому столбцу}

Представление \texttt{device\_id\_stats} отображает минимальное, максимальное, среднее значение и сумму по столбцу \texttt{id} в таблице \texttt{devices}.

\begin{table}[H]
  \centering
  \csvreader[
    respect all,
    tabular=|l|c|c|,
    table head=\hline Метрика & Значение & ID записи \\ \hline,
    late after line=\\\hline
  ]{code/device_id_stats.csv}{}{%
    \csvcoli & \csvcolii & \csvcoliii
  }
  \caption{Статистика по идентификаторам устройств}
  \label{tab:device_id_stats}
\end{table}

\section*{Вывод}

В ходе выполнения лабораторной работы №2\_1 были освоены основные DML-запросы в PostgreSQL. Был создан и выполнен скрипт для заполнения таблиц тестовыми данными с соблюдением ссылочной целостности. Реализованы команды \texttt{UPDATE} и \texttt{DELETE} для изменения и удаления записей. Также были созданы три представления: два с использованием \texttt{JOIN} для получения полной информации об устройствах и событиях, и одно — для отображения статистики по числовому столбцу. Работа позволила закрепить навыки написания SQL-скриптов и подготовила основу для последующих лабораторных работ.

\newpage

\section*{Приложение А1. Исходный код}
\inputminted{Dockerfile}{../Containerfile}

\section*{Приложение А2. SQL-скрипты}
\inputminted{sql}{code/init_data.sql}

\section*{Приложение А3. Представления}
\inputminted{sql}{code/views.sql}

\end{document}
