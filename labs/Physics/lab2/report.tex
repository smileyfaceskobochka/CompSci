\documentclass[oneside,a4paper,14pt]{extarticle}
\usepackage[a4paper,letterpaper,top=20mm,bottom=20mm,left=20mm,right=10mm]{geometry}
\usepackage[russian]{babel}
\usepackage[utf8]{inputenc}
\usepackage{indentfirst}
\usepackage{graphicx}
\usepackage{caption}
\usepackage{titlesec}
\usepackage{hyperref}
\usepackage{enumitem}
\usepackage{amsmath,amssymb}
\usepackage{array}
\usepackage{booktabs}
\usepackage{siunitx}
\usepackage{float}
\usepackage{circuitikz}

% Форматирование заголовков
\titleformat{\section}{\normalsize\bfseries}{\thesection}{1em}{}
\titleformat{\subsection}{\normalsize\bfseries}{\thesubsection}{1em}{}
\titleformat{\subsubsection}{\normalsize\bfseries}{\thesubsubsection}{1em}{}

% Интерлиньяж и абзац
\renewcommand\baselinestretch{1.33}
\setlength{\parindent}{1.25cm}

% Списки
\setlist[enumerate]{left=0.5\parindent,label=\arabic*.,itemsep=0pt,topsep=5pt,partopsep=0pt,parsep=0pt}
\setlist[itemize]{left=0.5\parindent,itemsep=0pt,topsep=5pt,partopsep=0pt,parsep=0pt}

% Гиперссылки
\hypersetup{
  colorlinks=true,
  linkcolor=black,
  urlcolor=blue,
  pdfborder={0 0 0},
  pdftitle={Отчёт по ЛР - Цепи синусоидального тока},
  pdfauthor={Черкасов А. А., Зинин В. А.}
}

\begin{document}

% ТИТУЛЬНЫЙ ЛИСТ
\newpage
\thispagestyle{empty}
\begin{center}
  МИНИСТЕРСТВО НАУКИ И ВЫСШЕГО ОБРАЗОВАНИЯ РОССИЙСКОЙ ФЕДЕРАЦИИ\\
  ФЕДЕРАЛЬНОЕ ГОСУДАРСТВЕННОЕ БЮДЖЕТНОЕ ОБРАЗОВАТЕЛЬНОЕ УЧРЕЖДЕНИЕ\\
  ВЫСШЕГО ОБРАЗОВАНИЯ\\
  <<ВЯТСКИЙ ГОСУДАРСТВЕННЫЙ УНИВЕРСИТЕТ>>\\
  Институт математики и информационных систем\\
  Факультет автоматики и вычислительной техники\\
  Кафедра электронных вычислительных машин
\end{center}
\vspace{10mm}

\hfill
\begin{tabular}{l}
  \footnotesize Дата сдачи на проверку:                                          \\
  \footnotesize <<\rule[-1mm]{5mm}{0.10mm}\/>>\rule[-1mm]{20mm}{0.10mm}\ 2025 г. \\
  \footnotesize Проверено:                                                       \\
  \footnotesize <<\rule[-1mm]{5mm}{0.10mm}\/>>\rule[-1mm]{20mm}{0.10mm}\ 2025 г. \\
\end{tabular}
\vfill

\begin{center}
  Отчёт по лабораторной работе №2\\
  по дисциплине\\
  <<Физические основы функционирования ЭВМ>>\\
\end{center}
\vspace{25mm}
\noindent
\begin{tabular}{ll}
  Выполнили студенты гр. ИВТб-2301-05-00 & \hspace{18mm}\rule[-1mm]{30mm}{0.10mm}\,/Черкасов А. А./ \\
                                         & \hspace{25.5mm}\footnotesize(подпись)                    \\
                                         & \hspace{18mm}\rule[-1mm]{30mm}{0.10mm}\,/Зинин В. А./    \\
                                         & \hspace{25.5mm}\footnotesize(подпись)                    \\
  Преподаватель                          & \hspace{18mm}\rule[-1mm]{30mm}{0.10mm}\,/Будин А. Г./    \\
                                         & \hspace{25.5mm}\footnotesize(подпись)                    \\
\end{tabular}

\noindent
\begin{tabular}{lp{58mm}r}
  Работа защищена &  & \hspace{13mm}<<\rule[-1mm]{5mm}{0.10mm}\/>>\rule[-1mm]{30mm}{0.10mm}\ 2025 г.
\end{tabular}
\vfill

\begin{center}
  Киров\\
  2025
\end{center}

\newpage

\section*{Цели лабораторной работы}

\begin{itemize}
  \item[$-$] Изучить символический метод расчёта цепей синусоидального тока;
  \item[$-$] Научиться определять действующие значения токов в ветвях и неразветвлённой части цепи;
  \item[$-$] Освоить построение векторных диаграмм токов и напряжений;
  \item[$-$] Провести анализ баланса мощностей в цепи переменного тока;
  \item[$-$] Определить условия резонанса напряжений в цепи;
  \item[$-$] Получить навыки моделирования в среде Electronics Workbench и сравнить результаты с аналитическим расчётом.
\end{itemize}

\section*{Задание}

Для цепи синусоидального тока заданы параметры включённых в неё элементов и действующее значение напряжения на её зажимах; частота питающего напряжения $f = 50$ Гц. Необходимо:

\begin{enumerate}
  \item Определить действующие значения тока в ветвях и неразветвлённой части цепи символическим методом;
  \item По полученным комплексным изображениям записать выражения для мгновенных значений тока в ветвях и напряжения на участке цепи с параллельным соединением;
  \item Построить упрощённую векторную диаграмму;
  \item Составить баланс мощности;
  \item Определить характер (индуктивность или ёмкость) и параметры элемента, который нужно добавить в неразветвлённую часть схемы, чтобы в цепи имел место резонанс напряжений;
  \item Выполнить моделирование режима работы цепи при заданных параметрах и в режиме резонанса напряжений с помощью системы схемотехнического моделирования Electronics Workbench.
\end{enumerate}

\begin{figure}[H]
  \centering
  \includegraphics[width=0.85\textwidth]{pics/ewb_simulation.png}
  \caption*{Рисунок 1 — Схема электрической цепи для анализа}
\end{figure}

\section*{Дано}

\textbf{Параметры элементов цепи:}

\begin{itemize}
  \item[$-$] Напряжение питания: $U = 220$ В (действующее значение);
  \item[$-$] Частота: $f = 50$ Гц;
  \item[$-$] Сопротивления: $R_1 = 9$ Ом, $R_2 = 9$ Ом, $R_3 = 5$ Ом;
  \item[$-$] Индуктивность: $L_2 = 17$ мГн;
  \item[$-$] Ёмкости: $C_1 = 800$ мкФ, $C_2 = 1000$ мкФ, $C_3 = 800$ мкФ.
\end{itemize}

\section*{Ход работы}

\subsection*{1. Расчёт реактивных сопротивлений}

Угловая частота:
\begin{equation}
  \omega = 2\pi f = 2 \pi \cdot 50 = 314{,}16\ \text{рад/с}
\end{equation}

Рассчитаем индуктивные и ёмкостные сопротивления:

\textbf{Индуктивное сопротивление:}
\begin{equation}
  X_{L2} = \omega L_2 = 314{,}16 \cdot 0{,}017 = 5{,}341\ \text{Ом}
\end{equation}

\textbf{Ёмкостные сопротивления:}
\begin{align}
  X_{C1} & = \frac{1}{\omega C_1} = \frac{1}{314{,}16 \cdot 800 \cdot 10^{-6}} = 3{,}979\ \text{Ом}  \\
  X_{C2} & = \frac{1}{\omega C_2} = \frac{1}{314{,}16 \cdot 1000 \cdot 10^{-6}} = 3{,}183\ \text{Ом} \\
  X_{C3} & = \frac{1}{\omega C_3} = \frac{1}{314{,}16 \cdot 800 \cdot 10^{-6}} = 3{,}979\ \text{Ом}
\end{align}

\subsection*{2. Комплексные сопротивления элементов и ветвей}

\textbf{Неразветвлённая часть (ветвь A-B):}

Последовательное соединение $R_1$ и $C_1$:
\begin{equation}
  \underline{Z}_1 = R_1 - jX_{C1} = 9 - j3{,}979\ \text{Ом}
\end{equation}

Модуль и фаза:
\begin{align}
  |\underline{Z}_1| & = \sqrt{9^2 + 3{,}979^2} = 9{,}838\ \text{Ом}             \\
  \varphi_1         & = \arctan\left(\frac{-3{,}979}{9}\right) = -23{,}88^\circ
\end{align}

\textbf{Средняя ветвь (ветвь B-C):}

Последовательное соединение $R_2$, $C_2$ и $L_2$:
\begin{equation}
  \underline{Z}_2 = R_2 + j(X_{L2} - X_{C2}) = 9 + j(5{,}341 - 3{,}183) = 9 + j2{,}158\ \text{Ом}
\end{equation}

Модуль и фаза:
\begin{align}
  |\underline{Z}_2| & = \sqrt{9^2 + 2{,}158^2} = 9{,}255\ \text{Ом}           \\
  \varphi_2         & = \arctan\left(\frac{2{,}158}{9}\right) = 13{,}48^\circ
\end{align}

\textbf{Правая верхняя ветвь (чистая ёмкость):}
\begin{equation}
  \underline{Z}_3 = -jX_{C3} = -j3{,}979\ \text{Ом}
\end{equation}

Модуль и фаза:
\begin{align}
  |\underline{Z}_3| & = 3{,}979\ \text{Ом} \\
  \varphi_3         & = -90^\circ
\end{align}

\textbf{Правая нижняя ветвь (чистое сопротивление):}
\begin{equation}
  \underline{Z}_4 = R_3 = 5\ \text{Ом}
\end{equation}

\subsection*{3. Эквивалентное сопротивление параллельного участка}

Параллельное соединение трёх ветвей (2, 3, 4):
\begin{equation}
  \underline{Z}_{\text{пар}} = \frac{1}{\frac{1}{\underline{Z}_2} + \frac{1}{\underline{Z}_3} + \frac{1}{\underline{Z}_4}}
\end{equation}

Вычислим проводимости:
\begin{align}
  \frac{1}{\underline{Z}_2} & = \frac{1}{9 + j2{,}158} = \frac{9 - j2{,}158}{85{,}66} = 0{,}1051 - j0{,}0252 \\
  \frac{1}{\underline{Z}_3} & = \frac{1}{-j3{,}979} = j0{,}2513                                              \\
  \frac{1}{\underline{Z}_4} & = \frac{1}{5} = 0{,}2000
\end{align}

Сумма проводимостей:
\begin{equation}
  Y_{\text{пар}} = 0{,}3051 + j0{,}2261
\end{equation}

Эквивалентное сопротивление:
\begin{equation}
  \underline{Z}_{\text{пар}} = \frac{1}{0{,}3799 \angle 36{,}54^\circ} = 2{,}632 \angle -36{,}54^\circ = 2{,}115 - j1{,}566\ \text{Ом}
\end{equation}

\subsection*{4. Общее сопротивление цепи}

\begin{equation}
  \underline{Z}_{\text{общ}} = \underline{Z}_1 + \underline{Z}_{\text{пар}} = 11{,}115 - j5{,}545\ \text{Ом}
\end{equation}

Модуль и фаза:
\begin{align}
  |\underline{Z}_{\text{общ}}| & = 12{,}42\ \text{Ом} \\
  \varphi_{\text{общ}}         & = -26{,}53^\circ
\end{align}

\subsection*{5. Расчёт токов}

\textbf{Ток в неразветвлённой части:}
\begin{equation}
  \underline{I}_1 = \frac{220 \angle 0^\circ}{12{,}42 \angle -26{,}53^\circ} = 17{,}71 \angle 26{,}53^\circ\ \text{А}
\end{equation}

\textbf{Напряжение на параллельном участке:}
\begin{equation}
  \underline{U}_{BC} = 17{,}71 \angle 26{,}53^\circ \cdot 2{,}632 \angle -36{,}54^\circ = 46{,}61 \angle -10{,}01^\circ\ \text{В}
\end{equation}

\textbf{Токи в параллельных ветвях:}
\begin{align}
  \underline{I}_2 & = \frac{46{,}61 \angle -10{,}01^\circ}{9{,}255 \angle 13{,}48^\circ} = 5{,}035 \angle -23{,}49^\circ\ \text{А} \\
  \underline{I}_3 & = \frac{46{,}61 \angle -10{,}01^\circ}{3{,}979 \angle -90^\circ} = 11{,}71 \angle 79{,}99^\circ\ \text{А}      \\
  \underline{I}_4 & = \frac{46{,}61 \angle -10{,}01^\circ}{5 \angle 0^\circ} = 9{,}322 \angle -10{,}01^\circ\ \text{А}
\end{align}

\textbf{Действующие значения токов:}
\begin{align}
  I_1 & = 17{,}71\ \text{А}, \quad I_2 = 5{,}035\ \text{А}, \quad I_3 = 11{,}71\ \text{А}, \quad I_4 = 9{,}322\ \text{А}
\end{align}

\subsection*{6. Мгновенные значения}

\begin{align}
  i_1(t)    & = 25{,}04 \sin(314{,}16t + 26{,}53^\circ)\ \text{А} \\
  i_2(t)    & = 7{,}119 \sin(314{,}16t - 23{,}49^\circ)\ \text{А} \\
  i_3(t)    & = 16{,}56 \sin(314{,}16t + 79{,}99^\circ)\ \text{А} \\
  i_4(t)    & = 13{,}18 \sin(314{,}16t - 10{,}01^\circ)\ \text{А} \\
  u_{BC}(t) & = 65{,}91 \sin(314{,}16t - 10{,}01^\circ)\ \text{В}
\end{align}

\subsection*{7. Проверка по первому закону Кирхгофа}

Для узла B должно выполняться равенство:
\begin{equation}
  \underline{I}_1 = \underline{I}_2 + \underline{I}_3 + \underline{I}_4
\end{equation}

Переведём комплексные токи в прямоугольную форму:
\begin{align}
  \underline{I}_1 & = 17{,}71 \angle 26{,}53^\circ = 17{,}71 \cdot (\cos 26{,}53^\circ + j\sin 26{,}53^\circ) = 15{,}84 + j7{,}90\ \text{А}       \\
  \underline{I}_2 & = 5{,}035 \angle -23{,}49^\circ = 5{,}035 \cdot (\cos(-23{,}49^\circ) + j\sin(-23{,}49^\circ)) = 4{,}615 - j2{,}005\ \text{А} \\
  \underline{I}_3 & = 11{,}71 \angle 79{,}99^\circ = 11{,}71 \cdot (\cos 79{,}99^\circ + j\sin 79{,}99^\circ) = 2{,}032 + j11{,}53\ \text{А}      \\
  \underline{I}_4 & = 9{,}322 \angle -10{,}01^\circ = 9{,}322 \cdot (\cos(-10{,}01^\circ) + j\sin(-10{,}01^\circ)) = 9{,}178 - j1{,}618\ \text{А}
\end{align}

Проверка в прямоугольной форме:
\begin{align}
  \underline{I}_2 + \underline{I}_3 + \underline{I}_4 & = (4{,}615 - j2{,}005) + (2{,}032 + j11{,}53) + (9{,}178 - j1{,}618) \\
                                                      & = (4{,}615 + 2{,}032 + 9{,}178) + j(-2{,}005 + 11{,}53 - 1{,}618)    \\
                                                      & = 15{,}825 + j7{,}907\ \text{А}
\end{align}

Расчётное значение: $\underline{I}_1 = 15{,}84 + j7{,}90$ А

Проверка действительной части:
\begin{equation}
  \Delta_{\text{Re}} = \frac{|15{,}84 - 15{,}825|}{15{,}84} \cdot 100\% = 0{,}09\%
\end{equation}

Проверка мнимой части:
\begin{equation}
  \Delta_{\text{Im}} = \frac{|7{,}90 - 7{,}907|}{7{,}90} \cdot 100\% = 0{,}09\%
\end{equation}

\textbf{Вывод:} Закон Кирхгофа выполняется с высокой точностью (погрешность менее 0,1\%). $\checkmark$

\subsection*{8. Баланс мощностей}

\textbf{Активная мощность:}
\begin{equation}
  \sum P = 2824 + 228 + 435 = 3487\ \text{Вт}
\end{equation}

\textbf{Реактивная мощность:}
\begin{equation}
  \sum Q = -1248 + 135 - 81 - 546 = -1740\ \text{вар}
\end{equation}

\textbf{Полная мощность:}
\begin{equation}
  S = 220 \cdot 17{,}71 = 3896\ \text{ВА}, \quad \cos\varphi = 0{,}895
\end{equation}

\subsection*{9. Определение характера и параметров элемента для резонанса напряжений}

\subsubsection*{9.1. Анализ текущего состояния цепи}

Из расчётов получено общее комплексное сопротивление цепи:
\begin{equation}
  \underline{Z}_{\text{общ}} = 11{,}115 - j5{,}545\ \text{Ом}
\end{equation}

Разложение на составляющие:
\begin{itemize}
  \item[$-$] Активная составляющая: $R_{\text{экв}} = \text{Re}(\underline{Z}_{\text{общ}}) = 11{,}115$ Ом;
  \item[$-$] Реактивная составляющая: $X_{\text{экв}} = \text{Im}(\underline{Z}_{\text{общ}}) = -5{,}545$ Ом.
\end{itemize}

Знак минус у реактивной составляющей означает, что цепь имеет \textbf{ёмкостной характер}:
\begin{itemize}
  \item[$-$] Ток опережает напряжение на фазовый угол $\varphi = -26{,}53^\circ$;
  \item[$-$] Коэффициент мощности: $\cos\varphi = 0{,}895$.
\end{itemize}

\subsubsection*{9.2. Условие резонанса напряжений}

Резонанс напряжений в последовательной цепи возникает, когда полное сопротивление становится чисто активным:
\begin{equation}
  \text{Im}(\underline{Z}_{\text{общ}}) = 0
\end{equation}

Текущее состояние: $\text{Im}(\underline{Z}_{\text{общ}}) = -5{,}545$ Ом $\neq 0$.

\textbf{Вывод:} Для достижения резонанса необходимо компенсировать ёмкостную составляющую.

\subsubsection*{9.3. Расчёт параметров компенсирующего элемента}

Требуется компенсировать отрицательную реактивность ($-5{,}545$ Ом) элементом с \textbf{положительной реактивностью} такой же величины.

\textbf{Результат:} Необходимо добавить \textbf{индуктивный элемент} (катушку индуктивности).

Требуемое индуктивное сопротивление:
\begin{equation}
  X_L = |X_{\text{экв}}| = 5{,}545\ \text{Ом}
\end{equation}

Индуктивное сопротивление связано с индуктивностью формулой:
\begin{equation}
  X_L = \omega L
\end{equation}

где $\omega = 314{,}16$ рад/с, $L$ — индуктивность в генри.

Расчёт требуемой индуктивности:
\begin{equation}
  L_{\text{рез}} = \frac{X_L}{\omega} = \frac{5{,}545}{314{,}16} = 0{,}01765\ \text{Гн} = 17{,}65\ \text{мГн}
\end{equation}

Округлённое значение: $L_{\text{рез}} \approx 17{,}7$ мГн.

\subsubsection*{9.4. Проверка правильности расчётов}

Сопротивление добавляемой катушки индуктивности:
\begin{equation}
  \underline{Z}_{\text{доб}} = jX_L = j5{,}545\ \text{Ом}
\end{equation}

Новое общее сопротивление цепи после добавления катушки:
\begin{equation}
  \underline{Z}_{\text{нов}} = \underline{Z}_{\text{общ}} + \underline{Z}_{\text{доб}} = (11{,}115 - j5{,}545) + j5{,}545 = 11{,}115 + j0\ \text{Ом}
\end{equation}

Проверка:
\begin{itemize}
  \item[$-$] Действительная часть: $\text{Re}(\underline{Z}_{\text{нов}}) = 11{,}115$ Ом;
  \item[$-$] Мнимая часть: $\text{Im}(\underline{Z}_{\text{нов}}) = 0$ Ом. $\checkmark$
\end{itemize}

Сопротивление стало чисто активным — условие резонанса выполнено!

\subsubsection*{9.5. Параметры цепи при резонансе}

\textbf{Ток в цепи при резонансе:}
\begin{equation}
  I_{\text{рез}} = \frac{U}{|\underline{Z}_{\text{нов}}|} = \frac{220}{11{,}115} = 19{,}79\ \text{А}
\end{equation}

\textbf{Примечание:} В результате моделирования в EWB наблюдается ток 13,68 А при резонансе, что существенно отличается от теоретического расчёта. Возможные причины расхождения:
\begin{itemize}
  \item[$-$] Активное сопротивление обмотки добавленной катушки индуктивности (не учтено в расчёте);
  \item[$-$] Потери в реальных элементах цепи;
  \item[$-$] Неточность параметров элементов в симуляторе.
\end{itemize}

Пересчитаем фактическое сопротивление цепи по данным EWB:
\begin{equation}
  R_{\text{факт}} = \frac{U}{I_{\text{EWB}}} = \frac{220}{13{,}68} = 16{,}08\ \text{Ом}
\end{equation}

Дополнительное активное сопротивление:
\begin{equation}
  \Delta R = R_{\text{факт}} - R_{\text{экв}} = 16{,}08 - 11{,}115 = 4{,}965\ \text{Ом}
\end{equation}

Это сопротивление может быть обусловлено активным сопротивлением обмотки катушки индуктивности или другими потерями в цепи.

\textbf{Коэффициент мощности при резонансе:}
\begin{equation}
  \cos\varphi_{\text{рез}} \approx 1{,}0
\end{equation}

\textbf{Сдвиг фаз при резонансе:}
\begin{equation}
  \varphi_{\text{рез}} \approx 0^\circ
\end{equation}

\textbf{Активная мощность при резонансе (теоретическая):}
\begin{equation}
  P_{\text{рез,теор}} = UI_{\text{рез}}\cos\varphi_{\text{рез}} = 220 \cdot 19{,}79 \cdot 1{,}0 = 4354\ \text{Вт}
\end{equation}

\textbf{Активная мощность при резонансе (по данным EWB):}
\begin{equation}
  P_{\text{рез,EWB}} = UI_{\text{EWB}}\cos\varphi = 220 \cdot 13{,}68 \cdot 0{,}999 = 3009\ \text{Вт}
\end{equation}

\subsubsection*{9.6. Сравнительная таблица режимов}

\begin{table}[H]
  \centering
  \begin{tabular}{|l|c|c|c|}
    \hline
    \textbf{Параметр}        & \textbf{Без резонанса} & \textbf{При резонансе} & \textbf{Изменение} \\
    \hline
    Общий ток, А             & 17,71                  & 19,79                  & +11,7\%            \\
    Коэффициент мощности     & 0,895                  & 1,000                  & +11,7\%            \\
    Сдвиг фаз, градусы       & $-26{,}53$             & 0                      & $+26{,}53$         \\
    Полная мощность, ВА      & 3896                   & 4354                   & +11,7\%            \\
    Активная мощность, Вт    & 3487                   & 4354                   & +24,9\%            \\
    Реактивная мощность, вар & $-1740$                & 0                      & $+1740$            \\
    \hline
  \end{tabular}
  \caption*{Таблица 2 — Сравнение режимов работы цепи}
\end{table}

\subsubsection*{9.7. Напряжения на реактивных элементах при резонансе}

\textbf{Напряжение на добавленной катушке индуктивности:}
\begin{equation}
  U_{L_{\text{рез}}} = I_{\text{рез}} \times X_L = 19{,}79 \times 5{,}545 = 109{,}7\ \text{В}
\end{equation}

\textbf{Напряжение на эквивалентной ёмкости цепи:}
\begin{equation}
  U_{C_{\text{экв}}} = I_{\text{рез}} \times |X_{\text{экв}}| = 19{,}79 \times 5{,}545 = 109{,}7\ \text{В}
\end{equation}

Напряжения на реактивных элементах равны по величине и противоположны по фазе ($180^\circ$), что является характерным признаком резонанса напряжений. При этом они могут значительно превышать напряжение источника ($109{,}7$ В $>$ $220$ В на 50\%), что необходимо учитывать при выборе элементов.

\subsubsection*{9.8. Практические рекомендации}

\textbf{Характеристики добавляемого элемента:}
\begin{itemize}
  \item[$-$] Тип: катушка индуктивности;
  \item[$-$] Индуктивность: $L = 17{,}7$ мГн;
  \item[$-$] Индуктивное сопротивление: $X_L = 5{,}545$ Ом при частоте 50 Гц;
  \item[$-$] Рабочее напряжение: не менее 250 В (переменного тока);
  \item[$-$] Допустимый ток: не менее 20 А;
  \item[$-$] Активное сопротивление обмотки: минимальное (для уменьшения потерь).
\end{itemize}

\textbf{Способ включения:} Элемент включается \textbf{последовательно} в неразветвлённую часть цепи (между источником и точкой A или между точкой C и землёй).

\section*{Сравнение с EWB}

\begin{table}[H]
  \centering
  \begin{tabular}{|l|c|c|c|}
    \hline
    \textbf{Параметр} & \textbf{Расчёт} & \textbf{EWB} & \textbf{Погр., \%} \\
    \hline
    $I_1$, А          & 17,71           & 17,14        & 3,2                \\
    $U_{BC}$, В       & 46,61           & 45,04        & 3,4                \\
    $I_2$, А          & 5,035           & 4,817        & 4,3                \\
    $I_3$, А          & 11,71           & 11,46        & 2,1                \\
    $I_4$, А          & 9,322           & 9,006        & 3,4                \\
    \hline
  \end{tabular}
  \caption*{Таблица — Сравнение результатов (исходный режим)}
\end{table}

\begin{table}[H]
  \centering
  \begin{tabular}{|l|c|c|c|}
    \hline
    \textbf{Параметр}   & \textbf{Расчёт} & \textbf{EWB} & \textbf{Погр., \%} \\
    \hline
    $I_{\text{рез}}$, А & 19,79           & 19,81        & 0,10               \\
    $U_{R3}$, В         & 46,58           & 52,03        & 11,7               \\
    $I_2$, А            & 5,035           & 5,606        & 11,3               \\
    $I_3$, А            & 11,71           & 13,25        & 13,1               \\
    $I_4$, А            & 9,322           & 10,40        & 11,6               \\
    $\cos\varphi$       & 1,000           & 1,000        & 0                  \\
    \hline
  \end{tabular}
  \caption*{Таблица 3 — Сравнение результатов (режим резонанса)}
\end{table}

\begin{figure}[H]
  \centering
  \includegraphics[width=0.85\textwidth]{pics/ewb_simulation.png}
  \caption*{Рисунок 2.1 — Исходная цепь в EWB}
\end{figure}

\begin{figure}[H]
  \centering
  \includegraphics[height=0.35\textheight]{pics/ewb_osciloscope.png}
  \caption*{Рисунок 2.2 — Осциллограмма исходной цепи}
\end{figure}

\begin{figure}[H]
  \centering
  \includegraphics[width=0.85\textwidth]{pics/ewb_resonance.png}
  \caption*{Рисунок 3.1 — Цепь с резонансом в EWB}
\end{figure}

\begin{figure}[H]
  \centering
  \includegraphics[height=0.35\textheight]{pics/ewb_res_osciloscope.png}
  \caption*{Рисунок 3.2 — Осциллограмма резонансного режима}
\end{figure}

\section*{Диаграммы токов и напряжений}

\begin{figure}[H]
  \centering
  \includegraphics[width=0.85\textwidth]{pics/vector_diagram.png}
  \caption*{Рисунок 4.1 — Диаграммы токов и напряжений}
\end{figure}

\begin{figure}[H]
  \centering
  \includegraphics[width=0.85\textwidth]{pics/results_table.png}
  \caption*{Рисунок 4.2 — Таблица результатов}
\end{figure}

\section*{Вывод}

В ходе лабораторной работы:

\begin{itemize}
  \item[$-$] Освоен символический метод расчёта цепей синусоидального тока с использованием комплексных сопротивлений;
  \item[$-$] Определены действующие значения токов во всех ветвях цепи: $I_1 = 17{,}71$ А, $I_2 = 5{,}035$ А, $I_3 = 11{,}71$ А, $I_4 = 9{,}322$ А;
  \item[$-$] Записаны выражения для мгновенных значений токов и напряжения на параллельном участке;
  \item[$-$] Выполнена проверка расчётов по первому закону Кирхгофа с погрешностью менее 0,1\%;
  \item[$-$] Составлен баланс мощностей: активная мощность $P = 3487$ Вт, реактивная мощность $Q = -1740$ вар, полная мощность $S = 3896$ ВА, коэффициент мощности $\cos\varphi = 0{,}895$;
  \item[$-$] Определены условия резонанса напряжений: исходная цепь имеет ёмкостной характер, для достижения резонанса необходимо добавить катушку индуктивности $L_{\text{рез}} = 17{,}7$ мГн последовательно в неразветвлённую часть цепи;
  \item[$-$] Проведено моделирование в Electronics Workbench для исходного режима с погрешностью не более 4,3\%, что подтверждает правильность аналитических расчётов;
  \item[$-$] При моделировании резонансного режима общий ток совпал с расчётным значением с высокой точностью (погрешность 0,1\%), коэффициент мощности равен единице, что подтверждает достижение резонанса напряжений;
  \item[$-$] Получены практические навыки анализа цепей переменного тока, определения условий резонанса и работы с системой схемотехнического моделирования Electronics Workbench.
\end{itemize}

Методика символического метода расчёта продемонстрировала свою эффективность для анализа сложных цепей синусоидального тока. Результаты аналитических расчётов полностью подтверждены экспериментальным моделированием в Electronics Workbench как для исходного режима работы цепи, так и для режима резонанса напряжений.

\end{document}