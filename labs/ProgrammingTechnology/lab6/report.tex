\documentclass[oneside,a4paper,14pt]{extarticle}
\usepackage[a4paper,letterpaper,top=20mm,bottom=20mm,left=20mm,right=10mm]{geometry}
\usepackage[russian]{babel}
\usepackage{indentfirst}
\usepackage{graphicx}
\usepackage{caption}
\usepackage{titlesec}
\usepackage{minted, fancyvrb}
\usepackage{hyperref}
\usepackage{enumitem}

% Форматирование листингов
\setminted{style = rainbow_dash, fontsize = \small}

% Форматирование заголовков
\titleformat{\section}{\normalsize\bfseries}{\thesection}{1em}{}
\titleformat{\subsection}{\normalsize\bfseries}{\thesubsection}{1em}{}
\titleformat{\subsubsection}{\normalsize\bfseries}{\thesubsubsection}{1em}{}

\renewcommand\baselinestretch{1.33}
\setlength{\parindent}{1.25cm}

\setlist[enumerate]{
  left=\parindent,
  label=\arabic*.,,
  itemsep=0pt,
  topsep=5pt,
  partopsep=0pt,
  parsep=0pt
}

\setlist[itemize]{
  left=\parindent,
  itemsep=0pt,
  topsep=5pt,
  partopsep=0pt,
  parsep=0pt
}

\hypersetup{
  colorlinks=true,
  linkcolor=black,
  urlcolor=blue,
  pdfborder={0 0 0},
  pdftitle={Мобильное приложение с базой данных},
  pdfauthor={Черкасов А.А.}
}

\begin{document}

\newpage
\thispagestyle{empty}
\begin{center}
	МИНИСТЕРСТВО НАУКИ И ВЫСШЕГО ОБРАЗОВАНИЯ РОССИЙСКОЙ ФЕДЕРАЦИИ\\
	ФЕДЕРАЛЬНОЕ ГОСУДАРСТВЕННОЕ БЮДЖЕТНОЕ ОБРАЗОВАТЕЛЬНОЕ УЧРЕЖДЕНИЕ ВЫСШЕГО ОБРАЗОВАНИЯ\\[1mm]
	«ВЯТСКИЙ ГОСУДАРСТВЕННЫЙ УНИВЕРСИТЕТ»\\
	Институт математики и информационных систем\\
	Факультет автоматики и вычислительной техники\\
	Кафедра электронных вычислительных машин
\end{center}

\vspace{10mm}

\hfill
\begin{tabular}{l}
	\footnotesize Дата сдачи на проверку:                                          \\
	\footnotesize <<\rule[-1mm]{5mm}{0.10mm}\/>>\rule[-1mm]{20mm}{0.10mm}\ 2025 г. \\
	\footnotesize Проверено:                                                       \\
	\footnotesize <<\rule[-1mm]{5mm}{0.10mm}\/>>\rule[-1mm]{20mm}{0.10mm}\ 2025 г. \\
\end{tabular}

\vfill

\begin{center}
	Разработка мобильного приложения с базой данных.\\[1mm]
	Отчёт по лабораторной работе №6\\
	по дисциплине\\
	«Технологии программирования»\\
\end{center}

\vspace{25mm}

\noindent
\begin{tabular}{ll}
	Разработал студент гр. ИВТб-2301-05-00 & \hspace{18mm}\rule[-1mm]{30mm}{0.10mm}\,/Черкасов А. А./ \\
	                                       & \hspace{25.5mm}\footnotesize(подпись)                    \\
	Преподаватель                          & \hspace{18mm}\rule[-1mm]{30mm}{0.10mm}\,/Пащенко Д. Э./  \\
	                                       & \hspace{25.5mm}\footnotesize(подпись)                    \\
\end{tabular}

\vfill

\begin{center}
	Киров\\
	2025
\end{center}

\newpage
\thispagestyle{plain}

\section*{Цели лабораторной работы}

\begin{itemize}
	\item[$-$] освоить практические навыки создания мобильных приложений с локальной базой данных;
	\item[$-$] познакомиться с основами работы с SQLite в Flutter;
	\item[$-$] изучить паттерн управления состоянием Riverpod;
	\item[$-$] понять принципы построения многооконных приложений с CRUD-операциями.
\end{itemize}

\section*{Задание}

Разработать многооконное мобильное приложение, которое хранит информацию в локальной базе данных SQLite и позволяет изменять её. Тема приложения: «Моя Библиотека» — управление коллекцией книг и авторов.

\clearpage
\section*{Реализация приложения}

В рамках лабораторной работы разработано Flutter-приложение «Моя Библиотека», предназначенное для управления коллекцией книг и авторов с использованием локальной базы данных SQLite.

\subsection*{Структура приложения}

Приложение состоит из трёх основных экранов, доступных через нижнюю навигационную панель:

\begin{itemize}
	\item[$-$] \textbf{Добавить} — экран для добавления новых книг и авторов;
	\item[$-$] \textbf{Книги} — список всех книг с возможностью редактирования и удаления;
	\item[$-$] \textbf{Авторы} — список всех авторов с возможностью удаления.
\end{itemize}

\subsection*{База данных}

Приложение использует SQLite базу данных с двумя таблицами:

\begin{enumerate}
	\item \textbf{books} — хранит информацию о книгах:
	\begin{itemize}
		\item \texttt{id} (INTEGER PRIMARY KEY) — уникальный идентификатор;
		\item \texttt{title} (TEXT NOT NULL) — название книги;
		\item \texttt{publication\_year} (INTEGER NOT NULL) — год издания;
		\item \texttt{genre} (TEXT NOT NULL) — жанр;
		\item \texttt{isbn} (TEXT NOT NULL) — ISBN книги.
	\end{itemize}

	\item \textbf{authors} — хранит информацию об авторах:
	\begin{itemize}
		\item \texttt{id} (INTEGER PRIMARY KEY) — уникальный идентификатор;
		\item \texttt{book\_id} (INTEGER, FOREIGN KEY) — связь с книгой;
		\item \texttt{first\_name} (TEXT NOT NULL) — имя автора;
		\item \texttt{last\_name} (TEXT NOT NULL) — фамилия автора;
		\item \texttt{nationality} (TEXT NOT NULL) — национальность;
		\item \texttt{birth\_year} (INTEGER NOT NULL) — год рождения.
	\end{itemize}
\end{enumerate}

\subsection*{Управление состоянием}

Для управления состоянием приложения используется Riverpod — декларативный фреймворк для управления состоянием:

\begin{itemize}
	\item[$-$] \texttt{booksProvider} — управление списком книг;
	\item[$-$] \texttt{authorsProvider} — управление списком авторов;
	\item[$-$] \texttt{fastDeleteProvider} — настройка быстрого удаления.
\end{itemize}

\subsection*{CRUD-операции}

Приложение поддерживает полный набор CRUD-операций:

\begin{itemize}
	\item[$-$] \textbf{Create} — добавление новых книг и авторов через формы ввода;
	\item[$-$] \textbf{Read} — отображение списков книг и авторов;
	\item[$-$] \textbf{Update} — редактирование информации о книгах;
	\item[$-$] \textbf{Delete} — удаление книг и авторов с подтверждением.
\end{itemize}

\subsection*{Пользовательский интерфейс}

Интерфейс построен с использованием Material Design 3:

\begin{itemize}
	\item[$-$] \textbf{Навигация} — нижняя панель с тремя вкладками;
	\item[$-$] \textbf{Формы} — валидированные формы для ввода данных;
	\item[$-$] \textbf{Списки} — карточки с информацией и кнопками действий;
	\item[$-$] \textbf{Диалоги} — модальные окна для редактирования и подтверждения удаления.
\end{itemize}

\subsection*{Скриншоты работы приложения}

\begin{figure}[H]
	\centering
	\includegraphics[height=0.7\textheight]{pics/home_screen.png}
	\caption*{Рисунок 1 --- Главный экран приложения}
\end{figure}

\begin{figure}[H]
	\centering
	\includegraphics[height=0.7\textheight]{pics/books_screen.png}
	\caption*{Рисунок 2 --- Экран списка книг}
\end{figure}

\begin{figure}[H]
	\centering
	\includegraphics[height=0.7\textheight]{pics/authors_screen.png}
	\caption*{Рисунок 3 --- Экран списка авторов}
\end{figure}

\section*{Контрольные вопросы}

\subsection*{1. Что такое SQLite?}

SQLite — это встраиваемая реляционная база данных, которая не требует отдельного серверного процесса. Она хранит всю базу данных в одном файле на диске и предоставляет полный набор SQL-функций.

\subsection*{2. Преимущества SQLite в мобильной разработке}

\begin{itemize}
	\item[$-$] Не требует установки сервера;
	\item[$-$] Хранит данные в одном файле;
	\item[$-$] Поддерживает ACID-транзакции;
	\item[$-$] Имеет небольшой размер библиотеки;
	\item[$-$] Поддерживает большинство SQL-стандартов.
\end{itemize}

\subsection*{3. Что такое Riverpod?}

Riverpod — это фреймворк для управления состоянием во Flutter, предоставляющий декларативный подход к управлению зависимостями и состоянием приложения. Он является развитием Provider с улучшенной типобезопасностью и API.

\subsection*{4. Основные компоненты Riverpod}

\begin{itemize}
	\item[$-$] \textbf{Provider} — контейнер для хранения состояния или зависимостей;
	\item[$-$] \textbf{ConsumerWidget/ConsumerStatefulWidget} — виджеты, которые могут читать провайдеры;
	\item[$-$] \textbf{WidgetRef} — объект для чтения и изменения состояния провайдеров;
	\item[$-$] \textbf{Notifier} — классы для управления изменяемым состоянием.
\end{itemize}

\subsection*{5. Что такое CRUD?}

CRUD — акроним, обозначающий четыре основные операции с данными в базе данных:
\begin{itemize}
	\item[$-$] \textbf{Create} — создание новых записей;
	\item[$-$] \textbf{Read} — чтение/получение данных;
	\item[$-$] \textbf{Update} — обновление существующих записей;
	\item[$-$] \textbf{Delete} — удаление записей.
\end{itemize}

\subsection*{6. Внешние ключи в SQLite}

Внешний ключ (FOREIGN KEY) — это поле в таблице, которое ссылается на первичный ключ другой таблицы. Он обеспечивает целостность данных и позволяет устанавливать связи между таблицами. В SQLite внешние ключи включаются с помощью \texttt{PRAGMA foreign\_keys = ON}.

\subsection*{7. Асинхронное программирование в Flutter}

Flutter использует \texttt{async/await} для работы с асинхронными операциями. Ключевые слова:
\begin{itemize}
	\item[$-$] \texttt{async} — помечает функцию как асинхронную;
	\item[$-$] \texttt{await} — ожидает завершения асинхронной операции;
	\item[$-$] \texttt{Future} — тип для представления асинхронного результата.
\end{itemize}

\subsection*{8. Валидация форм во Flutter}

Валидация форм осуществляется через callback \texttt{validator} в \texttt{TextFormField}. Функция валидации возвращает строку с ошибкой или \texttt{null}, если данные корректны. Для запуска валидации используется метод \texttt{FormState.validate()}.

\section*{Вывод}

В ходе выполнения лабораторной работы №6 было разработано полнофункциональное мобильное приложение «Моя Библиотека» с использованием локальной базы данных SQLite.  
Реализован полный набор CRUD-операций для управления книгами и авторами.  
Изучены принципы работы с SQLite в Flutter, включая создание таблиц, внешние ключи и асинхронные операции.  
Освоен паттерн управления состоянием Riverpod для декларативного управления данными приложения.  
Приложение демонстрирует современные подходы к разработке мобильных приложений с персистентным хранением данных.

\newpage

\section*{Приложение А. Исходный код main.dart}
\inputminted{dart}{code/lib/main.dart}

\newpage

\section*{Приложение А1. Исходный код database.dart}
\inputminted{dart}{code/lib/database.dart}

\newpage

\section*{Приложение А2. Исходный код error\_handle.dart}
\inputminted{dart}{code/lib/error_handle.dart}

\newpage

\section*{Приложение А3. Исходный код providers.dart}
\inputminted{dart}{code/lib/providers.dart}

\newpage

\section*{Приложение А4. Исходный код screens/authors\_page.dart}
\inputminted{dart}{code/lib/screens/authors_page.dart}

\newpage

\section*{Приложение А5. Исходный код screens/books\_page.dart}
\inputminted{dart}{code/lib/screens/books_page.dart}

\newpage

\section*{Приложение А6. Исходный код screens/home\_page.dart}
\inputminted{dart}{code/lib/screens/home_page.dart}

\newpage

\section*{Приложение А7. Исходный код widgets/author\_adding\_window.dart}
\inputminted{dart}{code/lib/widgets/author_adding_window.dart}

\newpage

\section*{Приложение А8. Исходный код widgets/book\_adding\_window.dart}
\inputminted{dart}{code/lib/widgets/book_adding_window.dart}

\newpage

\section*{Приложение А9. Исходный код widgets/book\_redacting\_popup.dart}
\inputminted{dart}{code/lib/widgets/book_redacting_popup.dart}

\newpage

\section*{Приложение А10. Исходный код utils/confirm\_delete\_popup.dart}
\inputminted{dart}{code/lib/utils/confirm_delete_popup.dart}

\newpage

\section*{Приложение А11. Исходный код validators/author\_validator.dart}
\inputminted{dart}{code/lib/validators/author_validator.dart}

\newpage

\section*{Приложение А12. Исходный код validators/book\_validator.dart}
\inputminted{dart}{code/lib/validators/book_validator.dart}

\end{document}
