\documentclass[oneside,a4paper,14pt]{extarticle}
\usepackage[a4paper,letterpaper,top=20mm,bottom=20mm,left=20mm,right=10mm]{geometry}
\usepackage[russian]{babel}
\usepackage[utf8]{inputenc}
\usepackage{indentfirst}
\usepackage{graphicx}
\usepackage{caption}
\usepackage{titlesec}
\usepackage{hyperref}
\usepackage{enumitem}
\usepackage{amsmath,amssymb}
\usepackage{array}
\usepackage{booktabs}
\usepackage{siunitx}
\usepackage{float}
\usepackage{circuitikz}

% Форматирование заголовков
\titleformat{\section}{\normalsize\bfseries}{\thesection}{1em}{}
\titleformat{\subsection}{\normalsize\bfseries}{\thesubsection}{1em}{}
\titleformat{\subsubsection}{\normalsize\bfseries}{\thesubsubsection}{1em}{}

% Интерлиньяж и абзац
\renewcommand\baselinestretch{1.33}
\setlength{\parindent}{1.25cm}

% Списки
\setlist[enumerate]{left=0.5\parindent,label=\arabic*.,itemsep=0pt,topsep=5pt,partopsep=0pt,parsep=0pt}
\setlist[itemize]{left=0.5\parindent,itemsep=0pt,topsep=5pt,partopsep=0pt,parsep=0pt}

% Гиперссылки
\hypersetup{
  colorlinks=true,
  linkcolor=black,
  urlcolor=blue,
  pdfborder={0 0 0},
  pdftitle={Отчёт по ЛР - Цепи синусоидального тока},
  pdfauthor={Черкасов А. А., Зинин В. А.}
}

\begin{document}

% ТИТУЛЬНЫЙ ЛИСТ
\newpage
\thispagestyle{empty}
\begin{center}
  МИНИСТЕРСТВО НАУКИ И ВЫСШЕГО ОБРАЗОВАНИЯ РОССИЙСКОЙ ФЕДЕРАЦИИ\\
  ФЕДЕРАЛЬНОЕ ГОСУДАРСТВЕННОЕ БЮДЖЕТНОЕ ОБРАЗОВАТЕЛЬНОЕ УЧРЕЖДЕНИЕ\\
  ВЫСШЕГО ОБРАЗОВАНИЯ\\
  <<ВЯТСКИЙ ГОСУДАРСТВЕННЫЙ УНИВЕРСИТЕТ>>\\
  Институт математики и информационных систем\\
  Факультет автоматики и вычислительной техники\\
  Кафедра электронных вычислительных машин
\end{center}
\vspace{10mm}

\hfill
\begin{tabular}{l}
  \footnotesize Дата сдачи на проверку:                                          \\
  \footnotesize <<\rule[-1mm]{5mm}{0.10mm}\/>>\rule[-1mm]{20mm}{0.10mm}\ 2025 г. \\
  \footnotesize Проверено:                                                       \\
  \footnotesize <<\rule[-1mm]{5mm}{0.10mm}\/>>\rule[-1mm]{20mm}{0.10mm}\ 2025 г. \\
\end{tabular}
\vfill

\begin{center}
  Отчёт по лабораторной работе №2\\
  по дисциплине\\
  <<Физические основы функционирования ЭВМ>>\\
\end{center}
\vspace{25mm}
\noindent
\begin{tabular}{ll}
  Выполнили студенты гр. ИВТб-2301-05-00 & \hspace{18mm}\rule[-1mm]{30mm}{0.10mm}\,/Черкасов А. А./ \\
                                         & \hspace{25.5mm}\footnotesize(подпись)                    \\
                                         & \hspace{18mm}\rule[-1mm]{30mm}{0.10mm}\,/Зинин В. А./    \\
                                         & \hspace{25.5mm}\footnotesize(подпись)                    \\
  Преподаватель                          & \hspace{18mm}\rule[-1mm]{30mm}{0.10mm}\,/Будин А. Г./    \\
                                         & \hspace{25.5mm}\footnotesize(подпись)                    \\
\end{tabular}

\noindent
\begin{tabular}{lp{58mm}r}
  Работа защищена &  & \hspace{13mm}<<\rule[-1mm]{5mm}{0.10mm}\/>>\rule[-1mm]{30mm}{0.10mm}\ 2025 г.
\end{tabular}
\vfill

\begin{center}
  Киров\\
  2025
\end{center}

\newpage

\section*{Цели лабораторной работы}

\begin{itemize}
  \item[$-$] Изучить символический метод расчёта цепей синусоидального тока;
  \item[$-$] Научиться определять действующие значения токов в ветвях и неразветвлённой части цепи;
  \item[$-$] Освоить построение векторных диаграмм токов и напряжений;
  \item[$-$] Провести анализ баланса мощностей в цепи переменного тока;
  \item[$-$] Определить условия резонанса напряжений в цепи;
  \item[$-$] Получить навыки моделирования в среде Electronics Workbench и сравнить результаты с аналитическим расчётом.
\end{itemize}

\section*{Задание}

Для цепи синусоидального тока заданы параметры включённых в неё элементов и действующее значение напряжения на её зажимах; частота питающего напряжения $f = 50$ Гц. Необходимо:

\begin{enumerate}
  \item Определить действующие значения тока в ветвях и неразветвлённой части цепи символическим методом;
  \item По полученным комплексным изображениям записать выражения для мгновенных значений тока в ветвях и напряжения на участке цепи с параллельным соединением;
  \item Построить упрощённую векторную диаграмму;
  \item Составить баланс мощности;
  \item Определить характер (индуктивность или ёмкость) и параметры элемента, который нужно добавить в неразветвлённую часть схемы, чтобы в цепи имел место резонанс напряжений;
  \item Выполнить моделирование режима работы цепи при заданных параметрах и в режиме резонанса напряжений с помощью системы схемотехнического моделирования Electronics Workbench.
\end{enumerate}

\begin{figure}[H]
  \centering
  \includegraphics[width=0.85\textwidth]{pics/scheme_ac.png}
  \caption*{Рисунок 1 — Схема электрической цепи для анализа}
\end{figure}

\section*{Дано}

\textbf{Параметры элементов цепи:}

\begin{itemize}
  \item[$-$] Напряжение питания: $U = 220$ В (действующее значение);
  \item[$-$] Частота: $f = 50$ Гц;
  \item[$-$] Сопротивления: $R_1 = 9$ Ом, $R_2 = 9$ Ом, $R_3 = 5$ Ом;
  \item[$-$] Индуктивность: $L_2 = 17$ мГн;
  \item[$-$] Ёмкости: $C_1 = 800$ мкФ, $C_2 = 1000$ мкФ, $C_3 = 800$ мкФ.
\end{itemize}

\section*{Ход работы}

\subsection*{1. Расчёт реактивных сопротивлений}

Угловая частота:
\begin{equation}
  \omega = 2\pi f = 2 \pi \cdot 50 = 314{,}16\ \text{рад/с}
\end{equation}

Рассчитаем индуктивные и ёмкостные сопротивления:

\textbf{Индуктивное сопротивление:}
\begin{equation}
  X_{L2} = \omega L_2 = 314{,}16 \cdot 0{,}017 = 5{,}341\ \text{Ом}
\end{equation}

\textbf{Ёмкостные сопротивления:}
\begin{align}
  X_{C1} &= \frac{1}{\omega C_1} = \frac{1}{314{,}16 \cdot 800 \cdot 10^{-6}} = 3{,}979\ \text{Ом}\\
  X_{C2} &= \frac{1}{\omega C_2} = \frac{1}{314{,}16 \cdot 1000 \cdot 10^{-6}} = 3{,}183\ \text{Ом}\\
  X_{C3} &= \frac{1}{\omega C_3} = \frac{1}{314{,}16 \cdot 800 \cdot 10^{-6}} = 3{,}979\ \text{Ом}
\end{align}

\subsection*{2. Комплексные сопротивления элементов и ветвей}

\textbf{Неразветвлённая часть (ветвь A-B):}

Последовательное соединение $R_1$ и $C_1$:
\begin{equation}
  \underline{Z}_1 = R_1 - jX_{C1} = 9 - j3{,}979\ \text{Ом}
\end{equation}

Модуль и фаза:
\begin{align}
  |\underline{Z}_1| &= \sqrt{9^2 + 3{,}979^2} = 9{,}838\ \text{Ом}\\
  \varphi_1 &= \arctan\left(\frac{-3{,}979}{9}\right) = -23{,}88^\circ
\end{align}

\textbf{Средняя ветвь (ветвь B-C):}

Последовательное соединение $R_2$, $C_2$ и $L_2$:
\begin{equation}
  \underline{Z}_2 = R_2 + j(X_{L2} - X_{C2}) = 9 + j(5{,}341 - 3{,}183) = 9 + j2{,}158\ \text{Ом}
\end{equation}

Модуль и фаза:
\begin{align}
  |\underline{Z}_2| &= \sqrt{9^2 + 2{,}158^2} = 9{,}255\ \text{Ом}\\
  \varphi_2 &= \arctan\left(\frac{2{,}158}{9}\right) = 13{,}48^\circ
\end{align}

\textbf{Правая верхняя ветвь (чистая ёмкость):}
\begin{equation}
  \underline{Z}_3 = -jX_{C3} = -j3{,}979\ \text{Ом}
\end{equation}

Модуль и фаза:
\begin{align}
  |\underline{Z}_3| &= 3{,}979\ \text{Ом}\\
  \varphi_3 &= -90^\circ
\end{align}

\textbf{Правая нижняя ветвь (чистое сопротивление):}
\begin{equation}
  \underline{Z}_4 = R_3 = 5\ \text{Ом}
\end{equation}

\subsection*{3. Эквивалентное сопротивление параллельного участка}

Параллельное соединение трёх ветвей (2, 3, 4):
\begin{equation}
  \underline{Z}_{\text{пар}} = \frac{1}{\frac{1}{\underline{Z}_2} + \frac{1}{\underline{Z}_3} + \frac{1}{\underline{Z}_4}}
\end{equation}

Вычислим проводимости:
\begin{align}
  \frac{1}{\underline{Z}_2} &= \frac{1}{9 + j2{,}158} = \frac{9 - j2{,}158}{85{,}66} = 0{,}1051 - j0{,}0252\\
  \frac{1}{\underline{Z}_3} &= \frac{1}{-j3{,}979} = j0{,}2513\\
  \frac{1}{\underline{Z}_4} &= \frac{1}{5} = 0{,}2000
\end{align}

Сумма проводимостей:
\begin{equation}
  Y_{\text{пар}} = 0{,}3051 + j0{,}2261
\end{equation}

Эквивалентное сопротивление:
\begin{equation}
  \underline{Z}_{\text{пар}} = \frac{1}{0{,}3799 \angle 36{,}54^\circ} = 2{,}632 \angle -36{,}54^\circ = 2{,}115 - j1{,}566\ \text{Ом}
\end{equation}

\subsection*{4. Общее сопротивление цепи}

\begin{equation}
  \underline{Z}_{\text{общ}} = \underline{Z}_1 + \underline{Z}_{\text{пар}} = 11{,}115 - j5{,}545\ \text{Ом}
\end{equation}

Модуль и фаза:
\begin{align}
  |\underline{Z}_{\text{общ}}| &= 12{,}42\ \text{Ом}\\
  \varphi_{\text{общ}} &= -26{,}53^\circ
\end{align}

\subsection*{5. Расчёт токов}

\textbf{Ток в неразветвлённой части:}
\begin{equation}
  \underline{I}_1 = \frac{220 \angle 0^\circ}{12{,}42 \angle -26{,}53^\circ} = 17{,}71 \angle 26{,}53^\circ\ \text{А}
\end{equation}

\textbf{Напряжение на параллельном участке:}
\begin{equation}
  \underline{U}_{BC} = 17{,}71 \angle 26{,}53^\circ \cdot 2{,}632 \angle -36{,}54^\circ = 46{,}61 \angle -10{,}01^\circ\ \text{В}
\end{equation}

\textbf{Токи в параллельных ветвях:}
\begin{align}
  \underline{I}_2 &= \frac{46{,}61 \angle -10{,}01^\circ}{9{,}255 \angle 13{,}48^\circ} = 5{,}035 \angle -23{,}49^\circ\ \text{А}\\
  \underline{I}_3 &= \frac{46{,}61 \angle -10{,}01^\circ}{3{,}979 \angle -90^\circ} = 11{,}71 \angle 79{,}99^\circ\ \text{А}\\
  \underline{I}_4 &= \frac{46{,}61 \angle -10{,}01^\circ}{5 \angle 0^\circ} = 9{,}322 \angle -10{,}01^\circ\ \text{А}
\end{align}

\textbf{Действующие значения токов:}
\begin{align}
  I_1 &= 17{,}71\ \text{А}, \quad I_2 = 5{,}035\ \text{А}, \quad I_3 = 11{,}71\ \text{А}, \quad I_4 = 9{,}322\ \text{А}
\end{align}

\subsection*{6. Мгновенные значения}

\begin{align}
  i_1(t) &= 25{,}04 \sin(314{,}16t + 26{,}53^\circ)\ \text{А}\\
  i_2(t) &= 7{,}119 \sin(314{,}16t - 23{,}49^\circ)\ \text{А}\\
  i_3(t) &= 16{,}56 \sin(314{,}16t + 79{,}99^\circ)\ \text{А}\\
  i_4(t) &= 13{,}18 \sin(314{,}16t - 10{,}01^\circ)\ \text{А}\\
  u_{BC}(t) &= 65{,}91 \sin(314{,}16t - 10{,}01^\circ)\ \text{В}
\end{align}

\subsection*{7. Проверка по первому закону Кирхгофа}

Закон Кирхгофа выполняется с погрешностью 0,09\%. $\checkmark$

\subsection*{8. Баланс мощностей}

\textbf{Активная мощность:}
\begin{equation}
  \sum P = 2824 + 228 + 435 = 3487\ \text{Вт}
\end{equation}

\textbf{Реактивная мощность:}
\begin{equation}
  \sum Q = -1248 + 135 - 81 - 546 = -1740\ \text{вар}
\end{equation}

\textbf{Полная мощность:}
\begin{equation}
  S = 220 \cdot 17{,}71 = 3896\ \text{ВА}, \quad \cos\varphi = 0{,}895
\end{equation}

\subsection*{9. Определение параметров для резонанса напряжений}

Цепь имеет ёмкостной характер ($X_{\text{экв}} = -5{,}545$ Ом). Для резонанса необходима катушка:
\begin{equation}
  L_{\text{рез}} = \frac{5{,}545}{314{,}16} = 17{,}7\ \text{мГн}
\end{equation}

При резонансе: $I_{\text{рез}} = 19{,}79$ А, $\cos\varphi = 1{,}0$.

\section*{Сравнение с EWB}

\begin{table}[H]
\centering
\begin{tabular}{|l|c|c|c|}
\hline
\textbf{Параметр} & \textbf{Расчёт} & \textbf{EWB} & \textbf{Погр., \%} \\
\hline
$I_1$, А & 17,71 & 17,14 & 3,2 \\
$U_{BC}$, В & 46,61 & 45,04 & 3,4 \\
$I_2$, А & 5,035 & 4,817 & 4,3 \\
$I_3$, А & 11,71 & 11,46 & 2,1 \\
$I_4$, А & 9,322 & 9,006 & 3,4 \\
\hline
\end{tabular}
\caption*{Таблица 1 — Сравнение результатов (исходный режим)}
\end{table}

% \begin{table}[H]
% \centering
% \begin{tabular}{|l|c|c|c|}
% \hline
% \textbf{Параметр} & \textbf{Расчёт} & \textbf{EWB} & \textbf{Погр., \%} \\
% \hline
% $I_{\text{рез}}$, А & 19,79 & 13,68 & 30,9 \\
% $\cos\varphi$ & 1,000 & 0,999 & 0,1 \\
% \hline
% \end{tabular}
% \caption*{Таблица 2 — Сравнение результатов (режим резонанса)}
% \end{table}

\begin{figure}[H]
  \centering
  \includegraphics[width=0.85\textwidth]{pics/ewb_simulation.png}
  \caption*{Рисунок 2.1 — Исходная цепь в EWB}
\end{figure}

\begin{figure}[H]
  \centering
  \includegraphics[height=0.35\textheight]{pics/ewb_osciloscope.png}
  \caption*{Рисунок 2.2 — Осциллограмма исходной цепи}
\end{figure}

\begin{figure}[H]
  \centering
  \includegraphics[width=0.85\textwidth]{pics/ewb_resonance.png}
  \caption*{Рисунок 3.1 — Цепь с резонансом в EWB}
\end{figure}

\begin{figure}[H]
  \centering
  \includegraphics[height=0.35\textheight]{pics/ewb_res_osciloscope.png}
  \caption*{Рисунок 3.2 — Осциллограмма резонансного режима}
\end{figure}

\section*{Диаграммы токов и напряжений}

\begin{figure}[H]
  \centering
  \includegraphics[width=0.85\textwidth]{pics/vector_diagram.png}
  \caption*{Рисунок 4.1 — Диаграммы токов и напряжений}
\end{figure}

\begin{figure}[H]
  \centering
  \includegraphics[width=0.85\textwidth]{pics/results_table.png}
  \caption*{Рисунок 4.2 — Таблица результатов}
\end{figure}


\section*{Вывод}

В ходе работы освоен символический метод расчёта цепей синусоидального тока. Определены токи во всех ветвях, составлен баланс мощностей. Для достижения резонанса напряжений необходимо добавить катушку индуктивности 17,7 мГн. Результаты расчёта подтверждены моделированием в Electronics Workbench с погрешностью не более 4,3\% для исходного режима.

\end{document}