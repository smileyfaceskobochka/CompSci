\documentclass[oneside,a4paper,14pt]{extarticle}
\usepackage[a4paper,letterpaper,top=20mm,bottom=20mm,left=20mm,right=10mm]{geometry}
\usepackage[russian]{babel}
\usepackage{indentfirst}
\usepackage{graphicx}
\usepackage{caption}
\usepackage{titlesec}
\usepackage{minted, fancyvrb}
\usepackage{hyperref}
\usepackage{enumitem}

\setminted{style = rainbow_dash, fontsize = \small}

\titleformat{\section}{\normalsize\bfseries}{\thesection}{1em}{}
\titleformat{\subsection}{\normalsize\bfseries}{\thesubsection}{1em}{}

\renewcommand\baselinestretch{1.33}
\setlength{\parindent}{1.25cm}

\setlist[enumerate]{
  left=\parindent,
  label=\arabic*.,
  itemsep=0pt,
  topsep=5pt,
  partopsep=0pt,
  parsep=0pt
}

\setlist[itemize]{
  left=\parindent,
  itemsep=0pt,
  topsep=5pt,
  partopsep=0pt,
  parsep=0pt
}

\hypersetup{
  colorlinks=true,
  linkcolor=black,
  urlcolor=blue,
  pdfborder={0 0 0},
  pdftitle={Веб-приложение на Express},
  pdfauthor={Черкасов А. А.}
}

\begin{document}

\newpage
\thispagestyle{empty}
\begin{center}
  МИНИСТЕРСТВО НАУКИ И ВЫСШЕГО ОБРАЗОВАНИЯ РОССИЙСКОЙ ФЕДЕРАЦИИ\\
  ФЕДЕРАЛЬНОЕ ГОСУДАРСТВЕННОЕ БЮДЖЕТНОЕ ОБРАЗОВАТЕЛЬНОЕ УЧРЕЖДЕНИЕ ВЫСШЕГО ОБРАЗОВАНИЯ\\
  «ВЯТСКИЙ ГОСУДАРСТВЕННЫЙ УНИВЕРСИТЕТ»\\
  Институт математики и информационных систем\\
  Факультет автоматики и вычислительной техники\\
  Кафедра электронных вычислительных машин
\end{center}
\vspace{10mm}

\hfill
\begin{tabular}{l}
  \footnotesize Дата сдачи на проверку:                                          \\
  \footnotesize <<\rule[-1mm]{5mm}{0.10mm}\/>>\rule[-1mm]{20mm}{0.10mm}\ 2025 г. \\
  \footnotesize Проверено:                                                       \\
  \footnotesize <<\rule[-1mm]{5mm}{0.10mm}\/>>\rule[-1mm]{20mm}{0.10mm}\ 2025 г. \\
\end{tabular}
\vfill

\begin{center}
  Разработка клиент–серверного веб-приложения на HTML, CSS, JavaScript и Express.js.\\
  Отчёт по лабораторной работе №2\\
  по дисциплине\\
  «Технологии Программирования»
\end{center}

\vspace{25mm}

\noindent
\begin{tabular}{ll}
  Разработал студент гр. ИВТб-2301-05-00 & \hspace{18mm}\rule[-1mm]{30mm}{0.10mm}\,/Черкасов А. А./ \\
                                         & \hspace{25.5mm}\footnotesize(подпись)                     \\
  Преподаватель                  & \hspace{18mm}\rule[-1mm]{30mm}{0.10mm}\,/Пащенко Д. Э./   \\
                                         & \hspace{25.5mm}\footnotesize(подпись)                     \\
\end{tabular}

\vfill
\begin{center}
  Киров\\
  2025
\end{center}

\newpage
\section*{Цели лабораторной работы}

\begin{itemize}
  \item[$-$] изучить структуру клиент–серверных веб-приложений;
  \item[$-$] освоить работу с Express.js и обработку POST-запросов;
  \item[$-$] закрепить навыки раздельного использования HTML, CSS и JavaScript;
  \item[$-$] научиться принимать и обрабатывать данные, отправленные из формы.
\end{itemize}

\section*{Задание}

Создать веб-приложение, работающее по архитектуре «клиент–сервер», удовлетворяющее требованиям:

\begin{enumerate}
  \item На стороне клиента реализовать HTML-страницу с формой отправки данных.
  \item Стили разместить в отдельном файле CSS, логику обработки формы — в отдельном JS.
  \item На стороне сервера реализовать обработку POST-запроса с использованием Express.js.
  \item Сервер должен принимать данные, логировать их и возвращать ответ клиенту.
  \item Организовать вывод всех полученных данных в виде массива.
  \item Файлы клиента должны раздаваться через статическую директорию.
\end{enumerate}

\newpage
\section*{Ход выполнения работы}

\subsection*{1. Создание структуры проекта}

Проект был создан в соответствии с типовой структурой веб-приложения:

\begin{itemize}
  \item \texttt{index.html} — основная страница с формой;
  \item \texttt{style.css} — стили оформления;
  \item \texttt{script.js} — логика отправки данных;
  \item \texttt{server.js} — сервер на Express.js;
  \item статические файлы размещены в папке \texttt{/public}.
\end{itemize}

\subsection*{2. Реализация клиентской части}

Была разработана HTML-форма, содержащая несколько полей ввода и кнопку отправки.
Все стили вынесены в отдельный файл CSS.
JavaScript осуществляет отправку данных методом POST и выводит ответ сервера.

\subsection*{3. Реализация серверной части}

Сервер был создан на Express.js.
Были подключены модули:

\begin{itemize}
  \item \texttt{express} — основной фреймворк;
  \item \texttt{body-parser} — обработка POST-запросов.
\end{itemize}

На сервере была реализована логика:

\begin{itemize}
  \item раздача статических файлов;
  \item логирование всех входящих запросов (данные формы, query, headers);
  \item сохранение всех полученных данных в массив;
  \item ответ клиенту после успешного получения данных;
  \item вывод массива по запросу \texttt{/list}.
\end{itemize}

Также дополнительно был реализован расширенный лог всего, что происходит в сервере: тело запроса, параметры, заголовки, время получения.

\subsection*{4. Тестирование работы приложения}

После запуска сервера через:

\begin{verbatim}
bun server.js
или
node server.js
\end{verbatim}

Приложение было открыто в браузере по адресу:

\begin{center}
  \texttt{http://localhost:8080}
\end{center}

Поля формы были заполнены, данные успешно отправлены и отобразились в консоли, а затем были возвращены пользователю.

\subsection*{5. Скриншот формы}

\begin{figure}[H] \centering \includegraphics[height=0.45\textheight]{pics/form.png} \caption*{Рисунок 1 --- Форма} \end{figure}

\newpage
\section*{Вывод}

В ходе выполнения лабораторной работы было разработано полноценное клиент–серверное веб-приложение.
Была изучена структура веб-проекта, реализован сервер на Express.js, выполнено разделение клиентского кода на HTML, CSS и JavaScript.

Были выполнены все требования задания: форма корректно отправляет данные, сервер принимает их, логирует, сохраняет и возвращает ответ клиенту.

Работа позволила закрепить знания по основам веб-разработки и принципам построения архитектуры «клиент–сервер».

\newpage
\section*{Приложение А1. Исходный код index.html}
\inputminted{html}{code/public/index.html}

\section*{Приложение А2. Исходный код script.js}
\inputminted{js}{code/public/script.js}

\section*{Приложение А3. Исходный код style.css}
\inputminted{css}{code/public/style.css}

\section*{Приложение А4. Исходный код server.js}
\inputminted{js}{code/server.js}

\end{document}
