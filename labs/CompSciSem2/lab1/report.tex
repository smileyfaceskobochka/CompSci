\documentclass[oneside,a4paper,14pt]{extarticle}
\usepackage[a4paper,letterpaper,top=20mm,bottom=20mm,left=20mm,right=10mm]{geometry}
\usepackage[russian]{babel}
\usepackage{indentfirst}
\usepackage{graphicx}
\usepackage{caption}
\usepackage{titlesec}
\usepackage{minted, fancyvrb}
\usepackage{hyperref}
\usepackage{enumitem}

% Форматирование листингов с кодом
\setminted{style = rainbow_dash, fontsize = \small} % https://pygments.org/styles/

% Форматирование заголовков
\titleformat{\section}{\normalsize\bfseries}{\thesection}{1em}{}
\titleformat{\subsection}{\normalsize\bfseries}{\thesubsection}{1em}{}
\titleformat{\subsubsection}{\normalsize\bfseries}{\thesubsubsection}{1em}{}

% Интерлиньяж и абзац
\renewcommand\baselinestretch{1.33}

\setlength{\parindent}{1.25cm}  % длина красной строки

% Для всех списков
\setlist[enumerate]{
  left=0.5\parindent,       % отступ слева
  label=\arabic*.,       % цифры
  itemsep=0pt,           % расстояние между пунктами
  topsep=5pt,            % отступ сверху
  partopsep=0pt,         % дополнительный отступ сверху, если абзац до списка
  parsep=0pt             % отступ между абзацами внутри пункта
}

\setlist[itemize]{
  left=0.5\parindent,       % отступ слева
  itemsep=0pt,           % расстояние между пунктами
  topsep=5pt,            % отступ сверху
  partopsep=0pt,
  parsep=0pt
}

% Гиперссылки
\hypersetup{
  colorlinks=true,
  linkcolor=black,
  urlcolor=blue,
  pdfborder={0 0 0},
  pdftitle={PyQt6 приложение для управления устройствами умного дома},
  pdfauthor={Черкасов А.А.}
}

\begin{document}

\newpage
\thispagestyle{empty}
\begin{center}
  МИНИСТЕРСТВО НАУКИ И ВЫСШЕГО ОБРАЗОВАНИЯ РОССИЙСКОЙ ФЕДЕРАЦИИ ФЕДЕРАЛЬНОЕ ГОСУДАРСТВЕННОЕ БЮДЖЕТНОЕ ОБРАЗОВАТЕЛЬНОЕ УЧРЕЖДЕНИЕ ВЫСШЕГО ОБРАЗОВАНИЯ\\
  <<ВЯТСКИЙ ГОСУДАРСТВЕННЫЙ УНИВЕРСИТЕТ>>\\
  Институт математики и информационных систем\\
  Факультет автоматики и вычислительной техники\\
  Кафедра электронных вычислительных машин
\end{center}
\vspace{10mm}

\hfill
\begin{tabular}{l}
  \footnotesize Дата сдачи на проверку:                                          \\
  \footnotesize <<\rule[-1mm]{5mm}{0.10mm}\/>>\rule[-1mm]{20mm}{0.10mm}\ 2025 г. \\
  \footnotesize Проверено:                                                       \\
  \footnotesize <<\rule[-1mm]{5mm}{0.10mm}\/>>\rule[-1mm]{20mm}{0.10mm}\ 2025 г. \\
\end{tabular}
\vfill

\begin{center}
  Отчёт по лабораторной работе №9\\
  по дисциплине\\
  <<Информатика>>\\
  <<Светодиодные индикаторы>>
\end{center}
\vspace{25mm}
\noindent
\begin{tabular}{ll}
  Разработал студент гр. ИВТб-1301-05-00 & \hspace{18mm}\rule[-1mm]{30mm}{0.10mm}\,/Черкасов А. А./ \\
                                         & \hspace{25.5mm}\footnotesize(подпись)                    \\
  Проверил преподователь                 & \hspace{18mm}\rule[-1mm]{30mm}{0.10mm}\,/Шмакова Н.А./   \\
                                         & \hspace{25.5mm}\footnotesize(подпись)                    \\
\end{tabular}

\noindent
\begin{tabular}{lp{58mm}r}
  Работа защищена &  & \hspace{13mm}<<\rule[-1mm]{5mm}{0.10mm}\/>>\rule[-1mm]{30mm}{0.10mm}\ 2025 г.
\end{tabular}
\vfill

\begin{center}
  Киров\\
  2025
\end{center}

\newpage\thispagestyle{plain}

\section*{Цель}

Закрепить на практике знания о светодиодных индикаторах на примере работы с arduino,

\section*{Задания}
\begin{enumerate}
  \item Бегущий огонек. На основе базовой схемы сделать движение светодиодов. Начальное состояние: вкл. Движение: четные/нечетные c 10.

  \item Пульсар. На основе базовой схемы сделать движение светодиодов. Начальное состояние: вкл. Транзистор: биполярный. мена состони: увелиение до 1/3, до 2/3 и max через сброс.

  \item Ночной светильник. Начальное состояние: выкл. Входной сигнал: фоторезистор. Делитель напряжения: 220 Ом. Смена состояний: увеличение в зависимости от значений на датчике.

  \item Кнопочный переключатель.

        Л – левая кнопка;\\
        П – правая кнопка;\\
        ЛП – последовательное нажатие левой, а потом правой кнопки;\\
        ПЛ – последовательное нажатие правой, а потом левой кнопки;\\
        1/2 - количество кнопок, которые нужны для включения/выключения.

        Включение: 2. Направление: ПЛ или ЛП. Выключение: 2. Направление: ПЛ.

  \item RGB светодиод. Фиксация. Срабатывание по отпусканию. Смешивание цветов. Светодиод с общим катодом.

\end{enumerate}

\newpage

\section*{Решение}
\section*{Задание 1}

\section*{Схема сборки на макетной плате}
Схема сборки на макетной плате представлена на рисунке 1.1.

\begin{figure}[H]
  \centering
  \includegraphics[width=0.9\linewidth]{pics/1.png}
  \caption*{Рисунок 1.1 --- Схема сборки задания №1}
\end{figure}

\newpage

\section*{Принципиальная схема}
Принципиальная схема представлена на рисунке 1.2.

\begin{figure}[H]
  \centering
  \includegraphics[width=1\linewidth]{pics/1п.png}
  \caption*{Рисунок 1.2 --- Принципиальная схема задания №1}
\end{figure}

\begin{center}
  \href{https://www.tinkercad.com/things/2D3TVgkxuD7/editel?sharecode=KaiK4c4qEgjOIbT-UGdksz3bU1xRun3s13X7-mWlXuM}
  {\fcolorbox{blue}{white}{\textcolor{blue}{Перейти на Tinkercad}}}
\end{center}

Исходный код решения задания представлен в приложении A1.

\newpage
\section*{Задание 2}
\section*{Схема сборки на макетной плате}
Схема сборки на макетной плате представлена на рисунке 2.1.

\begin{figure}[H]
  \centering
  \includegraphics[width=0.9\linewidth]{pics/2.png}
  \caption*{Рисунок 2.1 --- Схема сборки задания №2}
\end{figure}

\newpage

\section*{Принципиальная схема}
Принципиальная схема представлена на рисунке 2.2.

\begin{figure}[H]
  \centering
  \includegraphics[width=1\linewidth]{pics/2п.png}
  \caption*{Рисунок 2.2 --- Принципиальная схема задания №2}
\end{figure}


\begin{center}
  \href{https://www.tinkercad.com/things/6QXfTwQgOm2/editel?sharecode=hVZaKib-6TnGnMEjUAKuUgA9t2YHPhAyYL8WWsdq378}
  {\fcolorbox{blue}{white}{\textcolor{blue}{Перейти на Tinkercad}}}
\end{center}

Исходный код решения задания представлен в приложении A2.

\newpage

\section*{Решение}
\section*{Задание 3}
\section*{Схема сборки на макетной плате}
Схема сборки на макетной плате представлена на рисунке 3.1.

\begin{figure}[H]
  \centering
  \includegraphics[width=0.9\linewidth]{pics/3.png}
  \caption*{Рисунок 3.1 --- Схема сборки задания №3}
\end{figure}

\newpage

\section*{Принципиальная схема}
Принципиальная схема представлена на рисунке 3.2.

\begin{figure}[H]
  \centering
  \includegraphics[width=1\linewidth]{pics/3п.png}
  \caption*{Рисунок 3.2 --- Принципиальная схема задания №3}
\end{figure}

\begin{center}
  \href{https://www.tinkercad.com/things/9Gg0dgO6gEa/editel?sharecode=wnCi498ImFvTVeCCCTZrEJ1gGQIfhsfvnG4GSTBcBtA}
  {\fcolorbox{blue}{white}{\textcolor{blue}{Перейти на Tinkercad}}}
\end{center}

Исходный код решения задания представлен в приложении A3.

\newpage

\section*{Решение}
\section*{Задание 4}
\section*{Схема сборки на макетной плате}
Схема сборки на макетной плате представлена на рисунке 4.1.

\begin{figure}[H]
  \centering
  \includegraphics[width=0.9\linewidth]{pics/4.png}
  \caption*{Рисунок 4.1 --- Схема сборки задания №4}
\end{figure}

\newpage

\section*{Принципиальная схема}
Принципиальная схема представлена на рисунке 4.2.

\begin{figure}[H]
  \centering
  \includegraphics[width=1\linewidth]{pics/4п.png}
  \caption*{Рисунок 4.2 --- Принципиальная схема задания №4}
\end{figure}

\begin{center}
  \href{https://www.tinkercad.com/things/hTkrnFu6FHm/editel?sharecode=_KAjgeSUXeKAns_gorue3tzy4clrXL3kjBwbN6rs4kQ}
  {\fcolorbox{blue}{white}{\textcolor{blue}{Перейти на Tinkercad}}}
\end{center}

Исходный код решения задания представлен в приложении A4.

\newpage

\section*{Решение}
\section*{Задание 5}
\section*{Схема сборки на макетной плате}
Схема сборки на макетной плате представлена на рисунке 5.1.

\begin{figure}[H]
  \centering
  \includegraphics[width=0.9\linewidth]{pics/5.png}
  \caption*{Рисунок 5.1 --- Схема сборки задания №5}
\end{figure}

\newpage

\section*{Принципиальная схема}
Принципиальная схема представлена на рисунке 5.2.

\begin{figure}[H]
  \centering
  \includegraphics[width=1\linewidth]{pics/5п.png}
  \caption*{Рисунок 5.2 --- Принципиальная схема задания №5}
\end{figure}

\begin{center}
  \href{https://www.tinkercad.com/things/bu85NNm57kx/editel?sharecode=9HkeKX7Ep1wXolqB9uYS8ioTtNVZwScfSsIWMFTA-hA}
  {\fcolorbox{blue}{white}{\textcolor{blue}{Перейти на Tinkercad}}}
\end{center}

Исходный код решения задания представлен в приложении A5.

\newpage

\section*{Вывод}
В результате выполнения лабораторной работы была достигнута поставленная цель --- закрепление на практике знаний о светодиодных индикаторах на примере работы с платформой Arduino. Были успешно реализованы пять различных заданий, демонстрирующих разнообразные способы управления светодиодами:

\begin{enumerate}
  \item Бегущий огонек --- последовательное включение светодиодов с чередованием четных и нечетных позиций;
  \item Пульсар --- плавное изменение яркости светодиода с использованием ШИМ-модуляции;
  \item Ночной светильник --- автоматическое управление яркостью в зависимости от освещенности с помощью фоторезистора;
  \item Кнопочный переключатель --- управление включением/выключением с использованием последовательности нажатий кнопок;
  \item RGB-светодиод --- управление цветом путем смешивания трех основных цветов.
\end{enumerate}

В ходе работы были изучены основные принципы работы со светодиодными индикаторами, освоены методы цифрового и аналогового управления выводами Arduino, а также получены практические навыки программирования микроконтроллера. Были использованы следующие ключевые функции Arduino:

pinMode(pin, mode) – настройка режима работы пина (вход/выход)

digitalWrite(pin, value) – установка цифрового уровня сигнала (HIGH/LOW)

analogWrite(pin, value) – генерация ШИМ-сигнала для управления яркостью

digitalRead(pin) – чтение состояния цифрового входа

analogRead(pin) – чтение аналогового значения с датчика

delay(ms) – задержка выполнения программы

Serial.begin(speed) – инициализация последовательного порта для отладки

\newpage

\section*{Приложение А1. Исходный код решения для задания №1}
\inputminted{cpp}{code/lab1_1.cpp}

\section*{Приложение А2. Исходный код решения для задания №2}
\inputminted{cpp}{code/lab1_2.cpp}

\section*{Приложение А3. Исходный код решения для задания №3}
\inputminted{cpp}{code/lab1_3.cpp}

\section*{Приложение А4. Исходный код решения для задания №4}
\inputminted{cpp}{code/lab1_4.cpp}

\section*{Приложение А5. Исходный код решения для задания №5}
\inputminted{cpp}{code/lab1_5.cpp}

\end{document}
