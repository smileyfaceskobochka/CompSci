\documentclass[oneside,a4paper,14pt]{extarticle}
\usepackage[a4paper,letterpaper,top=20mm,bottom=20mm,left=20mm,right=10mm]{geometry}
\usepackage[russian]{babel}
\usepackage{indentfirst}
\usepackage{graphicx}
\usepackage{caption}
\usepackage{titlesec}
\usepackage{minted, fancyvrb}
\usepackage{hyperref}
\usepackage{enumitem}

% Форматирование листингов
\setminted{style = rainbow_dash, fontsize = \small}

% Форматирование заголовков
\titleformat{\section}{\normalsize\bfseries}{\thesection}{1em}{}
\titleformat{\subsection}{\normalsize\bfseries}{\thesubsection}{1em}{}
\titleformat{\subsubsection}{\normalsize\bfseries}{\thesubsubsection}{1em}{}

\renewcommand\baselinestretch{1.33}
\setlength{\parindent}{1.25cm}

\setlist[enumerate]{
  left=\parindent,
  label=\arabic*.,
  itemsep=0pt,
  topsep=5pt,
  partopsep=0pt,
  parsep=0pt
}

\setlist[itemize]{
  left=\parindent,
  itemsep=0pt,
  topsep=5pt,
  partopsep=0pt,
  parsep=0pt
}

\hypersetup{
  colorlinks=true,
  linkcolor=black,
  urlcolor=blue,
  pdfborder={0 0 0},
  pdftitle={Кроссплатформенное приложение Flutter},
  pdfauthor={Черкасов А.А.}
}

\begin{document}

\newpage
\thispagestyle{empty}
\begin{center}
	МИНИСТЕРСТВО НАУКИ И ВЫСШЕГО ОБРАЗОВАНИЯ РОССИЙСКОЙ ФЕДЕРАЦИИ\\
	ФЕДЕРАЛЬНОЕ ГОСУДАРСТВЕННОЕ БЮДЖЕТНОЕ ОБРАЗОВАТЕЛЬНОЕ УЧРЕЖДЕНИЕ ВЫСШЕГО ОБРАЗОВАНИЯ\\[1mm]
	«ВЯТСКИЙ ГОСУДАРСТВЕННЫЙ УНИВЕРСИТЕТ»\\
	Институт математики и информационных систем\\
	Факультет автоматики и вычислительной техники\\
	Кафедра электронных вычислительных машин
\end{center}

\vspace{10mm}

\hfill
\begin{tabular}{l}
	\footnotesize Дата сдачи на проверку:                                          \\
	\footnotesize <<\rule[-1mm]{5mm}{0.10mm}\/>>\rule[-1mm]{20mm}{0.10mm}\ 2025 г. \\
	\footnotesize Проверено:                                                       \\
	\footnotesize <<\rule[-1mm]{5mm}{0.10mm}\/>>\rule[-1mm]{20mm}{0.10mm}\ 2025 г. \\
\end{tabular}

\vfill

\begin{center}
	Разработка кроссплатформенного приложения на Flutter.\\
	Отчёт по лабораторной работе №5\\
	по дисциплине\\
	«Технологии программирования»\\
	Вариант 5
\end{center}

\vspace{25mm}

\noindent
\begin{tabular}{ll}
	Разработал студент гр. ИВТб-2301-05-00 & \hspace{18mm}\rule[-1mm]{30mm}{0.10mm}\,/Черкасов А. А./ \\
	                                       & \hspace{25.5mm}\footnotesize(подпись)                    \\
	Преподаватель                          & \hspace{18mm}\rule[-1mm]{30mm}{0.10mm}\,/Пащенко Д. Э./  \\
	                                       & \hspace{25.5mm}\footnotesize(подпись)                    \\
\end{tabular}

\vfill

\begin{center}
	Киров\\
	2025
\end{center}

\newpage
\thispagestyle{plain}

\section*{Цели лабораторной работы}

\begin{itemize}
	\item[$-$] освоить практические навыки создания кроссплатформенных приложений на Flutter;
	\item[$-$] познакомиться с основными виджетами для построения пользовательского интерфейса;
	\item[$-$] понять структуру проекта Flutter и особенности работы с виджетами компоновки.
\end{itemize}

\section*{Задание}

Разработать приложение на Flutter с использованием различных виджетов компоновки:

\begin{enumerate}
	\item Добавить в Leading AppBar произвольную иконку с помощью class Icons.
	\item В тело экрана добавить изображения из ассетов и сети, задать отступ.
	\item Использовать виджеты Wrap и Column для компоновки нескольких контейнеров и текстовых виджетов.
	\item Ознакомиться с параметрами и использованием виджетов Stack, Row, Column, Wrap для компоновки интерфейса.
\end{enumerate}

\clearpage
\section*{Реализация приложения}

В рамках лабораторной работы разработано Flutter-приложение, демонстрирующее использование основных виджетов компоновки и структурирования интерфейса.

\subsection*{Структура приложения}

Приложение состоит из единственного экрана, построенного с использованием следующих компонентов:

\begin{itemize}
	\item[$-$] \textbf{MaterialApp} — корневой виджет приложения, определяющий тему и базовые настройки.
	\item[$-$] \textbf{Scaffold} — каркас экрана, содержащий AppBar и основное тело.
	\item[$-$] \textbf{AppBar} — верхняя панель с иконкой и заголовком.
	\item[$-$] \textbf{Stack} — основной контейнер для компоновки элементов с наложением.
	\item[$-$] \textbf{Positioned} — виджет точного позиционирования элементов внутри Stack.
\end{itemize}

\subsection*{Настройка AppBar}

В верхнюю панель приложения добавлена иконка \texttt{Icons.code} и заголовок с ФИО студента:

\begin{minted}{dart}
appBar: AppBar(
  leading: Icon(Icons.code),
  title: Text('Черкасов Александр Андреевич'),
  backgroundColor: Colors.blue,
),
\end{minted}

\subsection*{Компоновка элементов}

Для размещения элементов использован виджет \texttt{Stack}, позволяющий накладывать виджеты друг на друга. Внутри Stack размещены:

\begin{enumerate}
	\item Четыре цветных контейнера по углам экрана (красный, зеленый, синий, оранжевый).
	\item Центральный текстовый блок.
\end{enumerate}

Пример позиционирования контейнера:

\begin{minted}{dart}
Positioned(
  top: 0,
  left: 0,
  child: Container(
    width: 100,
    height: 100,
    color: Colors.red,
  ),
),
\end{minted}

\subsection*{Центральный текстовый блок}

По центру экрана размещен текстовый виджет с надписью «Технологии программирования»:

\begin{minted}{dart}
Center(
  child: Container(
    padding: EdgeInsets.all(16),
    color: Colors.white.withOpacity(0.8),
    child: Text(
      'Технологии программирования',
      style: TextStyle(
        fontSize: 24,
        fontWeight: FontWeight.bold,
        color: Colors.black,
      ),
    ),
  ),
),
\end{minted}

\subsection*{Скриншот работы приложения}

\begin{figure}[H]
	\centering
	\includegraphics[height=0.7\textheight]{pics/screen.png}
	\caption*{Рисунок 1 --- Основной экран приложения}
\end{figure}

\section*{Контрольные вопросы}

\subsection*{1. Что такое Flutter?}

Flutter — это фреймворк от Google для разработки кроссплатформенных приложений под Android, iOS, Web и Desktop с использованием языка программирования Dart. Отличается высокой производительностью и собственной системой рендеринга.

\subsection*{2. Структура Flutter-проекта}

Типичная структура включает:
\begin{itemize}
	\item[$-$] \texttt{lib/} — основной код приложения на Dart;
	\item[$-$] \texttt{pubspec.yaml} — файл настроек проекта и зависимостей;
	\item[$-$] \texttt{android/}, \texttt{ios/}, \texttt{web/} — платформенная обёртка;
	\item[$-$] \texttt{assets/} — ресурсы (изображения, шрифты).
\end{itemize}

\subsection*{3. Основные виджеты и функции}

\begin{itemize}
	\item[$-$] \textbf{runApp()} — точка входа, запускает приложение;
	\item[$-$] \textbf{MaterialApp} — корневой компонент Material Design, задаёт тему и маршруты;
	\item[$-$] \textbf{Scaffold} — основной шаблон экрана (AppBar, тело, меню);
	\item[$-$] \textbf{AppBar} — верхняя панель приложения;
	\item[$-$] \textbf{Container} — блок с возможностью задать размеры, отступы, цвет;
	\item[$-$] \textbf{Text} — текстовый элемент;
	\item[$-$] \textbf{TextStyle} — оформление текста (цвет, размер, шрифт);
	\item[$-$] \textbf{Image} — отображение изображения (локального или сетевого);
	\item[$-$] \textbf{Icon} — иконка из набора Material Icons;
	\item[$-$] \textbf{Padding} — отступ вокруг дочернего виджета.
\end{itemize}

\subsection*{4. Виджет Column}

Column располагает элементы вертикально. Основные параметры:
\begin{itemize}
	\item[$-$] \texttt{children} — список виджетов;
	\item[$-$] \texttt{mainAxisAlignment} — выравнивание по вертикали;
	\item[$-$] \texttt{crossAxisAlignment} — выравнивание по горизонтали.
\end{itemize}

\subsection*{5. Виджет Row}

Row располагает элементы горизонтально. Параметры аналогичны Column:
\begin{itemize}
	\item[$-$] \texttt{children} — список дочерних элементов;
	\item[$-$] \texttt{mainAxisAlignment} — выравнивание по горизонтали;
	\item[$-$] \texttt{crossAxisAlignment} — выравнивание по вертикали.
\end{itemize}

\subsection*{6. Виджет Stack}

Stack накладывает элементы друг на друга. Основные параметры:
\begin{itemize}
	\item[$-$] \texttt{children} — список слоёв;
	\item[$-$] \texttt{alignment} — выравнивание содержимого;
	\item[$-$] \texttt{fit} — управление размером дочерних элементов.
\end{itemize}

\subsection*{7. Виджет Center}

Center располагает один дочерний элемент строго по центру экрана или доступной области.

\subsection*{8. Виджет Wrap}

Wrap размещает элементы в нескольких строках или столбцах, автоматически перенося их. Параметры:
\begin{itemize}
	\item[$-$] \texttt{spacing} — расстояние между элементами по горизонтали;
	\item[$-$] \texttt{runSpacing} — расстояние между строками;
	\item[$-$] \texttt{direction} — горизонтальное или вертикальное размещение.
\end{itemize}

\section*{Вывод}

В ходе выполнения лабораторной работы №5 были освоены практические навыки создания кроссплатформенных приложений на Flutter.  
Реализовано базовое приложение с использованием основных виджетов компоновки: AppBar, Stack, Positioned, Center.  
Изучены особенности работы с различными виджетами для построения пользовательского интерфейса.  
Полученный функционал полностью соответствует требованиям методических указаний.

\newpage

\section*{Приложение А. Исходный код main.dart}
\inputminted{dart}{code/lib/main.dart}

\end{document}