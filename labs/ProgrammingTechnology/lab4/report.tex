\documentclass[oneside,a4paper,14pt]{extarticle}
\usepackage[a4paper,letterpaper,top=20mm,bottom=20mm,left=20mm,right=10mm]{geometry}
\usepackage[russian]{babel}
\usepackage{indentfirst}
\usepackage{graphicx}
\usepackage{caption}
\usepackage{titlesec}
\usepackage{minted}
\usepackage{hyperref}
\usepackage{enumitem}

% Форматирование листингов
\setminted{style = rainbow_dash, fontsize = \small}

% Форматирование заголовков
\titleformat{\section}{\normalsize\bfseries}{\thesection}{1em}{}
\titleformat{\subsection}{\normalsize\bfseries}{\thesubsection}{1em}{}
\titleformat{\subsubsection}{\normalsize\bfseries}{\thesubsubsection}{1em}{}

\renewcommand\baselinestretch{1.33}
\setlength{\parindent}{1.25cm}

\setlist[enumerate]{
  left=\parindent,
  label=\arabic*.,
  itemsep=0pt,
  topsep=5pt,
  partopsep=0pt,
  parsep=0pt
}

\setlist[itemize]{
  left=\parindent,
  itemsep=0pt,
  topsep=5pt,
  partopsep=0pt,
  parsep=0pt
}

\hypersetup{
  colorlinks=true,
  linkcolor=black,
  urlcolor=blue,
  pdfborder={0 0 0},
  pdftitle={Библиотеки неявной и явной загрузки},
  pdfauthor={Черкасов А.А.}
}

\begin{document}

\newpage
\thispagestyle{empty}
\begin{center}
  МИНИСТЕРСТВО НАУКИ И ВЫСШЕГО ОБРАЗОВАНИЯ РОССИЙСКОЙ ФЕДЕРАЦИИ\\
  ФЕДЕРАЛЬНОЕ ГОСУДАРСТВЕННОЕ БЮДЖЕТНОЕ ОБРАЗОВАТЕЛЬНОЕ УЧРЕЖДЕНИЕ ВЫСШЕГО ОБРАЗОВАНИЯ\\[1mm]
  «ВЯТСКИЙ ГОСУДАРСТВЕННЫЙ УНИВЕРСИТЕТ»\\
  Институт математики и информационных систем\\
  Факультет автоматики и вычислительной техники\\
  Кафедра электронных вычислительных машин
\end{center}

\vspace{10mm}

\hfill
\begin{tabular}{l}
  \footnotesize Дата сдачи на проверку:                                          \\
  \footnotesize <<\rule[-1mm]{5mm}{0.10mm}\/>>\rule[-1mm]{20mm}{0.10mm}\ 2025 г. \\
  \footnotesize Проверено:                                                       \\
  \footnotesize <<\rule[-1mm]{5mm}{0.10mm}\/>>\rule[-1mm]{20mm}{0.10mm}\ 2025 г. \\
\end{tabular}

\vfill

\begin{center}
  Разработка библиотек неявной и явной загрузки с использованием ассемблера.\\[1mm]
  Отчёт по лабораторной работе №4\\
  по дисциплине\\
  «Технологии программирования»
\end{center}

\vspace{25mm}

\noindent
\begin{tabular}{ll}
  Разработал студент гр. ИВТб-2301-05-00 & \hspace{18mm}\rule[-1mm]{30mm}{0.10mm}\,/Черкасов А. А./ \\
                                         & \hspace{25.5mm}\footnotesize(подпись)                    \\
  Преподаватель                          & \hspace{18mm}\rule[-1mm]{30mm}{0.10mm}\,/Пащенко Д. Э./  \\
                                         & \hspace{25.5mm}\footnotesize(подпись)                    \\
\end{tabular}

\vfill

\begin{center}
  Киров\\
  2025
\end{center}

\newpage
\thispagestyle{plain}

\section{Цель работы}

Целью данной лабораторной работы является расширение программы из лабораторной работы №3 путем:
\begin{itemize}
  \item Переработки окон (диалогов) под формат библиотеки неявной загрузки
  \item Реализации вычислительных функций (хэширования паролей) в виде библиотеки явной загрузки
  \item Использования ассемблера для одной из вычислительных функций
  \item Тестирования функционала и исправления ошибок
\end{itemize}

\section{Теоретическая часть}

\subsection{Неявная загрузка библиотек}

Неявная загрузка библиотек (implicit linking) происходит во время запуска программы. Библиотека подключается на этапе компиляции через линкер, и операционная система автоматически загружает ее при запуске исполняемого файла.

Преимущества:
\begin{itemize}
  \item Простота использования
  \item Автоматическая загрузка
  \item Не требуется дополнительный код для загрузки
\end{itemize}

Недостатки:
\begin{itemize}
  \item Библиотека должна быть доступна во время запуска
  \item Сложнее обновлять библиотеку без перекомпиляции
\end{itemize}

\subsection{Явная загрузка библиотек}

Явная загрузка библиотек (explicit linking) происходит во время выполнения программы с помощью системных вызовов (dlopen/dlsym/dlclose в Linux).

Преимущества:
\begin{itemize}
  \item Гибкость - можно загружать разные версии библиотек
  \item Возможность обработки ошибок загрузки
  \item Можно загружать библиотеку по требованию
\end{itemize}

Недостатки:
\begin{itemize}
  \item Более сложный код
  \item Необходимость управления временем жизни библиотеки
\end{itemize}

\subsection{Использование ассемблера}

В данной работе использован ассемблер x86-64 для реализации простой хэш-функции. Ассемблер позволяет:
\begin{itemize}
  \item Максимальную производительность
  \item Низкоуровневый контроль над аппаратными ресурсами
  \item Реализацию специфичных алгоритмов
\end{itemize}

\section{Практическая часть}

\subsection{Структура проекта}

Проект состоит из следующих компонентов:

\begin{itemize}
  \item \texttt{libcrypto.so} - библиотека явной загрузки с функциями хэширования
  \item \texttt{libdialogs.so} - библиотека неявной загрузки с диалогами
  \item \texttt{lab4} - основное приложение
\end{itemize}

\subsection{Реализация библиотеки явной загрузки (libcrypto)}

Библиотека содержит функции для хэширования паролей с использованием PBKDF2 и дополнительной обработки ассемблерной функцией.

Ключевые файлы:
\begin{itemize}
  \item \texttt{crypto.h} - заголовочный файл с экспортируемыми функциями
  \item \texttt{crypto.cpp} - реализация функций хэширования
  \item \texttt{hash.S} - ассемблерная реализация простой хэш-функции
\end{itemize}

Ассемблерная функция \texttt{simple\_hash} реализует алгоритм хэширования djb2 в синтаксисе Intel:

\begin{minted}{nasm}
.intel_syntax noprefix
.global simple_hash

.section .text

simple_hash:
    mov eax, 5381          # hash = 5381
    test rsi, rsi          # if len == 0, return hash
    jz .done

.loop:
    xor r8d, r8d           # clear r8d
    mov r8b, [rdi]         # load byte into r8b
    imul eax, 33           # hash = hash * 33
    add eax, r8d           # hash = hash + byte

    inc rdi                # data++
    dec rsi                # len--
    jnz .loop

.done:
    ret
\end{minted}

\subsection{Реализация библиотеки неявной загрузки (libdialogs)}

Библиотека содержит все диалоговые окна приложения:
\begin{itemize}
  \item LoginDialog - диалог входа
  \item RegisterDialog - диалог регистрации
  \item DeviceDialog - диалог управления устройствами
  \item HubDialog - диалог управления хабами
\end{itemize}

Класс Database модифицирован для явной загрузки библиотеки libcrypto:

\begin{minted}{cpp}
Database::Database(const std::string &conninfo) {
    // ... инициализация БД ...
    
    // Load crypto library explicitly
    crypto_handle_ = dlopen("./libcrypto.so", RTLD_LAZY);
    if (!crypto_handle_) {
        throw std::runtime_error("Failed to load libcrypto.so");
    }
    hash_func_ = (std::string (*)(const std::string &))dlsym(crypto_handle_, "hash_password_base64");
    verify_func_ = (bool (*)(const std::string &, const std::string &))dlsym(crypto_handle_, "verify_password_base64");
    // ...
}
\end{minted}

\subsection{Сборка проекта}

Проект собирается с помощью CMake:

\begin{minted}{cmake}
cmake_minimum_required(VERSION 3.16)
project(lab4 LANGUAGES CXX ASM)

# libcrypto (explicit loading)
add_library(crypto SHARED
    libcrypto/crypto.cpp
    libcrypto/hash.S
)

# libdialogs (implicit loading)
add_library(dialogs SHARED
    libdialogs/dialogs.cpp
    libdialogs/database.cpp
)

# main executable
add_executable(lab4
    main/main.cpp
    main/mainwindow.cpp
)
target_link_libraries(lab4 PRIVATE Qt6::Widgets dialogs)
\end{minted}

\section{Результаты}

\subsection{Компиляция и сборка}

Проект успешно компилируется и собирается. Создаются следующие файлы:
\begin{itemize}
  \item \texttt{libcrypto.so} (61 KB) - библиотека криптографических функций
  \item \texttt{libdialogs.so} (588 KB) - библиотека диалогов
  \item \texttt{lab4} (348 KB) - исполняемый файл
\end{itemize}

\subsection{Функциональность}

\begin{itemize}
  \item Диалоговые окна успешно вынесены в отдельную библиотеку неявной загрузки
  \item Функции хэширования реализованы в библиотеке явной загрузки
  \item Ассемблерная функция интегрирована в процесс хэширования
  \item Реализована функциональность смены пароля для пользователей
  \item Приложение сохраняет полную функциональность лабораторной работы №3
\end{itemize}

\subsection{Тестирование}

Тестирование проводилось путем:
\begin{itemize}
  \item Компиляции проекта без ошибок
  \item Проверки загрузки библиотек
  \item Верификации работы ассемблерной функции
\end{itemize}

\section{Выводы}

В результате выполнения лабораторной работы были успешно реализованы:
\begin{itemize}
  \item Библиотека неявной загрузки для диалоговых окон
  \item Библиотека явной загрузки для вычислительных функций
  \item Интеграция ассемблерного кода в процесс хэширования
  \item Полная совместимость с существующей функциональностью
\end{itemize}

Работа демонстрирует практическое применение различных методов загрузки библиотек и интеграции ассемблерного кода в C++ приложения.

\newpage

\section*{Приложение А0. Исходный код crypto.h}
\inputminted{cpp}{code/libcrypto/crypto.h}

\newpage

\section*{Приложение А1. Исходный код crypto.cpp}
\inputminted{cpp}{code/libcrypto/crypto.cpp}

\newpage

\section*{Приложение А2. Исходный код hash.S}
\inputminted{asm}{code/libcrypto/hash.S}

\newpage

\section*{Приложение А3. Исходный код dialogs.h}
\inputminted{cpp}{code/libdialogs/dialogs.h}

\newpage

\section*{Приложение А4. Исходный код dialogs.cpp}
\inputminted{cpp}{code/libdialogs/dialogs.cpp}

\newpage

\section*{Приложение А5. Исходный код database.h}
\inputminted{cpp}{code/libdialogs/database.h}

\newpage

\section*{Приложение А6. Исходный код database.cpp}
\inputminted{cpp}{code/libdialogs/database.cpp}

\newpage

\section*{Приложение А7. Исходный код mainwindow.h}
\inputminted{cpp}{code/main/mainwindow.h}

\newpage

\section*{Приложение А8. Исходный код mainwindow.cpp}
\inputminted{cpp}{code/main/mainwindow.cpp}

\newpage

\section*{Приложение А9. Исходный код main.cpp}
\inputminted{cpp}{code/main/main.cpp}

\newpage

\section*{Приложение А10. Исходный код CMakeLists.txt}
\inputminted{cmake}{code/CMakeLists.txt}

\end{document}
