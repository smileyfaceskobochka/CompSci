\documentclass[oneside,a4paper,14pt]{extarticle}
\usepackage[a4paper,letterpaper,top=20mm,bottom=20mm,left=20mm,right=10mm]{geometry}
\usepackage[russian]{babel}
\usepackage[utf8]{inputenc}
\usepackage{indentfirst}
\usepackage{graphicx}
\usepackage{caption}
\usepackage{titlesec}
\usepackage{minted, fancyvrb}
\usepackage{hyperref}
\usepackage{enumitem}
\usepackage{amsmath,amssymb}
\usepackage{array}
\usepackage{booktabs}
\usepackage{siunitx}
\usepackage{float}

% Форматирование листингов с кодом
\setminted{style = rainbow_dash, fontsize = \small} % https://pygments.org/styles/

% Форматирование заголовков
\titleformat{\section}{\normalsize\bfseries}{\thesection}{1em}{}
\titleformat{\subsection}{\normalsize\bfseries}{\thesubsection}{1em}{}
\titleformat{\subsubsection}{\normalsize\bfseries}{\thesubsubsection}{1em}{}

% Интерлиньяж и абзац
\renewcommand\baselinestretch{1.33}

\setlength{\parindent}{1.25cm}  % длина красной строки

% Для всех списков
\setlist[enumerate]{
  left=0.5\parindent,
  label=\arabic*.,
  itemsep=0pt,
  topsep=5pt,
  partopsep=0pt,
  parsep=0pt
}

\setlist[itemize]{
  left=0.5\parindent,
  itemsep=0pt,
  topsep=5pt,
  partopsep=0pt,
  parsep=0pt
}

% Гиперссылки
\hypersetup{
  colorlinks=true,
  linkcolor=black,
  urlcolor=blue,
  pdfborder={0 0 0},
  pdftitle={Отчёт по ЛР №1},
  pdfauthor={Автор}
}

\begin{document}

\newpage
\thispagestyle{empty}
\begin{center}
  МИНИСТЕРСТВО НАУКИ И ВЫСШЕГО ОБРАЗОВАНИЯ РОССИЙСКОЙ ФЕДЕРАЦИИ ФЕДЕРАЛЬНОЕ ГОСУДАРСТВЕННОЕ БЮДЖЕТНОЕ ОБРАЗОВАТЕЛЬНОЕ УЧРЕЖДЕНИЕ ВЫСШЕГО ОБРАЗОВАНИЯ\\
  <<ВЯТСКИЙ ГОСУДАРСТВЕННЫЙ УНИВЕРСИТЕТ>>\\
  Институт математики и информационных систем\\
  Факультет автоматики и вычислительной техники\\
  Кафедра электронных вычислительных машин
\end{center}
\vspace{10mm}

\hfill
\begin{tabular}{l}
  \footnotesize Дата сдачи на проверку:                                          \\
  \footnotesize <<\rule[-1mm]{5mm}{0.10mm}\/>>\rule[-1mm]{20mm}{0.10mm}\ 2025 г. \\
  \footnotesize Проверено:                                                       \\
  \footnotesize <<\rule[-1mm]{5mm}{0.10mm}\/>>\rule[-1mm]{20mm}{0.10mm}\ 2025 г. \\
\end{tabular}
\vfill

\begin{center}
  Отчёт по лабораторной работе №1\\
  по дисциплине\\
  <<Физические основы функционирования ЭВМ>>\\
\end{center}
\vspace{25mm}
\begin{tabular}{ll}
  Выполнили студенты гр. ИВТб-2301-05-00 & \hspace{18mm}\rule[-1mm]{30mm}{0.10mm}\,/Черкасов А. А./ \\
                                         & \hspace{25.5mm}\footnotesize(подпись)                    \\
                                         & \hspace{18mm}\rule[-1mm]{30mm}{0.10mm}\,/Зинин В. А./    \\
                                         & \hspace{25.5mm}\footnotesize(подпись)                    \\
  Преподователь                          & \hspace{18mm}\rule[-1mm]{30mm}{0.10mm}\,/Будин А. Г./    \\
                                         & \hspace{25.5mm}\footnotesize(подпись)                    \\
\end{tabular}

\noindent
\begin{tabular}{lp{58mm}r}
  Работа защищена &  & \hspace{13mm}<<\rule[-1mm]{5mm}{0.10mm}\/>>\rule[-1mm]{30mm}{0.10mm}\ 2025 г.
\end{tabular}
\vfill

\begin{center}
  Киров\\
  2025
\end{center}

\newpage

\section*{Цели лабораторной работы}

\begin{itemize}
  \item[$-$] изучить методы расчёта простых электрических цепей постоянного тока;
  \item[$-$] научиться определять эквивалентное сопротивление схемы и распределение токов по её ветвям;
  \item[$-$] освоить практику проверки правильности расчётов через составление баланса мощности;
  \item[$-$] получить навыки схемотехнического моделирования в среде Electronics\\
        Workbench и сравнить результаты моделирования с аналитическим расчётом.
\end{itemize}

\section*{Задание 1}

\noindent Для заданной схемы электрической цепи необходимо:

\begin{enumerate}
  \item Рассчитать значения эквивалентного сопротивления схемы, в соответствии со 2 вариантом (Рисунок 1). Определить значения тока в каждой ветви для электрической цепи. Числовые значения сопротивлений приемников электрической схемы приведены в Таблице 1.
  \item Выполнить проверку правильности расчетов, составив баланс мощности и сделать вывод о правильности расчетов.
  \item Провести моделирование на ЭВМ режима работы рассматриваемой электрической цепи с помощью системы схемотехнического моделирования Electronics Workbench. Полученные результаты приложить к тексту оформленной работы.
  \item Сравнить результаты расчета и моделирования.
\end{enumerate}

\newpage

\begin{figure}[H]
  \centering
  \includegraphics[width=0.75\textwidth]{pics/scheme.png}
  \caption*{Рисунок 1 --- Схема электрической цепи для задания 1}
\end{figure}

\section*{Дано}
\addcontentsline{toc}{section}{Дано}

\begin{itemize}
  \item[$-$] Источник напряжения: $E = 50$ В.
  \item[$-$] Сопротивления резисторов:
        \begin{itemize}
          \item $R_1 = 14$ Ом
          \item $R_2 = 7$ Ом
          \item $R_3 = 12$ Ом
          \item $R_4 = 8$ Ом
          \item $R_5 = 17$ Ом
          \item $R_6 = 12$ Ом
        \end{itemize}
  \item[$-$] Узлы схемы (по рисунку): верхняя шина $A, B, C$ --- потенциал $V = 50$ В; нижняя левая точка $F$ --- опорный узел с потенциалом $V_F = 0$ В.
  \item[$-$] Требуется найти: потенциалы узлов $V_D$ и $V_E$, токи во всех ветвях.
\end{itemize}

\newpage

\section*{Ход работы}
\addcontentsline{toc}{section}{Ход работы}

\subsection*{1. Расчёт общего (эквивалентного) сопротивления схемы}

Для определения эквивалентного сопротивления схемы необходимо последовательно упростить её, начиная с самых удалённых от источника элементов.

\textbf{Шаг 1. Параллельное соединение $R_5$ и $R_6$:}

Резисторы $R_5 = 17$ Ом и $R_6 = 12$ Ом соединены параллельно. Их эквивалентное сопротивление:

\begin{equation}
  R_{5-6} = \frac{R_5 \cdot R_6}{R_5 + R_6} = \frac{17 \cdot 12}{17 + 12} = \frac{204}{29} \approx 7,034 \text{ Ом}
\end{equation}

\textbf{Шаг 2. Параллельное соединение $R_3$ и $R_4$:}

Резисторы $R_3 = 12$ Ом и $R_4 = 8$ Ом соединены параллельно. Их эквивалентное сопротивление:

\begin{equation}
  R_{3-4} = \frac{R_3 \cdot R_4}{R_3 + R_4} = \frac{12 \cdot 8}{12 + 8} = \frac{96}{20} = 4,8 \text{ Ом}
\end{equation}

\textbf{Шаг 3. Последовательное соединение $R_{3-4}$ и $R_{5-6}$:}

Эквивалентные сопротивления $R_{3-4}$ и $R_{5-6}$ соединены последовательно:

\begin{equation}
  R_{3-6} = R_{3-4} + R_{5-6} = 4,8 + 7,034 = 11,834 \text{ Ом}
\end{equation}

\textbf{Шаг 4. Последовательное соединение $R_2$ и $R_{3-6}$:}

Резистор $R_2 = 7$ Ом соединён последовательно с $R_{3-6}$:

\begin{equation}
  R_{2-6} = R_2 + R_{3-6} = 7 + 11,834 = 18,834 \text{ Ом}
\end{equation}

\textbf{Шаг 5. Параллельное соединение $R_1$ и $R_{2-6}$:}

Резистор $R_1 = 14$ Ом соединён параллельно с эквивалентным сопротивлением $R_{2-6}$. Общее сопротивление схемы:

\begin{equation}
  R_{\text{общ}} = \frac{R_1 \cdot R_{2-6}}{R_1 + R_{2-6}} = \frac{14 \cdot 18,834}{14 + 18,834} = \frac{263,676}{32,834} \approx 8,030 \text{ Ом}
\end{equation}

Таким образом, эквивалентное сопротивление всей схемы относительно источника напряжения составляет $R_{\text{общ}} \approx 8,03$ Ом.

\section*{Сравнение с результатами моделирования}

Результаты моделирования в Electronics Workbench на рисунке 2 показали значение общего сопротивления $R_{\text{общ}} \approx 8,547$ Ом. Расхождение с расчётным значением составляет около 6\%.

\begin{figure}[H]
  \centering
  \includegraphics[width=0.75\textwidth]{pics/resistance.png}
  \caption*{Рисунок 2 --- Модель схемы в среде Electronics Workbench}
\end{figure}

\subsection*{2. Обозначения токов в ветвях}

Зададим направления токов и выразим их через узловые потенциалы:

\begin{equation}
  I_1 = \frac{50 - 0}{R_1} = \frac{50}{14} = \frac{25}{7} \text{ А}
\end{equation}

\begin{equation}
  I_2 = \frac{50 - V_E}{R_2} = \frac{50 - V_E}{7}
\end{equation}

\begin{equation}
  I_3 = \frac{50 - V_D}{R_3} = \frac{50 - V_D}{12}
\end{equation}

\begin{equation}
  I_4 = \frac{50 - V_D}{R_4} = \frac{50 - V_D}{8}
\end{equation}

\begin{equation}
  I_6 = \frac{V_D - V_E}{R_6} = \frac{V_D - V_E}{12}
\end{equation}

\begin{equation}
  I_5 = \frac{V_E - 0}{R_5} = \frac{V_E}{17}
\end{equation}

\subsection*{3. Составление уравнений по первому закону Кирхгофа}

Запишем баланс токов для узлов $D$ и $E$.

\textbf{Узел $D$:}

В узел входят токи $I_3$ и $I_4$, из узла выходит ток $I_6$:

\begin{equation}
  \frac{50 - V_D}{12} + \frac{50 - V_D}{8} = \frac{V_D - V_E}{12}
\end{equation}

Умножим обе части уравнения на 24:

\begin{equation}
  2(50 - V_D) + 3(50 - V_D) = 2(V_D - V_E)
\end{equation}

\begin{equation}
  250 - 5V_D = 2V_D - 2V_E
\end{equation}

\begin{equation}
  7V_D - 2V_E = 250 \tag{1}
\end{equation}

\textbf{Узел $E$:}

В узел входят токи $I_2$ и $I_6$, из узла выходит ток $I_5$:

\begin{equation}
  \frac{50 - V_E}{7} + \frac{V_D - V_E}{12} = \frac{V_E}{17}
\end{equation}

Умножим обе части на $7 \cdot 12 \cdot 17 = 1428$:

\begin{equation}
  204(50 - V_E) + 119(V_D - V_E) = 84V_E
\end{equation}

\begin{equation}
  10200 - 204V_E + 119V_D - 119V_E = 84V_E
\end{equation}

\begin{equation}
  119V_D - 407V_E = -10200 \tag{2}
\end{equation}

\subsection*{4. Решение системы уравнений}

Из уравнения (1) выразим $V_D$:

\begin{equation}
  V_D = \frac{250 + 2V_E}{7}
\end{equation}

Подставим в уравнение (2). Учитывая, что $119/7 = 17$:

\begin{equation}
  17(250 + 2V_E) - 407V_E = -10200
\end{equation}

\begin{equation}
  4250 + 34V_E - 407V_E = -10200
\end{equation}

\begin{equation}
  -373V_E = -14450
\end{equation}

\begin{equation}
  V_E = \frac{14450}{373} \approx 38,74 \text{ В}
\end{equation}

Найдём $V_D$:

\begin{equation}
  V_D = \frac{250 + 2 \cdot \dfrac{14450}{373}}{7} = \frac{122150}{2611} \approx 46,78 \text{ В}
\end{equation}

\subsection*{5. Определение токов в ветвях}

Подставим найденные значения потенциалов в выражения для токов:

\begin{align}
  I_1 & = \frac{25}{7} \approx 3,571 \text{ А}                             \\[3mm]
  I_2 & = \frac{50 - V_E}{7} = \frac{4200}{2611} \approx 1,609 \text{ А}   \\[3mm]
  I_3 & = \frac{50 - V_D}{12} = \frac{700}{2611} \approx 0,268 \text{ А}   \\[3mm]
  I_4 & = \frac{50 - V_D}{8} = \frac{1050}{2611} \approx 0,402 \text{ А}   \\[3mm]
  I_6 & = \frac{V_D - V_E}{12} = \frac{1750}{2611} \approx 0,670 \text{ А} \\[3mm]
  I_5 & = \frac{V_E}{17} = \frac{14450}{6341} \approx 2,279 \text{ А}
\end{align}

\section*{Сравнение с результатами моделирования}

На рисунке 3 представлена смоделированная в среде Electronics Workbench схема с амперметрами для измерения токов в ветвях. Результаты моделирования совпадают с расчётными данными в пределах погрешности округления.

\begin{figure}[H]
  \centering
  \includegraphics[width=0.75\textwidth]{pics/currents.png}
  \caption*{Рисунок 3 --- Моделирование распределения токов в ветвях электрической цепи}
\end{figure}

\subsection*{6. Проверка по первому закону Кирхгофа}

\textbf{Узел $D$:}
\begin{equation}
  I_3 + I_4 = \frac{700}{2611} + \frac{1050}{2611} = \frac{1750}{2611} = I_6 \quad \checkmark
\end{equation}

\textbf{Узел $E$:}
\begin{equation}
  I_2 + I_6 = \frac{4200}{2611} + \frac{1750}{2611} = \frac{5950}{2611} = \frac{14450}{6341} = I_5 \quad \checkmark
\end{equation}

Суммарный ток, отдаваемый источником:
\begin{equation}
  I_{\text{ист}} = I_1 + I_2 + I_3 + I_4 \approx 3,571 + 1,609 + 0,268 + 0,402 \approx 5,850 \text{ А}
\end{equation}

\section*{Баланс мощностей}
\addcontentsline{toc}{section}{Баланс мощностей}

Для проверки правильности расчётов составим баланс мощностей.

\begin{table}[H]
  \centering
  \caption{Токи, падения напряжений и мощности на резисторах}
  \begin{tabular}{@{}lccc@{}}
    \toprule
    Резистор      & Ток $I$, А & Напряжение $U$, В & Мощность $P$, Вт \\
    \midrule
    $R_1 = 14$ Ом & 3,571      & 50,000            & 178,571          \\
    $R_2 = 7$ Ом  & 1,609      & 11,260            & 18,112           \\
    $R_3 = 12$ Ом & 0,268      & 3,217             & 0,862            \\
    $R_4 = 8$ Ом  & 0,402      & 3,217             & 1,294            \\
    $R_5 = 17$ Ом & 2,279      & 38,740            & 88,282           \\
    $R_6 = 12$ Ом & 0,670      & 8,043             & 5,391            \\
    \bottomrule
  \end{tabular}
\end{table}

Сумма рассеиваемых мощностей на резисторах:
\begin{equation}
  \sum_{i=1}^{6} P_i = 178,571 + 18,112 + 0,862 + 1,294 + 88,282 + 5,391 = 292,512 \text{ Вт}
\end{equation}

Мощность, отданная источником:
\begin{equation}
  P_{\text{ист}} = E \cdot I_{\text{ист}} = 50 \cdot 5,850 = 292,512 \text{ Вт}
\end{equation}

Проверка баланса:
\begin{equation}
  P_{\text{ист}} = \sum_{i=1}^{6} P_i = 292,512 \text{ Вт} \quad \checkmark
\end{equation}

Баланс мощностей выполнен, что подтверждает правильность проведённых расчётов.

\newpage

\section*{Вывод}

\end{document}