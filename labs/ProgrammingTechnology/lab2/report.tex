\documentclass[oneside,a4paper,14pt]{extarticle}
\usepackage[a4paper,letterpaper,top=20mm,bottom=20mm,left=20mm,right=10mm]{geometry}
\usepackage[russian]{babel}
\usepackage{indentfirst}
\usepackage{graphicx}
\usepackage{caption}
\usepackage{titlesec}
\usepackage{minted}
\usepackage{hyperref}
\usepackage{enumitem}

% Форматирование листингов
\setminted{style = rainbow_dash, fontsize = \small}

% Форматирование заголовков
\titleformat{\section}{\normalsize\bfseries}{\thesection}{1em}{}
\titleformat{\subsection}{\normalsize\bfseries}{\thesubsection}{1em}{}
\titleformat{\subsubsection}{\normalsize\bfseries}{\thesubsubsection}{1em}{}

\renewcommand\baselinestretch{1.33}
\setlength{\parindent}{1.25cm}

\setlist[enumerate]{
  left=\parindent,
  label=\arabic*.,
  itemsep=0pt,
  topsep=5pt,
  partopsep=0pt,
  parsep=0pt
}

\setlist[itemize]{
  left=\parindent,
  itemsep=0pt,
  topsep=5pt,
  partopsep=0pt,
  parsep=0pt
}

\hypersetup{
  colorlinks=true,
  linkcolor=black,
  urlcolor=blue,
  pdfborder={0 0 0},
  pdftitle={Веб-приложение на Express},
  pdfauthor={Черкасов А.А.}
}

\begin{document}

\newpage
\thispagestyle{empty}
\begin{center}
  МИНИСТЕРСТВО НАУКИ И ВЫСШЕГО ОБРАЗОВАНИЯ РОССИЙСКОЙ ФЕДЕРАЦИИ\\
  ФЕДЕРАЛЬНОЕ ГОСУДАРСТВЕННОЕ БЮДЖЕТНОЕ ОБРАЗОВАТЕЛЬНОЕ УЧРЕЖДЕНИЕ ВЫСШЕГО ОБРАЗОВАНИЯ\\
  «ВЯТСКИЙ ГОСУДАРСТВЕННЫЙ УНИВЕРСИТЕТ»\\
  Институт математики и информационных систем\\
  Факультет автоматики и вычислительной техники\\
  Кафедра электронных вычислительных машин
\end{center}
\vspace{10mm}

\hfill
\begin{tabular}{l}
  \footnotesize Дата сдачи на проверку:                                          \\
  \footnotesize <<\rule[-1mm]{5mm}{0.10mm}\/>>\rule[-1mm]{20mm}{0.10mm}\ 2025 г. \\
  \footnotesize Проверено:                                                       \\
  \footnotesize <<\rule[-1mm]{5mm}{0.10mm}\/>>\rule[-1mm]{20mm}{0.10mm}\ 2025 г. \\
\end{tabular}
\vfill

\begin{center}
  Разработка клиент–серверного веб-приложения на HTML, CSS, JavaScript и Express.js.\\
  Отчёт по лабораторной работе №2\\
  по дисциплине\\
  «Технологии Программирования»
\end{center}

\vspace{25mm}

\noindent
\begin{tabular}{ll}
  Разработал студент гр. ИВТб-2301-05-00 & \hspace{18mm}\rule[-1mm]{30mm}{0.10mm}\,/Черкасов А. А./ \\
                                         & \hspace{25.5mm}\footnotesize(подпись)                    \\
  Преподаватель                          & \hspace{18mm}\rule[-1mm]{30mm}{0.10mm}\,/Пащенко Д. Э./  \\
                                         & \hspace{25.5mm}\footnotesize(подпись)                    \\
\end{tabular}

\vfill
\begin{center}
  Киров\\
  2025
\end{center}

\newpage
\section*{Цели лабораторной работы}

\begin{itemize}
  \item[$-$] изучить структуру клиент–серверных веб-приложений с файловой загрузкой;
  \item[$-$] освоить работу с Express.js, multer для обработки файлов и JSON хранилищем;
  \item[$-$] реализовать систему голосования с защитой от повторного голосования;
  \item[$-$] разработать интерактивную галерею изображений с модальными окнами;
  \item[$-$] закрепить навыки работы с асинхронными запросами и DOM манипуляциями.
\end{itemize}

\section*{Задание}

Создать веб-приложение «Meme Repo» — репозиторий мемов с системой голосования, удовлетворяющее требованиям:

\begin{enumerate}
  \item На стороне клиента реализовать две HTML-страницы: загрузки и галереи изображений.
  \item Стили разместить в отдельном файле CSS, логику обработки — в отдельных JS файлах.
  \item На стороне сервера реализовать Express.js с multer для обработки загрузки файлов.
  \item Реализовать JSON-хранилище для метаданных изображений и голосов.
  \item Создать систему голосования с защитой от повторного голосования по IP.
  \item Организовать вывод топ-10 изображений для фонового отображения.
  \item Реализовать модальные окна для просмотра изображений в полном размере.
\end{enumerate}

\newpage
\section*{Ход выполнения работы}

\subsection*{1. Создание структуры проекта}

Проект «Meme Repo» был создан в соответствии с архитектурой веб-приложения:

\begin{itemize}
  \item \texttt{index.html} — страница загрузки мемов с формой;
  \item \texttt{gallery.html} — страница галереи с системой голосования;
  \item \texttt{style.css} — стили оформления для обеих страниц;
  \item \texttt{script.js} — клиентская логика для формы загрузки;
  \item \texttt{server.js} — сервер Express.js с API для загрузки и голосования;
  \item \texttt{images.json} — хранилище метаданных изображений;
  \item \texttt{votes.json} — хранилище данных голосования;
  \item \texttt{public/uploads/} — директория для загруженных изображений.
\end{itemize}

\subsection*{2. Реализация клиентской части}

Были разработаны две HTML-страницы с единой системой стилей:

\begin{itemize}
  \item \textbf{Страница загрузки} (\texttt{index.html}): форма с полями имени, email и выбора файла изображения;
  \item \textbf{Страница галереи} (\texttt{gallery.html}): динамически загружаемая галерея с кнопками голосования;
  \item \textbf{Модальные окна}: полноразмерный просмотр изображений с возможностью закрытия;
  \item \textbf{Фоновые элементы}: анимированные миниатюры топ-10 мемов для атмосферы.
\end{itemize}

JavaScript реализует асинхронные запросы к API сервера для загрузки изображений и голосования.

\subsection*{3. Реализация серверной части}

Сервер был создан на Express.js с использованием следующих модулей:

\begin{itemize}
  \item \texttt{express} — основной веб-фреймворк;
  \item \texttt{multer} — middleware для обработки multipart/form-data (загрузка файлов);
  \item \texttt{fs/path} — работа с файловой системой.
\end{itemize}

На сервере реализована следующая функциональность:

\begin{itemize}
  \item \textbf{API эндпоинты}:
        \begin{itemize}
          \item \texttt{GET /images} — получение топ-10 изображений для фона;
          \item \texttt{GET /gallery} — получение всех изображений с метаданными;
          \item \texttt{POST /vote} — голосование за изображение (с защитой от повторного голосования);
          \item \texttt{POST /} — загрузка нового изображения.
        \end{itemize}
  \item \textbf{Файловое хранилище}: сохранение изображений в \texttt{public/uploads/} с уникальными именами;
  \item \textbf{JSON-хранилище}: метаданные изображений и голоса хранятся в JSON-файлах;
  \item \textbf{Защита от спама}: IP-based ограничение голосования (один голос на изображение);
  \item \textbf{Обработка ошибок}: валидация данных и корректные HTTP-статусы.
\end{itemize}

\subsection*{4. Система голосования}

Реализована система голосования с следующими особенностями:

\begin{itemize}
  \item Два типа голосов: положительный и отрицательный;
  \item Защита от повторного голосования по IP-адресу;
  \item Автоматическая сортировка изображений по количеству голосов;
  \item Обновление интерфейса в реальном времени после голосования.
\end{itemize}

\subsection*{5. Тестирование работы приложения}

После запуска сервера:

\begin{verbatim}
node server.js
\end{verbatim}

Приложение доступно по адресу \texttt{http://localhost:8080}.

Были протестированы следующие сценарии:

\begin{itemize}
  \item Загрузка изображений с метаданными;
  \item Просмотр галереи с сортировкой по голосам;
  \item Голосование за изображения (с защитой от повторного голосования);
  \item Полноразмерный просмотр изображений в модальных окнах;
  \item Анимированный фон с миниатюрами популярных мемов.
\end{itemize}

\subsection*{6. Скриншоты}

\begin{figure}[H] \centering \includegraphics[width=0.6\textwidth]{pics/form.png} \caption*{Рисунок 1 --- Форма загрузки мемов} \end{figure}

\begin{figure}[H] \centering \includegraphics[width=0.6\textwidth]{pics/gallery.png} \caption*{Рисунок 2 --- Галлерея загруженных мемов} \end{figure}

\newpage
\section*{Вывод}

В ходе выполнения лабораторной работы было разработано полнофункциональное веб-приложение «Meme Repo» с системой голосования.
Были изучены современные подходы к разработке клиент-серверных приложений с файловой загрузкой.

\textbf{Ключевые достижения:}

\begin{itemize}
  \item Реализована комплексная система загрузки и хранения изображений с метаданными;
  \item Создан интерактивный интерфейс галереи с модальными окнами и системой голосования;
  \item Внедрена защита от повторного голосования на основе IP-адресов;
  \item Организовано JSON-хранилище для персистентности данных;
  \item Разработан responsive дизайн с анимированными элементами интерфейса.
\end{itemize}

Работа позволила закрепить навыки работы с Express.js, асинхронными запросами, файловой системой и современными веб-технологиями.
Приложение демонстрирует практическое применение архитектурных паттернов веб-разработки и принципов пользовательского интерфейса.

\newpage
\section*{Приложение А1. Исходный код index.html}
\inputminted{html}{code/public/index.html}

\section*{Приложение А2. Исходный код gallery.html}
\inputminted{html}{code/public/gallery.html}

\section*{Приложение А3. Исходный код script.js}
\inputminted{js}{code/public/script.js}

\section*{Приложение А4. Исходный код style.css}
\inputminted{css}{code/public/style.css}

\section*{Приложение А5. Исходный код server.js}
\inputminted{js}{code/server.js}

\end{document}
