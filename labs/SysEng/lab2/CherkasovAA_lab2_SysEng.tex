\documentclass[oneside,a4paper,14pt]{extarticle}
\usepackage[T2A]{fontenc}
\usepackage[utf8]{inputenc}
\usepackage[russian]{babel}
\usepackage[a4paper,top=20mm,bottom=20mm,left=30mm,right=10mm]{geometry}
\usepackage{indentfirst}
\usepackage{titlesec}
\usepackage{enumitem}
\usepackage{array}
\usepackage{hyperref}
\usepackage{rotating} 

% Форматирование заголовков
\titleformat{\section}{\normalsize\bfseries}{\thesection}{1em}{}
\titleformat{\subsection}{\normalsize\bfseries}{\thesubsection}{1em}{}
\titleformat{\subsubsection}{\normalsize\bfseries}{\thesubsubsection}{1em}{}

% Интерлиньяж и абзац
\renewcommand\baselinestretch{1.33}

\setlength{\parindent}{1.25cm}  % длина красной строки

% Для всех списков
\setlist[enumerate]{
    left=\parindent,       % отступ слева
    label=\arabic*.,       % цифры
    itemsep=0pt,           % расстояние между пунктами
    topsep=5pt,            % отступ сверху
    partopsep=0pt,         % дополнительный отступ сверху, если абзац до списка
    parsep=0pt             % отступ между абзацами внутри пункта
}

\setlist[itemize]{
    left=\parindent,       % отступ слева
    itemsep=0pt,           % расстояние между пунктами
    topsep=5pt,            % отступ сверху
    partopsep=0pt,
    parsep=0pt
}


% Гиперссылки
\hypersetup{
  colorlinks=true,
  linkcolor=black,
  urlcolor=blue,
  pdfborder={0 0 0},
  pdftitle={Техническое задание на разработку системы "ПозорДом"},
  pdfauthor={Черкасов А.А., Макаров С.А.}
}

\begin{document}

\thispagestyle{empty}

\vspace*{15mm}

\begin{center}
    \begin{tabular}{c@{\hspace{25mm}}c}
        УТВЕРЖДАЮ               & УТВЕРЖДАЮ             \\[10mm]
        \textbf{Заказчик:}      & \textbf{Исполнители:} \\[5mm]
        Заведующий кафедрой ЭВМ & ИП «Работает и ладно» \\[2mm]

        \begin{minipage}{0.4\textwidth}
            \centering
            ФГБОУ ВО «ВятГУ» \\[5mm]
            \rule[-1mm]{50mm}{0.15mm} /Долженкова М.Л./ \\
            «\rule{8mm}{0.15mm}» \rule{30mm}{0.15mm} 2025 г.
        \end{minipage}
                                &
        \begin{minipage}{0.4\textwidth}
            \centering
            Разработчик серверной части: \\[5mm]
            \rule[-1mm]{50mm}{0.15mm} /Черкасов А.А./ \\[4mm]
            Разработчик мобильного приложения: \\[5mm]
            \rule[-1mm]{50mm}{0.15mm} /Макаров С.А./ \\[4mm]
            «\rule{8mm}{0.15mm}» \rule{30mm}{0.15mm} 2025 г.
        \end{minipage}
    \end{tabular}
\end{center}

\vspace{20mm}
\begin{center}
    \textbf{РАЗРАБОТКА СИСТЕМЫ УМНОГО ДОМА «ПОЗОРДОМ»}

    \vspace{5mm}
    Техническое задание

    \vspace{5mm}
    Листов 12
\end{center}

\newpage
\tableofcontents
\newpage

\section{Общие сведения}

\subsection{Полное и краткое наименование системы}
\begin{itemize}
    \item[-] Полное наименование: Система умного дома «ПозорДом»
    \item[-] Краткое наименование: «ПозорДом»
\end{itemize}

\subsection{Шифр темы или шифр (номер) договора}
Шифр темы отсутствует.

\subsection{Наименование организации-заказчика}
ФГБОУ ВО «ВятГУ», кафедра ЭВМ.
Заказчик: Долженкова Мария Львовна, заведующая кафедрой ЭВМ.

\subsection{Наименование организации-разработчика}
Индивидуальный предприниматель Черкасов А.А. (ИП «Работает и ладно»).

\subsection{Перечень документов, на основании которых выполняется разработка Системы}
Основанием для разработки Системы является настоящее Техническое задание.

\subsection{Плановые сроки начала и окончания работ}
Срок начала работ: «10» сентября 2025 г.\\
Срок окончания работ: «20» декабря 2025 г.

\subsection{Общие сведения об источниках и порядке финансирования работ}
Финансирование работ осуществляется за счёт средств Заказчика. Порядок финансирования определяется Договором.

\section{Цели и назначение системы}

\subsection{Назначение системы}
Система «ПозорДом» предназначена для управления устройствами умного дома (освещение, климат, датчики, электроприборы). Она обеспечивает удобное и безопасное управление как в самом доме, так и на расстоянии.

\subsection{Цели создания системы}
\begin{itemize}
    \item[-] Объединить разные устройства в единую систему.
    \item[-] Обеспечить безопасный доступ к ним.
    \item[-] Сделать управление простым и понятным.
    \item[-] Повысить надёжность работы дома.
\end{itemize}

\section{Характеристика объектов автоматизации}

\subsection{Основные сведения об объекте автоматизации}
Объектом автоматизации является жилое помещение, оснащённое устройствами умного дома (датчики, реле, освещение, климат-контроль и др.), управляемыми через локальную сеть и/или удалённый доступ.

\subsection{Сведения об условиях эксплуатации объекта}
Условия эксплуатации оборудования должны соответствовать действующим нормам безопасности.
Рабочий диапазон температур: $-10^\circ$C \dots $+40^\circ$C.
Электропитание: 120–240 В.
Работа в обычных бытовых локальных сетях.

\section{Требования к системе}

\subsection{Требования к структуре системы в целом}

\subsubsection{Общие принципы}
\begin{itemize}
    \item[-] Все компоненты системы должны работать как единое целое.
    \item[-] Возможность добавления новых устройств.
    \item[-] Совместимость с современными устройствами.
    \item[-] Единый стиль интерфейса.
    \item[-] Модульная архитектура для упрощённого развития.
\end{itemize}

\subsubsection{Перечень подсистем}
\begin{enumerate}
    \item \textbf{Устройство управления} — координация работы всех подключённых устройств.
    \item \textbf{Сервер удалённого доступа} — соединение из любой точки.
    \item \textbf{Мобильное приложение} — удобное управление системой.
\end{enumerate}

\subsubsection{Способы информационного обмена}
Используются современные методы передачи данных и безопасные каналы связи.

\subsubsection{Взаимодействие со смежными системами}
Система является автономной и не требует внешней интеграции.

\subsubsection{Режимы функционирования}
\begin{itemize}
    \item[-] Локальный режим: работа внутри дома без интернета.
    \item[-] Удалённый режим: управление через сеть с аутентификацией.
\end{itemize}

\subsubsection{Диагностика}
Система ведёт журнал событий. В приложении предусмотрен раздел «Диагностика».

\subsubsection{Перспективы развития}
Возможна интеграция с голосовыми ассистентами, поддержка новых стандартов умных домов.

\subsection{Требования к функциям}
Система должна:
\begin{itemize}
    \item[-] Подключать и управлять различными устройствами.
    \item[-] Обеспечивать локальный и удалённый доступ.
    \item[-] Определять режим работы автоматически.
    \item[-] Обеспечивать защиту данных.
    \item[-] Выполнять резервное копирование каждые 6 часов.
    \item[-] Работать без интернета.
\end{itemize}

\subsection{Требования к видам обеспечения}

\subsubsection{Информационное обеспечение}
Данные хранятся локально. Сервер не сохраняет персональные данные.

\subsubsection{Лингвистическое обеспечение}
Интерфейс и документация — на русском языке.

\subsubsection{Техническое обеспечение}
\begin{enumerate}
    \item Устройство управления: компактный компьютер.
    \item Сервер: размещается у заказчика или в облаке.
    \item Приложение: поддержка современных смартфонов.
\end{enumerate}

\subsection{Общие технические требования}

\subsubsection{Персонал}
Эксплуатация возможна пользователем без специальных навыков.

\subsubsection{Основные показатели}
\begin{itemize}
    \item[-] Время отклика $\leq$ 1 с (локальная сеть).
    \item[-] До 100 устройств на одно устройство управления.
    \item[-] До 10 одновременных пользователей.
\end{itemize}

\subsubsection{Надёжность}
\begin{itemize}
    \item[-] Резервное копирование каждые 6 часов.
    \item[-] Работа без интернета.
    \item[-] Возможность восстановления.
\end{itemize}

\subsubsection{Безопасность}
\begin{itemize}
    \item[-] Обязательная аутентификация.
    \item[-] Использование надёжного шифрования.
    \item[-] Отсутствие хранения личных данных на сервере.
\end{itemize}

\subsubsection{Эргономика}
Интерфейс — интуитивный, адаптивный, на русском языке.

\section{Состав и содержание работ}
\begin{enumerate}
    \item Анализ требований, проектирование — 10.09–20.09.
    \item Разработка устройства управления — 20.09–15.10.
    \item Настройка сервера — 15.10–25.10.
    \item Разработка приложения — 25.10–15.11.
    \item Интеграция и тестирование — 15.11–01.12.
    \item Документирование, сдача — 01.12–20.12.
\end{enumerate}

\section{Порядок развития системы}
Работы выполняются в соответствии с настоящим ТЗ. Все этапы согласовываются с Заказчиком.

\section{Контроль и приёмка}
Испытания включают:
\begin{enumerate}
    \item Предварительные проверки.
    \item Опытную эксплуатацию (2 недели).
    \item Приёмочные испытания.
\end{enumerate}

Результаты оформляются протоколами и актами.

\section{Подготовка к вводу системы}
Разработчик устанавливает систему, проводит обучение пользователей.

\section{Документирование}
\begin{enumerate}
    \item Техническое задание.
    \item Руководство пользователя.
    \item Описание архитектуры.
    \item Отчёт о тестировании.
    \item Исходные материалы.
\end{enumerate}

Документация передаётся в бумажном и электронном виде.

\section{Источники разработки}
Настоящее ТЗ и требования Заказчика.

\newpage
\begin{sidewaystable}
    \centering
    \textbf{СОСТАВИЛИ} \\[3mm]
    \begin{tabular}{|p{60mm}|p{50mm}|p{50mm}|p{25mm}|p{25mm}|}
        \hline
        Наименование организации & Должность                         & Ф.И.О.        & Подпись & Дата \\
        \hline
        ИП «Работает и ладно»    & Разработчик серверной части       & Черкасов А.А. &         &      \\
        \hline
        ИП «Работает и ладно»    & Разработчик мобильного приложения & Макаров С.А.  &         &      \\
        \hline
    \end{tabular}

    \vspace{20mm}

    \textbf{СОГЛАСОВАНО} \\[3mm]
    \begin{tabular}{|p{60mm}|p{50mm}|p{50mm}|p{25mm}|p{25mm}|}
        \hline
        Наименование организации & Должность               & Ф.И.О.          & Подпись & Дата \\
        \hline
        ФГБОУ ВО «ВятГУ»         & Заведующий кафедрой ЭВМ & Долженкова М.Л. &         &      \\
        \hline
    \end{tabular}
\end{sidewaystable}

\end{document}

