\documentclass[oneside,a4paper,14pt]{extarticle}
\usepackage[T2A]{fontenc}
\usepackage[utf8]{inputenc}
\usepackage[russian]{babel}
\usepackage[a4paper,top=20mm,bottom=20mm,left=30mm,right=10mm]{geometry}
\usepackage{indentfirst}
\usepackage{titlesec}
\usepackage{enumitem}
\usepackage{array}
\usepackage{hyperref}
\usepackage{rotating} 

% Форматирование заголовков
\titleformat{\section}{\normalsize\bfseries}{\thesection}{1em}{}
\titleformat{\subsection}{\normalsize\bfseries}{\thesubsection}{1em}{}
\titleformat{\subsubsection}{\normalsize\bfseries}{\thesubsubsection}{1em}{}

% Интерлиньяж и абзац
\renewcommand\baselinestretch{1.33}
\setlength{\parindent}{1.25cm}

% Форматирование списков
\setlist[enumerate]{
    left=\parindent,
    label=\arabic*.,
    itemsep=0pt,
    topsep=5pt,
    partopsep=0pt,
    parsep=0pt
}
\setlist[itemize]{
    left=\parindent,
    itemsep=0pt,
    topsep=5pt,
    partopsep=0pt,
    parsep=0pt
}

% Гиперссылки
\hypersetup{
  colorlinks=true,
  linkcolor=black,
  urlcolor=blue,
  pdfborder={0 0 0},
  pdftitle={Техническое задание на разработку системы «ПозорДом»},
  pdfauthor={Черкасов А.А., Макаров С.А.}
}

\begin{document}
\thispagestyle{empty}

\vspace*{15mm}

\begin{center}
    \begin{tabular}{c@{\hspace{25mm}}c}
        УТВЕРЖДАЮ               & УТВЕРЖДАЮ             \\[10mm]
        \textbf{Заказчик:}      & \textbf{Исполнители:} \\[5mm]
        Заведующий кафедрой ЭВМ & ИП «Работает и ладно» \\[2mm]

        \begin{minipage}{0.4\textwidth}
            \centering
            ФГБОУ ВО «ВятГУ» \\[5mm]
            \rule[-1mm]{50mm}{0.15mm} /Долженкова М.Л./ \\
            «\rule{8mm}{0.15mm}» \rule{30mm}{0.15mm} 2025 г.
        \end{minipage}
                                &
        \begin{minipage}{0.4\textwidth}
            \centering
            Разработчик серверной части: \\[5mm]
            \rule[-1mm]{50mm}{0.15mm} /Черкасов А.А./ \\[4mm]
            Разработчик мобильного приложения: \\[5mm]
            \rule[-1mm]{50mm}{0.15mm} /Макаров С.А./ \\[4mm]
            «\rule{8mm}{0.15mm}» \rule{30mm}{0.15mm} 2025 г.
        \end{minipage}
    \end{tabular}
\end{center}

\vspace{20mm}
\begin{center}
    \textbf{РАЗРАБОТКА СИСТЕМЫ УМНОГО ДОМА «ПОЗОРДОМ»}

    \vspace{5mm}
    Техническое задание

    \vspace{5mm}
    Листов 12
\end{center}

\newpage
\tableofcontents
\newpage

\section{Общие сведения}

\subsection{Полное и краткое наименование системы}
\begin{itemize}
    \item Полное наименование: Система умного дома «ПозорДом».
    \item Краткое наименование: «ПозорДом».
\end{itemize}

\subsection{Шифр темы или договора}
Шифр темы отсутствует.

\subsection{Наименование организации-заказчика}
ФГБОУ ВО «ВятГУ», кафедра ЭВМ.  
Заказчик: Долженкова Мария Львовна, заведующая кафедрой ЭВМ.

\subsection{Наименование организации-разработчика}
Индивидуальный предприниматель Черкасов А.А. (ИП «Работает и ладно»).

\subsection{Перечень документов, на основании которых выполняется разработка системы}
Основанием для разработки является настоящее техническое задание.

\subsection{Плановые сроки выполнения работ}
Начало работ: «10» сентября 2025 г.  
Окончание работ: «20» декабря 2025 г.

\subsection{Финансирование}
Финансирование осуществляется за счёт средств заказчика. Порядок финансирования определяется договором.

\section{Цели и назначение системы}

\subsection{Назначение системы}
Система «ПозорДом» предназначена для управления устройствами умного дома (освещение, климат, датчики, электроприборы). Обеспечивает удобное и безопасное управление как локально, так и удалённо.

\subsection{Цели создания}
\begin{itemize}
    \item Интеграция устройств в единую систему.
    \item Безопасный доступ и управление.
    \item Простота и интуитивность интерфейса.
    \item Повышение надёжности эксплуатации.
\end{itemize}

\section{Характеристика объектов автоматизации}

\subsection{Основные сведения}
Объект автоматизации — жилое помещение, оснащённое устройствами умного дома (датчики, реле, освещение, климат-контроль и др.), управляемыми через локальную сеть или интернет.

\subsection{Условия эксплуатации}
Рабочая температура: $-10^\circ$C … $+40^\circ$C.  
Электропитание: 120–240 В.  
Сеть: стандартные бытовые локальные сети.

\section{Требования к системе}

\subsection{Требования к структуре}

\subsubsection{Общие принципы}
\begin{itemize}
    \item Модульная архитектура.
    \item Совместимость с современными устройствами.
    \item Возможность расширения.
    \item Единый стиль интерфейса.
\end{itemize}

\subsubsection{Подсистемы}
\begin{enumerate}
    \item Устройство управления.
    \item Сервер удалённого доступа.
    \item Мобильное приложение.
\end{enumerate}

\subsubsection{Протоколы и подключаемые устройства}
Поддержка: Wi-Fi, BLE, Zigbee, Ethernet, RS-485.  
Добавление новых устройств через автоопределение протокола, регистрацию и синхронизацию.

\subsubsection{Режимы работы}
\begin{itemize}
    \item Локальный (без интернета).
    \item Удалённый (с аутентификацией).
\end{itemize}

\subsubsection{Диагностика}
Журнал событий и раздел «Диагностика» в приложении.

\subsubsection{Перспективы развития}
Интеграция с голосовыми ассистентами, поддержка новых стандартов.

\subsection{Требования к функциям}
Система должна:
\begin{itemize}
    \item Подключать и управлять устройствами.
    \item Определять тип подключения (локальное/удалённое).
    \item Синхронизировать данные между серверами.
    \item Работать автономно без интернета.
\end{itemize}

\subsection{Обеспечение системы}
\subsubsection{Информационное обеспечение}
Данные хранятся локально. Персональные данные на сервере не сохраняются.

\subsubsection{Лингвистическое обеспечение}
Интерфейс и документация — на русском языке.

\subsubsection{Техническое обеспечение}
\begin{enumerate}
    \item Устройство управления — компактный компьютер.
    \item Сервер — у заказчика или в облаке.
    \item Мобильное приложение — современные смартфоны.
\end{enumerate}

\subsection{Общие технические требования}
\subsubsection{Персонал}
Эксплуатация доступна без специальных навыков.

\subsubsection{Показатели}
\begin{itemize}
    \item Время отклика $\leq$ 1 с (локально).
    \item До 100 устройств на одно управляющее устройство.
    \item До 10 пользователей одновременно.
\end{itemize}

\subsubsection{Надёжность}
\begin{itemize}
    \item Резервное копирование каждые 6 часов.
    \item Возможность работы без интернета.
    \item Восстановление системы после сбоев.
\end{itemize}

\subsubsection{Безопасность}
\begin{itemize}
    \item Аутентификация пользователей.
    \item Надёжное шифрование.
    \item Отсутствие хранения личных данных на сервере.
\end{itemize}

\subsubsection{Эргономика}
Интерфейс — интуитивный, адаптивный.

\section{Состав и содержание работ}
\begin{enumerate}
    \item Анализ требований, проектирование — 10.09–20.09.
    \item Разработка устройства управления — 20.09–15.10.
    \item Настройка сервера — 15.10–25.10.
    \item Разработка приложения — 25.10–15.11.
    \item Интеграция и тестирование — 15.11–01.12.
    \item Документирование и сдача — 01.12–20.12.
\end{enumerate}

\section{Порядок развития системы}
Все этапы согласовываются с заказчиком и выполняются по ТЗ.

\section{Контроль и приёмка}
Испытания:
\begin{enumerate}
    \item Предварительные проверки.
    \item Опытная эксплуатация (2 недели).
    \item Приёмочные испытания.
\end{enumerate}
Результаты оформляются актами и протоколами.

\section{Подготовка к вводу}
Разработчик устанавливает систему и проводит обучение пользователей.

\section{Документирование}
\begin{enumerate}
    \item Техническое задание.
    \item Руководство пользователя.
    \item Описание архитектуры.
    \item Отчёт о тестировании.
    \item Исходные материалы.
\end{enumerate}

Документация предоставляется в бумажном и электронном виде.

\section{Источники разработки}
Настоящее техническое задание и требования заказчика.

\newpage
\begin{sidewaystable}
    \centering
    \textbf{СОСТАВИЛИ} \\[3mm]
    \begin{tabular}{|p{60mm}|p{50mm}|p{50mm}|p{25mm}|p{25mm}|}
        \hline
        Наименование организации & Должность                         & Ф.И.О.        & Подпись & Дата \\
        \hline
        ИП «Работает и ладно»    & Разработчик серверной части       & Черкасов А.А. &         &      \\
        \hline
        ИП «Работает и ладно»    & Разработчик мобильного приложения & Макаров С.А.  &         &      \\
        \hline
    \end{tabular}

    \vspace{20mm}

    \textbf{СОГЛАСОВАНО} \\[3mm]
    \begin{tabular}{|p{60mm}|p{50mm}|p{50mm}|p{25mm}|p{25mm}|}
        \hline
        Наименование организации & Должность               & Ф.И.О.          & Подпись & Дата \\
        \hline
        ФГБОУ ВО «ВятГУ»         & Заведующий кафедрой ЭВМ & Долженкова М.Л. &         &      \\
        \hline
    \end{tabular}
\end{sidewaystable}

\end{document}
