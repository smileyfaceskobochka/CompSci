\documentclass[oneside,a4paper,14pt]{extarticle}
\usepackage[a4paper,letterpaper,top=20mm,bottom=20mm,left=20mm,right=10mm]{geometry}
\usepackage[russian]{babel}
\usepackage{indentfirst}
\usepackage{graphicx}
\usepackage{caption}
\usepackage{titlesec}
\usepackage{minted, fancyvrb}
\usepackage{hyperref}
\usepackage{enumitem}

% Форматирование листингов с кодом
\setminted{style = rainbow_dash, fontsize = \small} % https://pygments.org/styles/

% Форматирование заголовков
\titleformat{\section}{\normalsize\bfseries}{\thesection}{1em}{}
\titleformat{\subsection}{\normalsize\bfseries}{\thesubsection}{1em}{}
\titleformat{\subsubsection}{\normalsize\bfseries}{\thesubsubsection}{1em}{}

% Интерлиньяж и абзац
\renewcommand\baselinestretch{1.33}

\setlength{\parindent}{1.25cm}  % длина красной строки

% Для всех списков
\setlist[enumerate]{
  left=0.5\parindent,       % отступ слева
  label=\arabic*.,       % цифры
  itemsep=0pt,           % расстояние между пунктами
  topsep=5pt,            % отступ сверху
  partopsep=0pt,         % дополнительный отступ сверху, если абзац до списка
  parsep=0pt             % отступ между абзацами внутри пункта
}

\setlist[itemize]{
  left=0.5\parindent,       % отступ слева
  itemsep=0pt,           % расстояние между пунктами
  topsep=5pt,            % отступ сверху
  partopsep=0pt,
  parsep=0pt
}

% Гиперссылки
\hypersetup{
  colorlinks=true,
  linkcolor=black,
  urlcolor=blue,
  pdfborder={0 0 0},
  pdftitle={Бот для игры "Камень-Ножницы-Бумага"},
  pdfauthor={Черкасов А.А.}
}

\begin{document}

\newpage
\thispagestyle{empty}
\begin{center}
  МИНИСТЕРСТВО НАУКИ И ВЫСШЕГО ОБРАЗОВАНИЯ РОССИЙСКОЙ ФЕДЕРАЦИИ ФЕДЕРАЛЬНОЕ ГОСУДАРСТВЕННОЕ БЮДЖЕТНОЕ ОБРАЗОВАТЕЛЬНОЕ УЧРЕЖДЕНИЕ ВЫСШЕГО ОБРАЗОВАНИЯ\\
  <<ВЯТСКИЙ ГОСУДАРСТВЕННЫЙ УНИВЕРСИТЕТ>>\\
  Институт математики и информационных систем\\
  Факультет автоматики и вычислительной техники\\
  Кафедра электронных вычислительных машин
\end{center}
\vspace{10mm}

\hfill
\begin{tabular}{l}
  \footnotesize Дата сдачи на проверку:                                          \\
  \footnotesize <<\rule[-1mm]{5mm}{0.10mm}\/>>\rule[-1mm]{20mm}{0.10mm}\ 2025 г. \\
  \footnotesize Проверено:                                                       \\
  \footnotesize <<\rule[-1mm]{5mm}{0.10mm}\/>>\rule[-1mm]{20mm}{0.10mm}\ 2025 г. \\
\end{tabular}
\vfill

\begin{center}
  Отчёт по лабораторной работе №2\\
  по дисциплине\\
  <<Теория Автоматов>>\\
\end{center}
\vspace{25mm}
\noindent
\begin{tabular}{ll}
  Разработал студент гр. ИВТб-2301-05-00 & \hspace{18mm}\rule[-1mm]{30mm}{0.10mm}\,/Черкасов А. А./ \\
                                         & \hspace{25.5mm}\footnotesize(подпись)                    \\
  Старший Преподователь                  & \hspace{18mm}\rule[-1mm]{30mm}{0.10mm}\,/Мельцов В. Ю./  \\
                                         & \hspace{25.5mm}\footnotesize(подпись)                    \\
\end{tabular}

\noindent
\begin{tabular}{lp{58mm}r}
  Работа защищена &  & \hspace{13mm}<<\rule[-1mm]{5mm}{0.10mm}\/>>\rule[-1mm]{30mm}{0.10mm}\ 2025 г.
\end{tabular}
\vfill

\begin{center}
  Киров\\
  2025
\end{center}

\newpage\thispagestyle{plain}

\section*{Цель лабораторной работы}
Разработка интеллектуального бота для игры "Камень-Ножницы-Бумага" с двумя стратегиями анализа поведения противника.

\section*{Задание}
Разработать бота для игры "Камень-Ножницы-Бумага", который:
\begin{enumerate}
  \item Реализует две стратегии анализа поведения противника: поиск повторяющихся паттернов и частотный анализ с весовой системой
  \item Автоматически переключается между стратегиями при серии поражений
  \item Ведет историю последних 1000 раундов для анализа
  \item Использует бумагу по умолчанию при недостатке данных для прогноза
\end{enumerate}

\section*{Реализация бота}

Бот реализован на языке Pascal и состоит из следующих основных компонентов:

\begin{itemize}
  \item[$-$] \texttt{choose()} $-$ основная функция принятия решения о ходе
  \item[$-$] \texttt{analyzePatterns()} $-$ функция поиска повторяющихся последовательностей
  \item[$-$] \texttt{analyzeFrequency()} $-$ функция частотного анализа с весами
  \item[$-$] \texttt{switchTacticIfNeeded()} $-$ функция переключения стратегий
\end{itemize}

\subsection*{Основные правила игры}

Игра "Камень-Ножницы-Бумага" имеет простые правила:
\begin{itemize}
  \item[$-$] \textbf{Камень} побеждает \textbf{Ножницы}
  \item[$-$] \textbf{Бумага} побеждает \textbf{Камень}
  \item[$-$] \textbf{Ножницы} побеждают \textbf{Бумагу}
\end{itemize}

Бот ведет историю игры, сохраняя ходы противника, собственные ходы и результаты каждого раунда.

\subsection*{Система принятия решений}

\subsubsection*{Инициализация и первый ход}

При получении сигнала о начале игры (previousOpponentChoice = 0) бот:
\begin{enumerate}
  \item Инициализирует все структуры данных
  \item Выбирает \textbf{бумагу} как безопасный первый ход
\end{enumerate}

\subsubsection*{Алгоритм основного цикла}

В каждом раунде бот выполняет следующие шаги:

\begin{enumerate}
  \item \textbf{Запись данных}: Сохраняет ход противника и вычисляет результат предыдущего раунда
  \item \textbf{Обновление счетчика поражений}: Увеличивает при поражении, сбрасывает при победе/ничьей
  \item \textbf{Проверка переключения стратегии}: При 5 поражениях подряд меняет активную стратегию
  \item \textbf{Прогнозирование}: Использует активную стратегию для предсказания хода противника
  \item \textbf{Выбор ответа}: Выбирает контрход или использует бумагу по умолчанию
\end{enumerate}

\subsection*{Стратегии анализа}

\subsubsection*{Метод поиска паттернов}

\textbf{Принцип работы:} Система ищет повторяющиеся последовательности ходов противника в истории игры.

\textbf{Требования:} Минимум 5 завершенных раундов в истории.

\textbf{Детальный алгоритм поиска:}

\begin{enumerate}
  \item \textbf{Выбор длины паттерна}: Начинает с максимальной длины (10 ходов), постепенно уменьшает до 2
  \item \textbf{Определение текущего паттерна}: Берет последние N ходов противника как "текущий паттерн"
  \item \textbf{Поиск совпадений}: Для каждой позиции в истории проверяет совпадение последовательности с текущим паттерном
  \item \textbf{Проверка условия}: Если найдено совпадение И после него есть следующий ход - запоминает этот следующий ход
  \item \textbf{Оценка паттернов}: Вычисляет оценку для каждого найденного совпадения
\end{enumerate}

\textbf{Пример поиска паттерна:}

Для паттерна длины 3 [Камень, Бумага, Ножницы]:
\begin{itemize}
  \item Проверяет позицию 0: [Камень, Бумага, Ножницы] = [Камень, Бумага, Ножницы] $\to$ следующий ход Камень
  \item Проверяет позицию 1: [Бумага, Ножницы, Камень] = [Камень, Бумага, Ножницы] не совпадает
  \item Проверяет позицию 2: [Ножницы, Камень, Бумага] = [Камень, Бумага, Ножницы] не совпадает
  \item Проверяет позицию 3: [Камень, Бумага, Ножницы] = [Камень, Бумага, Ножницы] $\to$ следующий ход Камень
\end{itemize}

\textbf{Формула оценки паттерна:}

Пусть $l$ - длина паттерна, $s$ - базовая оценка. Тогда:
\[
  s = l \times 2.0
\]

\textbf{Модификация по результату раунда, следующего после паттерна:}
\begin{itemize}
  \item[$-$] Если после паттерна бот выиграл: $s = s + 10$
  \item[$-$] Если после паттерна была ничья: $s = s + 3$
  \item[$-$] Если после паттерна бот проиграл: $s = s - 6$
\end{itemize}

\textbf{Пример расчета оценки:}
\begin{itemize}
  \item Паттерн длины 3: базовая оценка = 3 × 2.0 = 6.0
  \item После паттерна бот выиграл: итоговая оценка = 6.0 + 10 = 16.0
  \item После паттерна бот проиграл: итоговая оценка = 6.0 - 6 = 0.0
\end{itemize}

\textbf{Выбор паттерна:}
\begin{itemize}
  \item Если найдена оценка $\geq$ 12 баллов - немедленно использует этот паттерн
  \item После проверки всех длин выбирает паттерн с лучшей оценкой
  \item Если ничего не найдено - возвращает "нет прогноза"
\end{itemize}

\subsubsection*{Метод частотного анализа}

\textbf{Принцип работы:} Анализирует статистическую частоту ходов противника с учетом временных факторов.

\textbf{Требования:} Минимум 3 завершенных раунда.

\textbf{Размер окна анализа:} Анализируются последние 80 раундов (максимум из доступных).

\textbf{Расчет весов для каждого хода:}

Пусть $p$ - позиция хода в окне (от 1 до размера окна), $w$ - размер окна, $n$ - фактор новизны, $b$ - базовый вес. Тогда:
\[
  n = \frac{p}{w}
\]
\[
  b = 0.2 + n \times 0.8
\]

\textbf{Принцип весов:}
\begin{itemize}
  \item Самые старые ходы получают минимальный вес 0.2
  \item Самые новые ходы получают максимальный вес 1.0
  \item Вес линейно увеличивается от старых к новым ходам
\end{itemize}

\textbf{Усиление после поражений:} Если после хода противника бот проиграл, вес этого хода удваивается.

\textbf{Пример расчета весов:}
\begin{itemize}
  \item Самый старый ход в окне: $n = 1/80 = 0.0125$, $b = 0.2 + 0.0125 \times 0.8 = 0.21$
  \item Самый новый ход: $n = 80/80 = 1.0$, $b = 0.2 + 1.0 \times 0.8 = 1.0$
  \item Если после хода бот проиграл: вес удваивается, например 1.0 × 2 = 2.0
\end{itemize}

\textbf{Накопление сумм весов:}
\begin{itemize}
  \item Для каждого хода Камень прибавляется соответствующий вес к сумме Камня
  \item Для каждого хода Бумага прибавляется соответствующий вес к сумме Бумаги
  \item Для каждого хода Ножницы прибавляется соответствующий вес к сумме Ножниц
\end{itemize}

\textbf{Выбор прогноза:} Выбирается ход с максимальной суммой весов. При равенстве сумм выбирается Бумага по умолчанию.

\subsubsection*{Переключение стратегий}

\textbf{Триггер:} 5 поражений подряд.

\textbf{Механизм:}
\begin{itemize}
  \item[$-$] Начало с метода паттернов
  \item[$-$] При серии поражений переключение на частотный анализ
  \item[$-$] Сброс счетчика после переключения
\end{itemize}



\subsection*{Практические примеры}

\subsubsection*{Пример 1: Противник с циклическим паттерном}

Противник играет повторяющийся цикл "Камень-Бумага-Ножницы-Камень-Бумага-Ножницы...".

{\footnotesize
\begin{tabular}{|c|c|p{4cm}|c|c|c|}
  \hline
  Раунд & Ход противника & Что происходит                                                                                               & Прогноз & Наш ход & Результат       \\
  \hline
  1     & Камень         & Инициализация, первый ход                                                                                    & -       & Бумага  & \textbf{Победа} \\
  2     & Бумага         & Недостаточно истории (1 завершенный раунд)                                                                   & -       & Бумага  & Ничья           \\
  3     & Ножницы        & Недостаточно истории (2 раунда)                                                                              & -       & Бумага  & Поражение       \\
  4     & Камень         & Недостаточно истории (3 раунда)                                                                              & -       & Бумага  & \textbf{Победа} \\
  5     & Бумага         & Недостаточно истории (4 раунда)                                                                              & -       & Бумага  & Ничья           \\
  6     & Ножницы        & Минимум достигнут (5 раундов). Ищем паттерны, но они еще слабые                                              & Камень? & Бумага  & Поражение       \\
  7     & Камень         & Найден паттерн [Камень,Бумага,Ножницы] длины 3, повторяется! После первого был Камень. Оценка: 3×2.6+10=17.8 & Камень  & Бумага  & \textbf{Победа} \\
  8     & Бумага         & Паттерн [Бумага,Ножницы,Камень] повторяется. После был Бумага. Оценка высокая                                & Бумага  & Ножницы & \textbf{Победа} \\
  9     & Ножницы        & Паттерн [Ножницы,Камень,Бумага] повторяется. После был Ножницы. Оценка высокая                               & Ножницы & Камень  & \textbf{Победа} \\
  10    & Камень         & Длинные паттерны подтверждаются, цикл распознан                                                              & Камень  & Бумага  & \textbf{Победа} \\
  \hline
\end{tabular}
}


\subsubsection*{Пример 2: Изменение стратегии противника}

Противник сначала часто играет Камень (6 раз), затем переходит на Бумагу.

Используется частотный анализ (допустим, переключились после неудач в предыдущей игре).

  {\footnotesize
    \begin{tabular}{|c|c|p{4cm}|c|c|c|}
      \hline
      Раунд & Ход противника & Распределение весов                                                                                                                                          & Прогноз & Наш ход & Результат       \\
      \hline
      6     & Камень         & Камень доминирует (6 ходов с весами 0.32-1.6, некоторые удвоены из-за проигрышей). Сумма весов: Камень $\approx$ 10, Бумага $\approx$ 0, Ножницы $\approx$ 0 & Камень  & Бумага  & \textbf{Победа} \\
      7     & Камень         & Камень все еще доминирует (7 ходов). Последний имеет вес 1.6. Сумма: Камень $\approx$ 12, остальные $\approx$ 0                                              & Камень  & Бумага  & \textbf{Победа} \\
      8     & Бумага         & Камень: 7 старых ходов, Бумага: 1 новый ход (вес 1.6). Камень все еще впереди: К $\approx$ 10, Б $\approx$ 1.6                                               & Камень  & Бумага  & Ничья           \\
      9     & Бумага         & Камень: 7 старых, Бумага: 2 новых (веса 1.4 и 1.6). Камень начинает терять вес: К $\approx$ 8, Б $\approx$ 3                                                 & Камень  & Бумага  & Ничья           \\
      10    & Бумага         & Бумага: 3 новых хода с высокими весами. Бумага начинает доминировать: Б $\approx$ 4.5, К $\approx$ 6                                                         & Камень  & Бумага  & Ничья           \\
      11    & Ножницы        & Бумага: 3 хода, Ножницы: 1 новый (вес 1.6). Б $\approx$ 4, Н $\approx$ 1.6, К $\approx$ 5                                                                    & Камень  & Бумага  & Поражение       \\
      12    & Бумага         & Бумага: 4 новых хода. Бумага выходит вперед: Б $\approx$ 5.5, К $\approx$ 4, Н $\approx$ 1.3                                                                 & Бумага  & Ножницы & \textbf{Победа} \\
      13    & Бумага         & Бумага: 5 новых ходов, явно доминирует: Б $\approx$ 7, К $\approx$ 3, Н $\approx$ 1                                                                          & Бумага  & Ножницы & \textbf{Победа} \\
      14    & Камень         & Появился новый Камень (вес 1.6). Бумага пока впереди: Б $\approx$ 6, К $\approx$ 4, Н $\approx$ 1                                                            & Бумага  & Ножницы & Поражение       \\
      15    & Ножницы        & Ножницы: 2 хода (последний × 2 от проигрыша = 3.2). Н $\approx$ 4.6, Б $\approx$ 5                                                                           & Бумага  & Ножницы & Ничья           \\
      \hline
    \end{tabular}
  }

\newpage

\subsection*{Схема алгоритма}

\begin{figure}[H]
  \centering
  \includegraphics[width=0.9\textwidth]{pics/flowchart.png}
  \caption*{Рисунок 1 - Схема алгоритма бота}
\end{figure}

\newpage

\section*{Вывод}

В ходе выполнения лабораторной работы №2 был разработан интеллектуальный бот для игры "Камень-Ножницы-Бумага" с двумя стратегиями анализа поведения противника. Бот реализует:

\begin{itemize}
  \item[$-$] Метод поиска повторяющихся паттернов с оценкой эффективности
  \item[$-$] Частотный анализ с весовой системой учета временных факторов
  \item[$-$] Автоматическое переключение между стратегиями при неудачах
  \item[$-$] Ведение подробной истории игры для анализа
\end{itemize}

Бот написан на языке Pascal с использованием объектно-ориентированного подхода. Реализована система состояний, позволяющая отслеживать ходы противника, собственные решения и результаты каждого раунда. Бот эффективно анализирует поведение противника и адаптируется к различным стилям игры.

\end{document}
