\documentclass[oneside,a4paper,14pt]{extarticle}
\usepackage[T1,T2A,TU]{fontenc}
\usepackage[a4paper,letterpaper,top=20mm,bottom=20mm,left=20mm,right=10mm]{geometry}
\usepackage[russian]{babel}
\usepackage{textcomp}
\usepackage{indentfirst}
\usepackage{graphicx}
\usepackage{mwe}
\usepackage{wrapfig}
\usepackage{caption}
\usepackage{amsmath}
\usepackage{amsfonts}
\usepackage{amsthm}
\usepackage{amssymb}
\usepackage[all]{xy}
\usepackage[breaklinks]{hyperref}
\usepackage{titlesec}
\usepackage{verbatim, fancyvrb}

\titleformat{\section} % Настройка формата заголовков секций
{\normalsize\bfseries} % Устанавливает размер шрифта на нормальный и делает его жирным
{\thesection} % Указывает, что номер секции будет отображаться перед заголовком
{1em} % Устанавливает расстояние между номером секции и заголовком в 1em
{} % Дополнительные параметры.

\titleformat{\subsection} % Настройка формата заголовков подсекций
{\normalsize\bfseries} % Устанавливает размер шрифта на нормальный и делает его жирным
{\thesubsection} % Указывает, что номер подсекции будет отображаться перед заголовком
{1em} % Устанавливает расстояние между номером подсекции и заголовком в 1em
{} % Дополнительные параметры.

\titleformat{\subsubsection} % Настройка формата заголовков подподсекций
{\normalsize\bfseries} % Устанавливает размер шрифта на нормальный и делает его жирным
{\thesubsection} % Указывает, что номер подподсекции будет отображаться перед заголовком
{1em} % Устанавливает расстояние между номером подподсекции и заголовком в 1em
{} % Дополнительные параметры.

\renewcommand\baselinestretch{1.45}\normalsize %межстр интервал
\setlength{\parindent}{1.25cm} %длина отступа нового абзаца

\begin{document}
\setcounter{page}{24}
\newpage\thispagestyle{plain}
\section*{Приложение 2.1. Програмный код для задания 1.}
\VerbatimInput[fontsize=\small]{code/1.pas}
\pagebreak
\section*{Приложение 2.2. Програмный код для задания 2.}
\VerbatimInput[fontsize=\small]{code/2.c}
\pagebreak
\section*{Приложение 2.3. Програмный код для задания 3.}
\VerbatimInput[fontsize=\small]{code/3.pas}
\pagebreak
\section*{Приложение 2.4. Програмный код для задания 4.}
\VerbatimInput[fontsize=\small]{code/4.c}
\section*{Приложение 2.5. Програмный код для задания 5.}
\VerbatimInput[fontsize=\small]{code/5.pas}
\pagebreak
\section*{Приложение 2.6. Програмный код для задания 6.}
\VerbatimInput[fontsize=\small]{code/6.c}
\pagebreak
\section*{Приложение 2.7. Програмный код для задания 7.}
\VerbatimInput[fontsize=\small]{code/7.pas}
\pagebreak
\section*{Приложение 2.8. Програмный код для задания 8.}
\VerbatimInput[fontsize=\small]{code/8.c}
\pagebreak
\section*{Приложение 2.9. Програмный код для задания 9.}
\VerbatimInput[fontsize=\small]{code/9.pas}
\pagebreak
\section*{Приложение 2.10. Програмный код для задания 10.}
\VerbatimInput[fontsize=\small]{code/10.c}
\pagebreak
\section*{Приложение 2.11. Програмный код для задания 11.}
\VerbatimInput[fontsize=\small]{code/11.pas}
\pagebreak
\section*{Приложение 2.12. Програмный код для задания 12.}
\VerbatimInput[fontsize=\small]{code/12.c}
\end{document}