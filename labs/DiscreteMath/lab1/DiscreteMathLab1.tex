\documentclass[oneside,a4paper,14pt]{extarticle}
\usepackage[a4paper,letterpaper,top=20mm,bottom=20mm,left=20mm,right=10mm]{geometry}
\usepackage[russian]{babel}
\usepackage{textcomp}
\usepackage{indentfirst}
\usepackage{graphicx}
\usepackage{mwe}
\usepackage{wrapfig}
\usepackage{caption}
\usepackage{amsmath}
\usepackage{amsfonts}
\usepackage{amsthm}
\usepackage{amssymb}
\usepackage[all]{xy}
\usepackage[breaklinks]{hyperref}
\usepackage{titlesec}
\usepackage{xcolor}
\usepackage{nicematrix}
\usepackage{multirow}
\usepackage{tikz}
\usepackage{verbatim, fancyvrb}

\titleformat{\section} % Настройка формата заголовков секций
{\normalsize\bfseries} % Устанавливает размер шрифта на нормальный и делает его жирным
{\thesection} % Указывает, что номер секции будет отображаться перед заголовком
{1em} % Устанавливает расстояние между номером секции и заголовком в 1em
{} % Дополнительные параметры.

\titleformat{\subsection} % Настройка формата заголовков подсекций
{\normalsize\bfseries} % Устанавливает размер шрифта на нормальный и делает его жирным
{\thesubsection} % Указывает, что номер подсекции будет отображаться перед заголовком
{1em} % Устанавливает расстояние между номером подсекции и заголовком в 1em
{} % Дополнительные параметры.

\titleformat{\subsubsection} % Настройка формата заголовков подподсекций
{\normalsize\bfseries} % Устанавливает размер шрифта на нормальный и делает его жирным
{\thesubsection} % Указывает, что номер подподсекции будет отображаться перед заголовком
{1em} % Устанавливает расстояние между номером подподсекции и заголовком в 1em
{} % Дополнительные параметры.

\renewcommand\baselinestretch{1.45}\normalsize %межстр интервал
\setlength{\parindent}{1.25cm} %длина отступа нового абзаца

\begin{document}
\newpage
\thispagestyle{empty}
\begin{center}
	МИНИСТЕРСТВО НАУКИ И ВЫСШЕГО ОБРАЗОВАНИЯ\\
	РОССИЙСКОЙ ФЕДЕРАЦИИ
	ФЕДЕРАЛЬНОЕ ГОСУДАРСТВЕННОЕ БЮДЖЕТНОЕ\\
	ОБРАЗОВАТЕЛЬНОЕ
	УЧРЕЖДЕНИЕ ВЫСШЕГО ОБРАЗОВАНИЯ\\
	«ВЯТСКИЙ ГОСУДАРСТВЕННЫЙ УНИВЕРСИТЕТ»\\
	Институт математики и информационных систем\\
	Факультет автоматики и вычислительной техники\\
	Кафедра электронных вычислительных машин
\end{center}
\vspace{20mm}

\begin{center}
	Отчёт по лабораторной работе №1\\
	по дисциплине\\
	<<Дискретная Математика>>\\
	<<Работа с множествами.>>\\
\end{center}
\vspace{40mm}
\noindent
\begin{tabular}{ll}
	Разработал студент гр. ИВТб-1301-05-00 & \rule[-1mm]{30mm}{0.10mm}\,/Черкасов А. А./ \\
	                                       & \hspace{8mm}\footnotesize(подпись)          \\

	Проверил старший преподаватель         & \rule[-1mm]{30mm}{0.10mm}\,/Пахарева И. В./ \\
	                                       & \hspace{8mm}\footnotesize(подпись)          \\
\end{tabular}

\vfill
\begin{center}
	Киров\\
	2025
\end{center}

\newpage\thispagestyle{plain}
\section*{Цель работы}
Цель работы: Научиться использовать множества на языке Pascal.\\
\section*{Задание}
Требуется реализовать программу для выполнения заданных операций над множествами.\\
Вариант 27.\\
\begin{center}
	\begin{NiceTabular}{|c|c|c|c|c|c|c|c|}[hvlines]
		\hline
		$A$          & $B$          & $C$          & $D$  & $E$  & $X$               & $Y$        & $K$             \\
		\hline
		$\mathbb{R}$ & $\mathbb{Q}$ & $\mathbb{Z}$ & лат. & кир. & $A \cap B \cap C$ & $E \cup D$ & $X \triangle Y$ \\
		\hline
	\end{NiceTabular}
\end{center}

\begin{enumerate}
	\item Программа должна позволять вводить множества с именами (A, B, C и т.д.) за счёт
	      использования строки с заданным синтаксисом (<имя множества> = {<элемент 1>, <элемента
	      2>, ...}) или поэлементного ввода согласно критерию типа множества (по варианту).
	      Множество может включать в себя до десяти элементов.
	\item Программа должна позволять выполнять заданные операции (по вариантам) над введенными
	      множествами с выводом результата.
	\item В случае некорректного введения элемента множества должно появляться информационное
	      сообщение об ошибке.
	\item Определить мощность итогового множества К.
\end{enumerate}

\section*{Решение}

\begin{figure}[h!]
	\centering
	% \includegraphics[height=0.7\textheight]{pics/1.png}
	\caption*{Рисунок 1.1 - Схема прямого счетчика.}
\end{figure}

\begin{flushright}
    \textcolor{black!30}{\textbf{Приложения}}
\end{flushright}
\section*{Приложение А1}
\VerbatimInput[fontsize=\small]{code/main.pas}

\newpage
\section*{Вывод}
В ходе работы удалось научиться использовать и проводить операции над множествами на языке Pascal.\\

\end{document}