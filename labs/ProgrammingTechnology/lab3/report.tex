\documentclass[oneside,a4paper,14pt]{extarticle}
\usepackage[a4paper,letterpaper,top=20mm,bottom=20mm,left=20mm,right=10mm]{geometry}
\usepackage[russian]{babel}
\usepackage{indentfirst}
\usepackage{graphicx}
\usepackage{caption}
\usepackage{titlesec}
\usepackage{minted, fancyvrb}
\usepackage{hyperref}
\usepackage{enumitem}

% Форматирование листингов
\setminted{style = rainbow_dash, fontsize = \small}

% Форматирование заголовков
\titleformat{\section}{\normalsize\bfseries}{\thesection}{1em}{}
\titleformat{\subsection}{\normalsize\bfseries}{\thesubsection}{1em}{}
\titleformat{\subsubsection}{\normalsize\bfseries}{\thesubsubsection}{1em}{}

\renewcommand\baselinestretch{1.33}
\setlength{\parindent}{1.25cm}

\setlist[enumerate]{
  left=\parindent,
  label=\arabic*.,
  itemsep=0pt,
  topsep=5pt,
  partopsep=0pt,
  parsep=0pt
}

\setlist[itemize]{
  left=\parindent,
  itemsep=0pt,
  topsep=5pt,
  partopsep=0pt,
  parsep=0pt
}

\hypersetup{
  colorlinks=true,
  linkcolor=black,
  urlcolor=blue,
  pdfborder={0 0 0},
  pdftitle={Многооконное приложение Qt6},
  pdfauthor={Черкасов А.А.}
}

\begin{document}

\newpage
\thispagestyle{empty}
\begin{center}
  МИНИСТЕРСТВО НАУКИ И ВЫСШЕГО ОБРАЗОВАНИЯ РОССИЙСКОЙ ФЕДЕРАЦИИ\\
  ФЕДЕРАЛЬНОЕ ГОСУДАРСТВЕННОЕ БЮДЖЕТНОЕ ОБРАЗОВАТЕЛЬНОЕ УЧРЕЖДЕНИЕ ВЫСШЕГО ОБРАЗОВАНИЯ\\[1mm]
  «ВЯТСКИЙ ГОСУДАРСТВЕННЫЙ УНИВЕРСИТЕТ»\\
  Институт математики и информационных систем\\
  Факультет автоматики и вычислительной техники\\
  Кафедра электронных вычислительных машин
\end{center}

\vspace{10mm}

\hfill
\begin{tabular}{l}
  \footnotesize Дата сдачи на проверку:                                          \\
  \footnotesize <<\rule[-1mm]{5mm}{0.10mm}\/>>\rule[-1mm]{20mm}{0.10mm}\ 2025 г. \\
  \footnotesize Проверено:                                                       \\
  \footnotesize <<\rule[-1mm]{5mm}{0.10mm}\/>>\rule[-1mm]{20mm}{0.10mm}\ 2025 г. \\
\end{tabular}

\vfill

\begin{center}
  Разработка многооконного приложения на C++ с использованием Qt6.\\
  Отчёт по лабораторной работе №3\\
  по дисциплине\\
  «Технологии программирования»
\end{center}

\vspace{25mm}

\noindent
\begin{tabular}{ll}
  Разработал студент гр. ИВТб-2301-05-00 & \hspace{18mm}\rule[-1mm]{30mm}{0.10mm}\,/Черкасов А. А./ \\
                                         & \hspace{25.5mm}\footnotesize(подпись) \\
  Преподаватель                  & \hspace{18mm}\rule[-1mm]{30mm}{0.10mm}\,/Пащенко Д. Э./   \\
                                         & \hspace{25.5mm}\footnotesize(подпись) \\
\end{tabular}

\vfill

\begin{center}
  Киров\\
  2025
\end{center}

\newpage
\thispagestyle{plain}

\section*{Цели лабораторной работы}

\begin{itemize}
  \item[$-$] изучить основы разработки многооконных приложений на C++ с использованием библиотеки Qt6;
  \item[$-$] освоить механизм передачи данных между главным и дочерними окнами;
  \item[$-$] научиться правильно организовывать интерфейс и взаимодействие компонентов.
\end{itemize}

\section*{Задание}

Разработать приложение с графическим интерфейсом, содержащее минимум два окна:

\begin{enumerate}
  \item Главное окно — отображает данные и содержит кнопки для вызова дочернего окна.
  \item Дочернее окно — содержит элементы ввода; после закрытия передаёт данные обратно в главное окно.
\end{enumerate}

Требования:
\begin{itemize}
  \item[$-$] корректная передача параметров в дочернее окно при открытии;
  \item[$-$] возврат данных в главное окно (через сигнал–слот или callback);
  \item[$-$] обновление элементов интерфейса главного окна на основе полученных данных;
  \item[$-$] обработка ошибок ввода;
  \item[$-$] логическая связь между окнами.
\end{itemize}

\clearpage
\section*{Реализация приложения}

В лабораторной работе используется уже существующее приложение — разработанное ранее приложение на Qt6 (для ЛР по Базам Данных).  
В рамках ЛР №3 оно модифицировано и рассматривается как многооконная система:

\begin{itemize}
  \item \textbf{MainWindow} — главное окно.
  \item \textbf{DeviceDialog} — дочернее окно редактирования сущности.
  \item \textbf{HubDialog} — дополнительное дочернее окно.
\end{itemize}

Передача данных происходит следующим образом:

\begin{enumerate}
  \item Главное окно открывает дочернее, передавая в него выбранную строку.
  \item Дочернее окно изменяет данные.
  \item По нажатию кнопки «Сохранить» окно испускает сигнал \texttt{dataReady()}.
  \item Главное окно принимает сигнал, обновляет таблицу.
\end{enumerate}

\subsection*{Класс MainWindow}

Фрагмент вызова дочернего окна:

\begin{minted}{cpp}
void MainWindow::onAddDeviceClicked() {
    DeviceDialog* dlg = new DeviceDialog(db_, this);
    connect(dlg, &DeviceDialog::deviceSaved,
            this, &MainWindow::refresh_data);
    dlg->show();
}
\end{minted}

\subsection*{Класс DeviceDialog}

Передача данных обратно:

\begin{minted}{cpp}
void DeviceDialog::onSaveClicked() {
    // ... валидация и сохранение ...
    emit deviceSaved();
    close();
}
\end{minted}

Это полностью удовлетворяет требованию передачи данных «из дочернего окна в основное».

\subsection*{Валидация ввода}

Для предотвращения ошибок используется проверка полей:

\begin{minted}{cpp}
if (name_edit_->text().trimmed().isEmpty()) {
    QMessageBox::warning(this, "Ошибка", "Название не может быть пустым");
    return;
}
\end{minted}

\subsection*{Скриншоты работы приложения}

\begin{figure}[H]
  \centering
  \includegraphics[width=0.6\textwidth]{pics/login.png}
  \caption*{Рисунок 1 --- Окно входа}
\end{figure}

\begin{figure}[H]
  \centering
  \includegraphics[width=0.8\textwidth]{pics/main_window.png}
  \caption*{Рисунок 2 --- Главное окно}
\end{figure}

\begin{figure}[H]
  \centering
  \includegraphics[width=0.6\textwidth]{pics/device_dialog.png}
  \caption*{Рисунок 3 --- Дочернее окно (редактирование устройства)}
\end{figure}

\section*{Вывод}

В ходе выполнения лабораторной работы №3 были изучены основные механизмы создания многооконных приложений на языке C++ с использованием Qt6.  
Реализована передача данных между главным и дочерним окном, выполнена корректная обработка событий, валидация ввода и обновление интерфейса.  
Полученный функционал полностью соответствует требованиям методических указаний.

\newpage

\section*{Приложение А1. Исходный код mainwindow.cpp}
\inputminted{cpp}{code/src/mainwindow.cpp}

\section*{Приложение А2. Исходный код dialogs.cpp}
\inputminted{cpp}{code/src/dialogs.cpp}

\section*{Приложение А3. Исходный код dialogs.h}
\inputminted{cpp}{code/src/dialogs.h}

\end{document}
