\documentclass[oneside,a4paper,14pt]{extarticle}
\usepackage[a4paper,letterpaper,top=20mm,bottom=20mm,left=20mm,right=10mm]{geometry}
\usepackage[russian]{babel}
\usepackage{indentfirst}
\usepackage{graphicx}
\usepackage{caption}
\usepackage{titlesec}
\usepackage{minted, fancyvrb}
\usepackage{hyperref}
\usepackage{enumitem}

% Форматирование листингов с кодом
\setminted{style = rainbow_dash, fontsize = \small} % https://pygments.org/styles/

% Форматирование заголовков
\titleformat{\section}{\normalsize\bfseries}{\thesection}{1em}{}
\titleformat{\subsection}{\normalsize\bfseries}{\thesubsection}{1em}{}
\titleformat{\subsubsection}{\normalsize\bfseries}{\thesubsubsection}{1em}{}

% Интерлиньяж и абзац
\renewcommand\baselinestretch{1.33}

\setlength{\parindent}{1.25cm}  % длина красной строки

% Для всех списков
\setlist[enumerate]{
  left=0.5\parindent,       % отступ слева
  label=\arabic*.,       % цифры
  itemsep=0pt,           % расстояние между пунктами
  topsep=5pt,            % отступ сверху
  partopsep=0pt,         % дополнительный отступ сверху, если абзац до списка
  parsep=0pt             % отступ между абзацами внутри пункта
}

\setlist[itemize]{
  left=0.5\parindent,       % отступ слева
  itemsep=0pt,           % расстояние между пунктами
  topsep=5pt,            % отступ сверху
  partopsep=0pt,
  parsep=0pt
}

% Гиперссылки
\hypersetup{
  colorlinks=true,
  linkcolor=black,
  urlcolor=blue,
  pdfborder={0 0 0},
  pdftitle={PyQt6 приложение для управления устройствами умного дома},
  pdfauthor={Черкасов А.А.}
}

\begin{document}

\newpage
\thispagestyle{empty}
\begin{center}
  МИНИСТЕРСТВО НАУКИ И ВЫСШЕГО ОБРАЗОВАНИЯ РОССИЙСКОЙ ФЕДЕРАЦИИ ФЕДЕРАЛЬНОЕ ГОСУДАРСТВЕННОЕ БЮДЖЕТНОЕ ОБРАЗОВАТЕЛЬНОЕ УЧРЕЖДЕНИЕ ВЫСШЕГО ОБРАЗОВАНИЯ\\
  <<ВЯТСКИЙ ГОСУДАРСТВЕННЫЙ УНИВЕРСИТЕТ>>\\
  Институт математики и информационных систем\\
  Факультет автоматики и вычислительной техники\\
  Кафедра электронных вычислительных машин
\end{center}
\vspace{10mm}

\hfill
\begin{tabular}{l}
  \footnotesize Дата сдачи на проверку:                                          \\
  \footnotesize <<\rule[-1mm]{5mm}{0.10mm}\/>>\rule[-1mm]{20mm}{0.10mm}\ 2025 г. \\
  \footnotesize Проверено:                                                       \\
  \footnotesize <<\rule[-1mm]{5mm}{0.10mm}\/>>\rule[-1mm]{20mm}{0.10mm}\ 2025 г. \\
\end{tabular}
\vfill

\begin{center}
  Отчёт по лабораторной работе №10\\
  по дисциплине\\
  <<Информатика>>\\
  <<Пьезоэлемент, микросхемы>>
\end{center}
\vspace{25mm}
\noindent
\begin{tabular}{ll}
  Разработал студент гр. ИВТб-1301-05-00 & \hspace{18mm}\rule[-1mm]{30mm}{0.10mm}\,/Черкасов А. А./ \\
                                         & \hspace{25.5mm}\footnotesize(подпись)                    \\
  Проверил преподователь                 & \hspace{18mm}\rule[-1mm]{30mm}{0.10mm}\,/Шмакова Н.А./   \\
                                         & \hspace{25.5mm}\footnotesize(подпись)                    \\
\end{tabular}

\noindent
\begin{tabular}{lp{58mm}r}
  Работа защищена &  & \hspace{13mm}<<\rule[-1mm]{5mm}{0.10mm}\/>>\rule[-1mm]{30mm}{0.10mm}\ 2025 г.
\end{tabular}
\vfill

\begin{center}
  Киров\\
  2025
\end{center}

\newpage\thispagestyle{plain}

\section*{Цель}

Закрепить на практике знания о пьезоэлементе и микросхемах.

\section*{Задания}
\begin{enumerate}
  \item Азбука Морзе. Используйте мелодии для пьезодинамика из вложения (можно выбрать любые 2). Добавьте световую индикацию (не менее 3-х светодиодов). Добавьте кнопку для переключения мелодий.

  \item Терменвокс. При падении освещённости звук уменьшается. Звуковой сигнал непрерывен. Первая октава.

  \item Мерзкое пианино. 5 кнопок. Без резистора. Входной сигнал - switch/case. Малая октава.
  
  \item Перетягивание каната. 14 светодиодов. Стягивающий резистор. Сигнализация на светодиоде. 1 нажатие для следующего перехода.
\end{enumerate}

\newpage

\section*{Решение}
\section*{Задание 1}

\section*{Схема сборки на макетной плате}
Схема сборки на макетной плате представлена на рисунке 1.1.

\begin{figure}[H]
  \centering
  \includegraphics[width=0.9\linewidth]{pics/1.png}
  \caption*{Рисунок 1.1 --- Схема сборки задания №1}
\end{figure}

\newpage

\section*{Принципиальная схема}
Принципиальная схема представлена на рисунке 1.2.

\begin{figure}[H]
  \centering
  \includegraphics[width=1\linewidth]{pics/1п.png}
  \caption*{Рисунок 1.2 --- Принципиальная схема задания №1}
\end{figure}

\begin{center}
  \href{https://www.tinkercad.com/things/i78dwx4EMSs/editel?returnTo=%2Fdashboard%2Fdesigns%2Fcircuits&sharecode=kfmTSz69JLCleJXD6IXBhBt07unUbmD6bs8LCQvzxgk}
  {\fcolorbox{blue}{white}{\textcolor{blue}{Перейти на Tinkercad}}}
\end{center}

Исходный код решения задания представлен в приложении~\hyperref[app:A1]{A1}.

\newpage
\section*{Задание 2}
\section*{Схема сборки на макетной плате}
Схема сборки на макетной плате представлена на рисунке 2.1.

\begin{figure}[H]
  \centering
  \includegraphics[width=0.9\linewidth]{pics/2.png}
  \caption*{Рисунок 2.1 --- Схема сборки задания №2}
\end{figure}

\newpage

\section*{Принципиальная схема}
Принципиальная схема представлена на рисунке 2.2.

\begin{figure}[H]
  \centering
  \includegraphics[width=1\linewidth]{pics/2п.png}
  \caption*{Рисунок 2.2 --- Принципиальная схема задания №2}
\end{figure}


\begin{center}
  \href{https://www.tinkercad.com/things/9yOZ3FxwJwV-lab22/editel?returnTo=https%3A%2F%2Fwww.tinkercad.com%2Fdashboard%2Fdesigns%2Fcircuits&sharecode=U4rQQVLub2I8sGHTCo1-z_94pKYgY7H8c4cGdz57yFU}
  {\fcolorbox{blue}{white}{\textcolor{blue}{Перейти на Tinkercad}}}
\end{center}

Исходный код решения задания представлен в приложении~\hyperref[app:A2]{A2}.

\newpage

\section*{Решение}
\section*{Задание 3}
\section*{Схема сборки на макетной плате}
Схема сборки на макетной плате представлена на рисунке 3.1.

\begin{figure}[H]
  \centering
  \includegraphics[width=0.9\linewidth]{pics/3.png}
  \caption*{Рисунок 3.1 --- Схема сборки задания №3}
\end{figure}

\newpage

\section*{Принципиальная схема}
Принципиальная схема представлена на рисунке 3.2.

\begin{figure}[H]
  \centering
  \includegraphics[width=1\linewidth]{pics/3п.png}
  \caption*{Рисунок 3.2 --- Принципиальная схема задания №3}
\end{figure}

\begin{center}
  \href{https://www.tinkercad.com/things/0UmK0fQDE29/editel?returnTo=%2Fdashboard%2Fdesigns%2Fcircuits&sharecode=WAz08KPMvjl0xAE5G9v3ZGVyzoLG5ny51pZe7tmVJOw}
  {\fcolorbox{blue}{white}{\textcolor{blue}{Перейти на Tinkercad}}}
\end{center}

Исходный код решения задания представлен в приложении~\hyperref[app:A3]{A3}.

\newpage

\section*{Решение}
\section*{Задание 4}
\section*{Схема сборки на макетной плате}
Схема сборки на макетной плате представлена на рисунке 4.1.

\begin{figure}[H]
  \centering
  \includegraphics[width=0.9\linewidth]{pics/4.png}
  \caption*{Рисунок 4.1 --- Схема сборки задания №4}
\end{figure}

\newpage

\section*{Принципиальная схема}
Принципиальная схема представлена на рисунке 4.2 и 4.3.

\begin{figure}[H]
  \centering
  \includegraphics[width=1\linewidth]{pics/4п1.png}
  \caption*{Рисунок 4.2 --- Принципиальная схема задания №4}
\end{figure}

\begin{figure}[H]
  \centering
  \includegraphics[width=1\linewidth]{pics/4п2.png}
  \caption*{Рисунок 4.3 --- Принципиальная схема задания №4}
\end{figure}

\begin{center}
  \href{https://www.tinkercad.com/things/eVmUqwXKJUh/editel?returnTo=%2Fdashboard%2Fdesigns%2Fall&sharecode=LdqwTQxMZX7VRx0SvQpXEQ-Lc0u3yQt5HHMYTgrvZO8}
  {\fcolorbox{blue}{white}{\textcolor{blue}{Перейти на Tinkercad}}}
\end{center}

Исходный код решения задания представлен в приложении~\hyperref[app:A4]{A4}.

\newpage

\section*{Вывод}
В результате выполнения лабораторной работы была достигнута поставленная цель --- закрепление на практике знаний о пьезоэлементе и микросхемах на примере работы с платформой Arduino. Были успешно реализованы четыре различных задания, демонстрирующих разнообразные способы работы с пьезоэлектрическими излучателями и интегральными микросхемами:

\begin{enumerate}
  \item Азбука Морзе --- воспроизведение мелодий на пьезодинамике с одновременной световой индикацией на трех светодиодах и возможностью переключения между мелодиями с помощью кнопки;
  \item Терменвокс --- непрерывная генерация звука первой октавы с изменением громкости в зависимости от уровня освещенности;
  \item Мерзкое пианино --- пятикнопочное пианино малой октавы с использованием оператора switch/case для обработки входных сигналов без подтягивающих резисторов;
  \item Перетягивание каната --- игра с использованием 14 светодиодов и микросхемы для управления последовательностью включения, с сигнализацией о переходах.
\end{enumerate}

В ходе работы были изучены принципы работы пьезоэлектрических элементов для генерации звуковых сигналов, освоены методы управления интегральными микросхемами, а также получены практические навыки создания интерактивных устройств с использованием датчиков и исполнительных механизмов. Были использованы следующие ключевые функции Arduino:

tone(pin, frequency, duration) – генерация тонов заданной частоты и длительности

noTone(pin) – остановка генерации тона

analogRead(pin) – чтение аналогового значения с датчика

digitalRead(pin) – чтение состояния цифрового входа

digitalWrite(pin, value) – установка цифрового уровня сигнала

pinMode(pin, mode) – настройка режима работы пина

delay(ms) – задержка выполнения программы

Serial.begin(speed) – инициализация последовательного порта

\newpage
\section*{Приложение А1. Исходный код решения для задания №1}\label{app:A1}
\inputminted{cpp}{code/lab2_1.cpp}
\newpage
\section*{Приложение А2. Исходный код решения для задания №2}\label{app:A2}
\inputminted{cpp}{code/lab2_2.cpp}
\newpage
\section*{Приложение А3. Исходный код решения для задания №3}\label{app:A3}
\inputminted{cpp}{code/lab2_3.cpp}
\newpage
\section*{Приложение А4. Исходный код решения для задания №4}\label{app:A4}
\inputminted{cpp}{code/lab2_4.cpp}

% \section*{Приложение А5. Исходный код решения для дополнительного задания}
% \inputminted{cpp}{code/lab2_dop.cpp}

\end{document}
