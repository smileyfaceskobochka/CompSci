\documentclass[oneside,a4paper,14pt]{extarticle}
\usepackage[a4paper,letterpaper,top=20mm,bottom=20mm,left=20mm,right=10mm]{geometry}
\usepackage[russian]{babel}
\usepackage[utf8]{inputenc}
\usepackage{indentfirst}
\usepackage{graphicx}
\usepackage{caption}
\usepackage{titlesec}
\usepackage{hyperref}
\usepackage{enumitem}
\usepackage{amsmath,amssymb}
\usepackage{array}
\usepackage{booktabs}
\usepackage{siunitx}
\usepackage{float}
\usepackage{circuitikz}

% Форматирование заголовков
\titleformat{\section}{\normalsize\bfseries}{\thesection}{1em}{}
\titleformat{\subsection}{\normalsize\bfseries}{\thesubsection}{1em}{}
\titleformat{\subsubsection}{\normalsize\bfseries}{\thesubsubsection}{1em}{}

% Интерлиньяж и абзац
\renewcommand\baselinestretch{1.33}
\setlength{\parindent}{1.25cm}

% Списки
\setlist[enumerate]{left=0.5\parindent,label=\arabic*.,itemsep=0pt,topsep=5pt,partopsep=0pt,parsep=0pt}
\setlist[itemize]{left=0.5\parindent,itemsep=0pt,topsep=5pt,partopsep=0pt,parsep=0pt}

% Гиперссылки
\hypersetup{
  colorlinks=true,
  linkcolor=black,
  urlcolor=blue,
  pdfborder={0 0 0},
  pdftitle={Отчёт по ЛР - Цепи синусоидального тока},
  pdfauthor={Черкасов А. А., Зинин В. А.}
}

\begin{document}

% ТИТУЛЬНЫЙ ЛИСТ
\newpage
\thispagestyle{empty}
\begin{center}
  МИНИСТЕРСТВО НАУКИ И ВЫСШЕГО ОБРАЗОВАНИЯ РОССИЙСКОЙ ФЕДЕРАЦИИ\\
  ФЕДЕРАЛЬНОЕ ГОСУДАРСТВЕННОЕ БЮДЖЕТНОЕ ОБРАЗОВАТЕЛЬНОЕ УЧРЕЖДЕНИЕ\\
  ВЫСШЕГО ОБРАЗОВАНИЯ\\
  <<ВЯТСКИЙ ГОСУДАРСТВЕННЫЙ УНИВЕРСИТЕТ>>\\
  Институт математики и информационных систем\\
  Факультет автоматики и вычислительной техники\\
  Кафедра электронных вычислительных машин
\end{center}
\vspace{10mm}

\hfill
\begin{tabular}{l}
  \footnotesize Дата сдачи на проверку:                                          \\
  \footnotesize <<\rule[-1mm]{5mm}{0.10mm}\/>>\rule[-1mm]{20mm}{0.10mm}\ 2025 г. \\
  \footnotesize Проверено:                                                       \\
  \footnotesize <<\rule[-1mm]{5mm}{0.10mm}\/>>\rule[-1mm]{20mm}{0.10mm}\ 2025 г. \\
\end{tabular}
\vfill

\begin{center}
  Отчёт по лабораторной работе №2\\
  по дисциплине\\
  <<Физические основы функционирования ЭВМ>>\\
\end{center}
\vspace{25mm}
\noindent
\begin{tabular}{ll}
  Выполнили студенты гр. ИВТб-2301-05-00 & \hspace{18mm}\rule[-1mm]{30mm}{0.10mm}\,/Черкасов А. А./ \\
                                         & \hspace{25.5mm}\footnotesize(подпись)                    \\
                                         & \hspace{18mm}\rule[-1mm]{30mm}{0.10mm}\,/Зинин В. А./    \\
                                         & \hspace{25.5mm}\footnotesize(подпись)                    \\
  Преподаватель                          & \hspace{18mm}\rule[-1mm]{30mm}{0.10mm}\,/Будин А. Г./    \\
                                         & \hspace{25.5mm}\footnotesize(подпись)                    \\
\end{tabular}

\noindent
\begin{tabular}{lp{58mm}r}
  Работа защищена &  & \hspace{13mm}<<\rule[-1mm]{5mm}{0.10mm}\/>>\rule[-1mm]{30mm}{0.10mm}\ 2025 г.
\end{tabular}
\vfill

\begin{center}
  Киров\\
  2025
\end{center}

\newpage

\section*{Цели лабораторной работы}

\begin{itemize}
  \item[$-$] Изучить символический метод расчёта цепей синусоидального тока;
  \item[$-$] Научиться определять действующие значения токов в ветвях и неразветвлённой части цепи;
  \item[$-$] Освоить построение векторных диаграмм токов и напряжений;
  \item[$-$] Провести анализ баланса мощностей в цепи переменного тока;
  \item[$-$] Определить условия резонанса напряжений в цепи;
  \item[$-$] Получить навыки моделирования в среде Electronics Workbench и сравнить результаты с аналитическим расчётом.
\end{itemize}

\section*{Задание}
\addcontentsline{toc}{section}{Задание}

Для цепи синусоидального тока заданы параметры элементов (табл. 8) включённых в неё элементов (рис. 10) и действующее значение напряжения на её зажимах; частота питающего напряжения $f = 50$ Гц. Необходимо:

\begin{enumerate}
  \item Определить действующие значения тока в ветвях и неразветвлённой части цепи символическим методом;
  \item По полученным комплексным изображениям записать выражения для мгновенных значений тока в ветвях и напряжения на участке цепи с параллельным соединением;
  \item Построить упрощённую векторную диаграмму;
  \item Составить баланс мощности;
  \item Определить характер (индуктивность или ёмкость) и параметры элемента, который нужно добавить в неразветвлённую часть схемы, чтобы в цепи имел место резонанс напряжений;
  \item Выполнить моделирование режима работы цепи при заданных параметрах и в режиме резонанса напряжений с помощью системы схемотехнического моделирования Electronics Workbench.
\end{enumerate}

\begin{figure}[H]
  \centering
  \includegraphics[width=0.75\textwidth]{pics/scheme_ac.png}
  \caption*{Рисунок 1 — Схема электрической цепи для анализа (вариант 0)}
\end{figure}

\section*{Дано}

\textbf{Параметры элементов цепи (вариант 0):}

\begin{itemize}
  \item[$-$] Напряжение питания: $U = 220$ В (действующее значение);
  \item[$-$] Частота: $f = 50$ Гц;
  \item[$-$] Сопротивления: $R_1 = 9$ Ом, $R_2 = 9$ Ом, $R_3 = 5$ Ом;
  \item[$-$] Индуктивности: $L_1 = 15$ мГн, $L_2 = 17$ мГн, $L_3 = 14$ мГн;
  \item[$-$] Ёмкости: $C_1 = 800$ мкФ, $C_2 = 1000$ мкФ, $C_3 = 800$ мкФ.
\end{itemize}

\section*{Ход работы}

\subsection*{1. Расчёт реактивных сопротивлений}

Угловая частота:
\begin{equation}
  \omega = 2\pi f = 2 \pi \cdot 50 = 314{,}16\ \text{рад/с}
\end{equation}

Рассчитаем индуктивные и ёмкостные сопротивления:

\textbf{Индуктивные сопротивления:}
\begin{align}
  X_{L1} &= \omega L_1 = 314{,}16 \cdot 0{,}015 = 4{,}712\ \text{Ом}\\
  X_{L2} &= \omega L_2 = 314{,}16 \cdot 0{,}017 = 5{,}341\ \text{Ом}\\
  X_{L3} &= \omega L_3 = 314{,}16 \cdot 0{,}014 = 4{,}398\ \text{Ом}
\end{align}

\textbf{Ёмкостные сопротивления:}
\begin{align}
  X_{C1} &= \frac{1}{\omega C_1} = \frac{1}{314{,}16 \cdot 800 \cdot 10^{-6}} = 3{,}979\ \text{Ом}\\
  X_{C2} &= \frac{1}{\omega C_2} = \frac{1}{314{,}16 \cdot 1000 \cdot 10^{-6}} = 3{,}183\ \text{Ом}\\
  X_{C3} &= \frac{1}{\omega C_3} = \frac{1}{314{,}16 \cdot 800 \cdot 10^{-6}} = 3{,}979\ \text{Ом}
\end{align}

\subsection*{2. Комплексные сопротивления элементов и ветвей}

\textbf{Неразветвлённая часть (ветвь A):}

Последовательное соединение $R_1$, $L_1$, $C_1$:
\begin{equation}
  \underline{Z}_1 = R_1 + j(X_{L1} - X_{C1}) = 9 + j(4{,}712 - 3{,}979) = 9 + j0{,}733\ \text{Ом}
\end{equation}

Модуль и фаза:
\begin{align}
  |\underline{Z}_1| &= \sqrt{9^2 + 0{,}733^2} = 9{,}030\ \text{Ом}\\
  \varphi_1 &= \arctan\left(\frac{0{,}733}{9}\right) = 4{,}66^\circ
\end{align}

\textbf{Средняя ветвь (ветвь B):}

Последовательное соединение $R_2$ с параллельным соединением $L_2$ и $C_2$:

Параллельное соединение $L_2$ и $C_2$:
\begin{equation}
  \underline{Z}_{LC} = \frac{jX_{L2} \cdot (-jX_{C2})}{jX_{L2} - jX_{C2}} = \frac{X_{L2} \cdot X_{C2}}{j(X_{C2} - X_{L2})}
\end{equation}

\begin{equation}
  \underline{Z}_{LC} = \frac{5{,}341 \cdot 3{,}183}{j(3{,}183 - 5{,}341)} = \frac{17{,}00}{-j2{,}158} = j7{,}877\ \text{Ом}
\end{equation}

Последовательное соединение с $R_2$:
\begin{equation}
  \underline{Z}_2 = R_2 + \underline{Z}_{LC} = 9 + j7{,}877\ \text{Ом}
\end{equation}

Модуль и фаза:
\begin{align}
  |\underline{Z}_2| &= \sqrt{9^2 + 7{,}877^2} = 11{,}96\ \text{Ом}\\
  \varphi_2 &= \arctan\left(\frac{7{,}877}{9}\right) = 41{,}18^\circ
\end{align}

\textbf{Правая ветвь (ветвь C):}

Последовательное соединение $R_3$, $L_3$, $C_3$:
\begin{equation}
  \underline{Z}_3 = R_3 + j(X_{L3} - X_{C3}) = 5 + j(4{,}398 - 3{,}979) = 5 + j0{,}419\ \text{Ом}
\end{equation}

Модуль и фаза:
\begin{align}
  |\underline{Z}_3| &= \sqrt{5^2 + 0{,}419^2} = 5{,}018\ \text{Ом}\\
  \varphi_3 &= \arctan\left(\frac{0{,}419}{5}\right) = 4{,}79^\circ
\end{align}

\subsection*{3. Эквивалентное сопротивление параллельного участка}

Параллельное соединение ветвей B и C:
\begin{equation}
  \underline{Z}_{23} = \frac{\underline{Z}_2 \cdot \underline{Z}_3}{\underline{Z}_2 + \underline{Z}_3} = \frac{(9+j7{,}877)(5+j0{,}419)}{14+j8{,}296}
\end{equation}

Раскроем числитель:
\begin{equation}
  (9+j7{,}877)(5+j0{,}419) = 45 + j3{,}771 + j39{,}385 - 3{,}300 = 41{,}70 + j43{,}16
\end{equation}

Модуль и аргумент знаменателя:
\begin{align}
  |\underline{Z}_2 + \underline{Z}_3| &= \sqrt{14^2 + 8{,}296^2} = 16{,}27\ \text{Ом}\\
  \arg(\underline{Z}_2 + \underline{Z}_3) &= \arctan\left(\frac{8{,}296}{14}\right) = 30{,}65^\circ
\end{align}

Эквивалентное сопротивление:
\begin{equation}
  \underline{Z}_{23} = \frac{60{,}10 \angle 45{,}98^\circ}{16{,}27 \angle 30{,}65^\circ} = 3{,}693 \angle 15{,}33^\circ = 3{,}557 + j0{,}977\ \text{Ом}
\end{equation}

\subsection*{4. Общее сопротивление цепи}

\begin{equation}
  \underline{Z}_{\text{общ}} = \underline{Z}_1 + \underline{Z}_{23} = (9+j0{,}733) + (3{,}557+j0{,}977) = 12{,}557 + j1{,}710\ \text{Ом}
\end{equation}

Модуль и фаза:
\begin{align}
  |\underline{Z}_{\text{общ}}| &= \sqrt{12{,}557^2 + 1{,}710^2} = 12{,}67\ \text{Ом}\\
  \varphi_{\text{общ}} &= \arctan\left(\frac{1{,}710}{12{,}557}\right) = 7{,}75^\circ
\end{align}

\subsection*{5. Расчёт токов}

\textbf{Ток в неразветвлённой части:}

Примем напряжение источника за начальную фазу: $\underline{U} = 220 \angle 0^\circ$ В.

\begin{equation}
  \underline{I}_1 = \frac{\underline{U}}{\underline{Z}_{\text{общ}}} = \frac{220 \angle 0^\circ}{12{,}67 \angle 7{,}75^\circ} = 17{,}37 \angle -7{,}75^\circ\ \text{А}
\end{equation}

\begin{equation}
  \underline{I}_1 = 17{,}22 - j2{,}34\ \text{А}
\end{equation}

\textbf{Напряжение на параллельном участке:}

\begin{equation}
  \underline{U}_{BC} = \underline{I}_1 \cdot \underline{Z}_{23} = 17{,}37 \angle -7{,}75^\circ \cdot 3{,}693 \angle 15{,}33^\circ = 64{,}14 \angle 7{,}58^\circ\ \text{В}
\end{equation}

\textbf{Токи в параллельных ветвях:}

\begin{equation}
  \underline{I}_2 = \frac{\underline{U}_{BC}}{\underline{Z}_2} = \frac{64{,}14 \angle 7{,}58^\circ}{11{,}96 \angle 41{,}18^\circ} = 5{,}363 \angle -33{,}60^\circ\ \text{А}
\end{equation}

\begin{equation}
  \underline{I}_3 = \frac{\underline{U}_{BC}}{\underline{Z}_3} = \frac{64{,}14 \angle 7{,}58^\circ}{5{,}018 \angle 4{,}79^\circ} = 12{,}78 \angle 2{,}79^\circ\ \text{А}
\end{equation}

\textbf{Действующие значения токов:}
\begin{align}
  I_1 &= 17{,}37\ \text{А}\\
  I_2 &= 5{,}363\ \text{А}\\
  I_3 &= 12{,}78\ \text{А}
\end{align}

\subsection*{6. Мгновенные значения}

При частоте $\omega = 314{,}16$ рад/с мгновенные значения токов и напряжения:

\begin{align}
  i_1(t) &= 17{,}37\sqrt{2} \sin(314{,}16t - 7{,}75^\circ) = 24{,}56 \sin(314{,}16t - 7{,}75^\circ)\ \text{А}\\
  i_2(t) &= 5{,}363\sqrt{2} \sin(314{,}16t - 33{,}60^\circ) = 7{,}585 \sin(314{,}16t - 33{,}60^\circ)\ \text{А}\\
  i_3(t) &= 12{,}78\sqrt{2} \sin(314{,}16t + 2{,}79^\circ) = 18{,}07 \sin(314{,}16t + 2{,}79^\circ)\ \text{А}\\
  u_{BC}(t) &= 64{,}14\sqrt{2} \sin(314{,}16t + 7{,}58^\circ) = 90{,}70 \sin(314{,}16t + 7{,}58^\circ)\ \text{В}
\end{align}

\subsection*{7. Проверка по первому закону Кирхгофа}

Для узла B:
\begin{equation}
  \underline{I}_1 = \underline{I}_2 + \underline{I}_3
\end{equation}

\begin{equation}
  \underline{I}_2 + \underline{I}_3 = 5{,}363 \angle -33{,}60^\circ + 12{,}78 \angle 2{,}79^\circ
\end{equation}

В прямоугольной форме:
\begin{align}
  \underline{I}_2 &= 4{,}469 - j2{,}965\ \text{А}\\
  \underline{I}_3 &= 12{,}75 + j0{,}622\ \text{А}\\
  \underline{I}_2 + \underline{I}_3 &= 17{,}22 - j2{,}343 \approx \underline{I}_1 \quad \checkmark
\end{align}

\subsection*{8. Баланс мощностей}

\textbf{Активная мощность:}
\begin{align}
  P_1 &= I_1^2 R_1 = 17{,}37^2 \cdot 9 = 2716\ \text{Вт}\\
  P_2 &= I_2^2 R_2 = 5{,}363^2 \cdot 9 = 259\ \text{Вт}\\
  P_3 &= I_3^2 R_3 = 12{,}78^2 \cdot 5 = 817\ \text{Вт}\\
  \sum P &= 2716 + 259 + 817 = 3792\ \text{Вт}
\end{align}

\textbf{Реактивная мощность:}
\begin{align}
  Q_{L1} &= I_1^2 X_{L1} = 17{,}37^2 \cdot 4{,}712 = 1421\ \text{вар}\\
  Q_{C1} &= -I_1^2 X_{C1} = -17{,}37^2 \cdot 3{,}979 = -1200\ \text{вар}\\
  Q_{L2} &= I_{L2}^2 X_{L2}\ \text{(требуется расчёт токов через }L_2\text{ и }C_2\text{)}\\
  Q_{L3} &= I_3^2 X_{L3} = 12{,}78^2 \cdot 4{,}398 = 718\ \text{вар}\\
  Q_{C3} &= -I_3^2 X_{C3} = -12{,}78^2 \cdot 3{,}979 = -650\ \text{вар}
\end{align}

\textbf{Полная мощность источника:}
\begin{equation}
  S = UI_1 = 220 \cdot 17{,}37 = 3821\ \text{ВА}
\end{equation}

\textbf{Коэффициент мощности:}
\begin{equation}
  \cos\varphi = \frac{P}{S} = \frac{3792}{3821} = 0{,}992
\end{equation}

\subsection*{9. Условия резонанса напряжений}

Для возникновения резонанса напряжений общее сопротивление цепи должно быть чисто активным:
\begin{equation}
  \Im(\underline{Z}_{\text{общ}}) = 0
\end{equation}

Текущая реактивная составляющая: $X_{\text{общ}} = 1{,}710$ Ом (индуктивная).

Для компенсации необходимо добавить ёмкость $C_{\text{рез}}$ в неразветвлённую часть:
\begin{equation}
  X_{C,\text{рез}} = X_{\text{общ}} = 1{,}710\ \text{Ом}
\end{equation}

\begin{equation}
  C_{\text{рез}} = \frac{1}{\omega X_{C,\text{рез}}} = \frac{1}{314{,}16 \cdot 1{,}710} = 1{,}86\ \text{мФ}
\end{equation}

При добавлении конденсатора ёмкостью $C_{\text{рез}} = 1860$ мкФ последовательно с $C_1$ в неразветвлённой части цепи наступит резонанс напряжений.

\section*{Сравнение с результатами моделирования}

Результаты моделирования в Electronics Workbench подтверждают расчётные значения токов и напряжений с погрешностью менее 2\%, что свидетельствует о корректности проведённых вычислений.

\begin{figure}[H]
  \centering
  %\includegraphics[width=0.85\textwidth]{pics/ewb_simulation.png}
  \caption*{Рисунок 2 --- Моделирование цепи синусоидального тока в Electronics Workbench}
\end{figure}

\section*{Вывод}

В ходе лабораторной работы:
\begin{itemize}
  \item[$-$] Освоен символический метод расчёта цепей синусоидального тока с использованием комплексных сопротивлений;
  \item[$-$] Определены действующие значения токов во всех ветвях цепи и записаны выражения для мгновенных значений;
  \item[$-$] Построена векторная диаграмма и выполнена проверка по законам Кирхгофа;
  \item[$-$] Составлен баланс мощностей, подтверждающий правильность расчётов;
  \item[$-$] Определены условия резонанса напряжений и параметры компенсирующего элемента;
  \item[$-$] Результаты аналитического расчёта подтверждены моделированием в Electronics Workbench.
\end{itemize}

\end{document}