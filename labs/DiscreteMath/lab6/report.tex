\documentclass[oneside,a4paper,14pt]{extarticle}
\usepackage[a4paper,letterpaper,top=20mm,bottom=20mm,left=20mm,right=10mm]{geometry}
\usepackage[russian]{babel}
\usepackage{indentfirst}
\usepackage{amsmath}
\usepackage{amsfonts}
\usepackage{amsthm}
\usepackage{graphicx}
\usepackage{caption}
\usepackage{titlesec}
\usepackage{minted, fancyvrb}

\titleformat{\section} {\normalsize\bfseries} {\thesection} {1em} {}
\titleformat{\subsection} {\normalsize\bfseries} {\thesubsection} {1em} {}
\titleformat{\subsubsection} {\normalsize\bfseries} {\thesubsection} {1em} {}
\renewcommand\baselinestretch{1.45}\normalsize
\setlength{\parindent}{1.25cm}

\begin{document}

\newpage
\thispagestyle{empty}
\begin{center}
	МИНИСТЕРСТВО НАУКИ И ВЫСШЕГО ОБРАЗОВАНИЯ РОССИЙСКОЙ ФЕДЕРАЦИИ ФЕДЕРАЛЬНОЕ ГОСУДАРСТВЕННОЕ БЮДЖЕТНОЕ ОБРАЗОВАТЕЛЬНОЕ УЧРЕЖДЕНИЕ ВЫСШЕГО ОБРАЗОВАНИЯ\\
	«ВЯТСКИЙ ГОСУДАРСТВЕННЫЙ УНИВЕРСИТЕТ»\\
	Институт математики и информационных систем\\
	Факультет автоматики и вычислительной техники\\
	Кафедра электронных вычислительных машин
\end{center}
\vspace{10mm}

\hfill
\begin{tabular}{l}
  \footnotesize Дата сдачи на проверку: \\
  \footnotesize <<\rule[-1mm]{5mm}{0.10mm}\/>>\rule[-1mm]{20mm}{0.10mm}\ 2025 г.\\
  \footnotesize Проверено: \\
  \footnotesize <<\rule[-1mm]{5mm}{0.10mm}\/>>\rule[-1mm]{20mm}{0.10mm}\ 2025 г. \\
\end{tabular}
\vfill

\begin{center}
  ГРАФЫ. ПОИСК ПУТЕЙ В ГРАФАХ.\\
	Отчёт по лабораторной работе №6\\
	по дисциплине\\
	<<Дискретная математика>>\\
\end{center}
\vspace{25mm}
\noindent
\begin{tabular}{ll}
	Разработал студент гр. ИВТб-1301-05-00 & \rule[-1mm]{30mm}{0.10mm}\,/Черкасов А. А./   \\
	                                       & \hspace{8mm}\footnotesize(подпись)            \\
	Проверила преподаватель                & \rule[-1mm]{30mm}{0.10mm}\,/Пахарева И. В./ \\
	                                       & \hspace{8mm}\footnotesize(подпись)            \\
\end{tabular}

\noindent
  \begin{tabular}{lp{58mm}r}
    Работа защищена &  & <<\rule[-1mm]{5mm}{0.10mm}\/>>\rule[-1mm]{30mm}{0.10mm}\ 2025 г.
  \end{tabular}
  \vfill

\begin{center}
	Киров\\
	2025
\end{center}

\newpage\thispagestyle{plain}

\section*{Цель}

Цель работы: изучить методы представления ориентированного графа в виде матрицы инцидентности и освоить приёмы полного перебора путей с помощью поиска в глубину. На их основе разработать программу, способную находить все ориентированные цепи, ведущие в заданную вершину, без повторения дуг.
\section*{Задание}
\begin{itemize}
	\item[$-$] Ориентированный граф \(G\) задаётся матрицей инцидентности, записанной в файле \texttt{input.txt}. Ограничить размерность: число вершин и число дуг (ребер) \(\ge 5\).
	\item[$-$] Найти возможные ориентированные цепи (дуги в пути не должны повторяться) в вершину, номер которой вводится с клавиатуры.
\end{itemize}

\section*{Решение}

% Схема алгоритма решения представлена на рисунках 1.1 и 1.2. Примеры работы программы будут показаны в разделе «Результаты» вместе с графом, построенным по заданной матрице инцидентности и матрицей достижимости.

\clearpage

\section*{Вывод}

В результате выполнения лабораторной работы была создана и протестирована программа на языке C, которая:

\begin{itemize}
    \item[$-$] считывает из файла \texttt{input.txt} матрицу инцидентности ориентированного графа размером не менее 5 вершин и дуг;
    \item[$-$] запрашивает у пользователя номер целевой вершины и проверяет корректность ввода;
    \item[$-$] выполняет поиск в глубину (DFS) с учётом запрета на повторное использование дуг для полного перебора всех ориентированных цепей, заканчивающихся в заданной вершине;
    \item[$-$] формирует и выводит на экран список найденных путей в удобочитаемом виде.
\end{itemize}

\setminted{style = rainbow_dash, fontsize = \small} % https://pygments.org/styles/

\clearpage
\section*{Приложение А1. Исходный код}
% \inputminted{cpp}{code/src/main.c}


\end{document}